%+=+=+=+=+=+=+=+=+=+=+=+=+=+=+=+=+=+=+=+=+=+=+=+=+=+=+=+=+=+=+=+=+=+=+=+= PART
%+=+=+=+=+=+=+=+=+=+=+=+=+=+=+=+=+=+=+=+=+=+=+=+=+=+=+=+=+=+=+=+=+=+=+=+= PART
%+=+=+=+=+=+=+=+=+=+=+=+=+=+=+=+=+=+=+=+=+=+=+=+=+=+=+=+=+=+=+=+=+=+=+=+= PART
%+=+=+=+=+=+=+=+=+=+=+=+=+=+=+=+=+=+=+=+=+=+=+=+=+=+=+=+=+=+=+=+=+=+=+=+= PART
%+=+=+=+=+=+=+=+=+=+=+=+=+=+=+=+=+=+=+=+=+=+=+=+=+=+=+=+=+=+=+=+=+=+=+=+= PART
%+=+=+=+=+=+=+=+=+=+=+=+=+=+=+=+=+=+=+=+=+=+=+=+=+=+=+=+=+=+=+=+=+=+=+=+= PART
%+=+=+=+=+=+=+=+=+=+=+=+=+=+=+=+=+=+=+=+=+=+=+=+=+=+=+=+=+=+=+=+=+=+=+=+= PART
%+=+=+=+=+=+=+=+=+=+=+=+=+=+=+=+=+=+=+=+=+=+=+=+=+=+=+=+=+=+=+=+=+=+=+=+= PART
\newpage
\phantomsection\addcontentsline{toc}{section}{SUPPLEMENT III bis -- ELDRITCH WIZARDRY}\begin{center}
{\Huge \ODDtitlefont{DONJONS \& DRAGONS}}{\normalsize \textsuperscript{\sffamily\textregistered}}

\vspace{1.8cm}

{\Large \textbf{Supplément III bis}}

\vspace{1.3cm}

{\Huge \ODDtitlebisfont{ELDRITCH}}

\vspace{0.3cm}

{\Huge \ODDtitlebisfont{WIZARDRY}}

\vspace{2.0cm}

{\Large \textbf{MAGIE ANCIENNE ET PUISSANTE}}

\vspace{0.5cm}

{\Large \textbf{POUVOIRS PSIONIQUES}}

\vspace{1cm}

{\large PAR

\vspace{0.1cm}

GARY GYGAX \& BRIAN BLUME \& ROUBOUDOU}
\end{center}

\newpage
%======================Blank page
\phantom{-}
\newpage
%==========================================================================SECTION
\phantomsection\section*{Hommes \& Magie}
\addcontentsline{toc}{section}{Hommes \& Magie}

\begin{center}
\textbf{[POUVOIRS PSIONIQUES]}
\end{center}

%----------------------------------------------------- SUB SECTION
\phantomsection\subsection*{INTRODUCTION}
\addcontentsline{toc}{subsection}{INTRODUCTION}

Les pouvoirs psioniques sont divisés en deux catégories :

\bigskip

\begin{itemize}
\item Les aptitudes psioniques, des compétences psioniques ressemblant parfois à des sorts ;
\item Les modes d'attaques et de défenses psioniques, servant dans le combat psionique.
\end{itemize}

\bigskip

Chaque pouvoir, lorsqu'il est utilisé, aura généralement un coût en points de \textbf{Force Psionique (FP)}.

L'acquisition des modes d'attaques et de défense psioniques est liée à l'acquisition des aptitudes psioniques. Dans la partie qui suit, nous détaillons d'abord l'acquisition des aptitudes puis celle des modes d'attaques et de défense psioniques avant de passer au combat psionique à proprement parler.

%----------------------------------------------------- SUB SECTION
\phantomsection\subsection*{CONDITIONS D'ACCES AUX POUVOIRS PSIONIQUES}
\addcontentsline{toc}{subsection}{CONDITIONS D'ACCES AUX POUVOIRS PSIONIQUES}

Les conditions suivantes doivent être remplies :

\bigskip

\begin{itemize}
\item Etre humain,
\item Pas de restriction de classe\footnote{L'édition originale indique que les moines ou les druides ne peuvent pas avoir de pouvoirs mentaux, sans que cette restriction ne soit vraiment expliquée. Plus loin, dans la description des aptitudes psychiques, les aptitudes sont données pour les clercs, les moines et les druides. Nous avons donc choisi, dans cette synthèse, de ne pas retenir la restriction de classes.}
\item Avoir un score non modifié de 15 en Intelligence, Sagesse ou Charisme,
\item Faire plus de 90 (91--100) sur un jet de pourcentage (une chance sur 10).
\end{itemize}

\bigskip

Les personnages ayant des pouvoirs psioniques deviennent sensibles aux monstres ayant des capacités psioniques.

%----------------------------------------------------- SUB SECTION
\phantomsection\subsection*{POTENTIEL PSYCHIQUE}
\addcontentsline{toc}{subsection}{POTENTIEL PSYCHIQUE}

Lorsque le personnage a accès pour la première fois aux pouvoirs psioniques, le joueur fait un jet de 1d100 pour déterminer le potentiel psychique (voir table ci-dessous).
\bigskip

{\parindent3cm\begin{tabular}{cl}
\textbf{Potentiel psychique} & \textbf{Chance de gagner une aptitude)} \\
01--10 & niveau x 4\% \\
11--25 & niveau x 5\% \\
26--50 & niveau x 6\% \\
51--75 & niveau x 10\% \\
76--90 & niveau x 11\% \\
91--99 & niveau x 12\% \\
\hspace{0.4cm}00 & niveau x 13\% \\
\end{tabular}}

%----------------------------------------------------- SUB SECTION
\phantomsection\subsection*{APTITUDES PSIONIQUES}
\addcontentsline{toc}{subsection}{APTITUDES PSIONIQUES}

%- - - - - - - - - - - - - - - - - - - - - - - - - - - SUB SUB SECTION
\phantomsection\subsubsection*{\uline{\textit{Gain d'aptitudes}}}
\addcontentsline{toc}{subsubsection}{Gain d'aptitudes}
\label{aptitudes-gain}

Le joueur peut tenter de gagner une ou deux aptitudes psioniques dans deux conditions :

\bigskip

\begin{itemize}
\item Il vient de calculer son potentiel psychique et a accès pour la première fois aux pouvoirs psioniques ;
\item Il a déjà accès aux pouvoirs psioniques et change de niveau.
\end{itemize}

\bigskip

Dans les deux cas, le joueur doit lancer 1d100 sous sa chance de gagner une aptitude psionique. Si le jet de 1d100 sous la Chance de gagner une aptitude psionique est réussi :

\bigskip

\begin{enumerate}
\item Le joueur lance 1d20 et consulte la table relative à sa classe ; il rejoue si :
\begin{itemize}
\item L'aptitude déterminée par le jet est déjà en sa possession.
\item Avec cette aptitude, il dispose de plus d'aptitudes supérieures que d'aptitudes basiques, ce qui ne doit pas arriver.
\end{itemize}
\item Le joueur lance de nouveau 1d100, mais sous son Potentiel Psychique cette fois ; si le jet est réussi, il gagne une seconde aptitude psionique.
\item Il note sur son annexe à la feuille de personnage à quel niveau il a reçu son ou ses aptitudes. \end{enumerate}

\bigskip

Note : si la chance de gagner une aptitude psionique est supérieure à 100\%, alors le joueur peut choisir librement son aptitude.

%- - - - - - - - - - - - - - - - - - - - - - - - - - - SUB SUB SECTION
\phantomsection\subsubsection*{\uline{\textit{Niveau de maîtrise}}}
\addcontentsline{toc}{subsubsection}{Niveau de maîtrise}
\label{niveau-maitrise}

Le niveau de maîtrise d'une aptitude est définie par la différence entre le niveau du personnage et le niveau auquel il a acquis l'aptitude + 1.

\bigskip

Par exemple, un guerrier de niveau  6 qui aurait acquis l'aptitude "Réduction" alors qu'il était au niveau 4, aurait un niveau de maîtrise de cette aptitude de 3 (6 - 4 + 1).

\bigskip

Le niveau de maîtrise est aussi important pour les modes d'attaques et de défenses car il influe sur leur portée.


%- - - - - - - - - - - - - - - - - - - - - - - - - - - SUB SUB SECTION
\phantomsection\subsubsection*{\uline{\textit{Option 1 : Utilisation des catégories d'aptitudes psioniques par catégorie}}}
\addcontentsline{toc}{subsubsection}{Option 1 : Utilisation des catégories d'aptitudes psioniques par catégorie}

Les règles originales indiquent : \og Dans la sélection aléatoire, il est suggéré de mettre un poids supérieur aux probabilités de gain d’aptitudes liées à des aptitudes déjà possédées, par exemple Empathie augmenterait les chances de gagner Perception extrasensorielle, Télépathie avec les animaux et Projection télépathique\fg{}.

\bigskip

Cette section propose une règle optionnelle\footnote{\og Homebrew rule \fg{} comme diraient les américains.} pouvant être activée dès le choix de la deuxième aptitude. En revanche, pour le choix de la première aptitude, le joueur doit lancer 1d20 et ignorer cette option dans tous les cas.

\bigskip

Le fait de trier les aptitudes psioniques par catégories permet de restreindre la détermination aléatoire de l'aptitude à un sous-ensemble des aptitudes possibles de la même catégorie. Nous avons classé les aptitudes psioniques en quatre catégories.

\bigskip

\begin{tabular}{p{5.5cm}p{5.5cm}p{5.5cm}}
\textbf{A -- Contrôle de la matière} & \textbf{B -- Contrôle de l'esprit}   & \textbf{D -- Voyage psychique} \\
\textbf{et de l'énergie}             & \textbf{via l'esprit}                & Forme éthérée \\
Agitation moléculaire                & Altération de l'aura                 & Marche dimensionnelle \\
Ajustement cellulaire               & Barrière de l'esprit                 & Porte dimensionnelle \\
Altération de la forme              & Domination                           & Projection astrale \\
Changer le poids du corps           & Domination des masses                & Téléportation \\
Contrôle de l'énergie               & Empathie                             &  Voyage probabiliste\\
Contrôle de l'esprit sur le corps   & Hypnose & \\
Contrôle du corps                   & Perception extrasensorielle & \\
Corps comme arme                    & Projection télépathique &\\
Expansion                           & Télépathie avec les animaux &\\
Hibernation                         & \textbf{C -- Sens} & \\
Invisibilité                        & Clairaudience & \\
Lévitation                          & Clairvoyance & \\
Manipulation moléculaire            & Détection de la magie & \\
Réarrangement moléculaire           & Détection du Mal/du Bien & \\
Réduction                           & Prémonition & \\
Télékinésie                && \\
\end{tabular}

\bigskip

Les catégories sont :

\bigskip

\begin{itemize}
\item[A :] Contrôle de la matière et de l'énergie,
\item[B :] Contrôle de son propre esprit et de l'esprit des autres via l'esprit,
\item[C :] Développement des sens,
\item[D :] Voyage psychique de l'esprit et/ou du corps.
\end{itemize}

\bigskip

Avec cette option, au lieu de lancer 1d20 (première colonne des tables ci-après), l'arbitre peut lancer le dé correspondant à la catégorie dans laquelle le personnage possède déjà une ou des aptitudes (deuxième colonne de chaque tableau). Après chaque table d'aptitude par classe, un tableau indique le type de jet à faire.

%- - - - - - - - - - - - - - - - - - - - - - - - - - - SUB SUB SECTION
\phantomsection\subsubsection*{\uline{\textit{Option 2 : Détermination des aptitudes psioniques par changement des probabilités}}}
\addcontentsline{toc}{subsubsection}{Option 2 : Détermination des aptitudes psioniques par changement des probabilités}

Si l'on lit la règle originale avec attention, on peut voir que l'option 1 ne répond pas tout à fait à la règle. En effet, la règle ne dit pas de cantonner le choix dans les aptitudes d'une même catégorie, mais de donner plus de probabilités à un choix dans la même catégorie que l'une des aptitudes déjà acquises.

\bigskip

Il est difficile de donner une méthode générique pour résoudre ce point qui s'inscrit parfaitement dans la tradition d'interprétation des règles de \texttt{OD\&D}. Nous pouvons illustrer une méthode mais le MD devra l'ignorer ou l'adapter le cas échéant.

\bigskip

Prenons un magicien ayant déjà eu un pouvoir psionique dans la catégorie C. Lors de son passage au niveau supérieur, il réussi son jet de 1d100 sous la Chance de gagner une aptitude psionique. On peut voir que 5 aptitudes font partie de la catégorie C sur 18. Il reste donc 13 aptitudes faisant partie des autres catégories. De plus, parmi les 5, le magicien en possède une (mais nous négligerons ce fait).

\bigskip

Il est possible d'augmenter les probabilités des aptitudes de la classe C (mais en gardant les autres aptitudes comme choix potentiels) en utilisant, par exemple 1d30\footnote{Si vous n'avez pas de d30, prenez 1d6 \& 1d10 et lancez les deux. Le d6 vous donnera les dizaines avec la règles suivante : 1--2 donne 0, 3--4 donne 1 et 5--6 donne 2. Le d10 vous donnera les unités avec 0 comptant pour 10.} au lieu de 1d20. De 1 à 18, vous pouvez prendre les entrées de la première colonne des aptitudes et ajouter 2 fois chaque élément de C : 19 et 20 donnent C1, 21 et 22 donnent C2, ..., 27 et 28 donnent C5 et le joueur rejoue s'il fait 29 ou 30. Dans cet exemple, les aptitudes de la catégorie C ont 3 fois plus de chances de sortir que les autres.

\bigskip

Notez que si la catégorie est très représentée (comme la catégorie A dans le cas du guerrier), il est possible de ne proposer que deux occurrences de la catégorie A (au lieu de 3 dans l'exemple précédent). Il faudrait alors sans doute prendre 1d40\footnote{Même principe que le d30 avec 1d4 (comptant de 0 à 3) et 1d10 (comptant de 1 à 10).}. De 21 à 33 réapparaîtraient les choix de A:1 à A:13 et le jet serait rejoué si son résultat était supérieur ou égal à 34.

\bigskip

Si l'on voulait faire apparaître les choix de A trois fois plus que les autres (comme dans le cas du magicien ci-dessus), il faudrait utiliser un 1d50\footnote{Même principe que le d30 avec 1d10 (1--2 donnent 0, ..., 9--10 donne 4) et 1d10 (comptant de 1 à 10).} et rejouer si le d50 faisait strictement plus que 46.

\bigskip

Chaque table ci-dessous est munie deux colonnes, une avec les scores du d20 et l'autre avec une numérotation des pouvoirs par catégorie. Bien entendu, le MD reste seul juge de la bonne méthode à appliquer.

%- - - - - - - - - - - - - - - - - - - - - - - - - - - SUB SUB SECTION
\phantomsection\subsubsection*{\uline{\textit{Notion de nombre de niveaux}}}
\addcontentsline{toc}{subsubsection}{Notion de nombre de niveaux}

\texttt{OD\&D} utilise fréquemment la notion de "nombre de niveaux", que ce soit pour les PJs ou pour les monstres. Cette notion est souvent implicite dans les règles originales et il n'est, le plus souvent, possible de savoir quand elle s'applique qu'au travers des exemples donnés.

\bigskip

Ainsi, quand une aptitude est dite progresser \og par niveau \fg{}, il faut déterminer si elle progresse de N \og par niveau \fg{} ou si elle progresse de N par \textit{nombre de niveaux} :

\bigskip

\begin{tabular}{ccp{3cm}ccc}
\textbf{Niveau} & && \textbf{Niveau} & \textbf{Nombre de} & \\
\textbf{du PJ} & \textbf{Score}&& \textbf{du PJ}  & \textbf{niveaux du PJ}& \textbf{Score} \\
1               & N             && 1                & 1                         & N  \\
2               & 2 x N         && 2                & 3=1+2                     & 3 x N \\
3               & 3 x N         && 3                & 6=1+2+3                   & 6 x N \\
...             & ...           && ...              & ...                       & ... \\
\end{tabular}

\bigskip

Les aptitudes psioniques ont des façons de progresser, parfois par niveaux, parfois par nombre de niveaux. De plus, la plupart du temps, le niveau qui compte, dans la partie pouvoirs psioniques, n'est pas le niveau du personnage mais le niveau de maîtrise de l'aptitude (voir page \pageref{niveau-maitrise}).

\bigskip

Certaines aptitudes ont aussi un impact sur un \og nombre de niveaux de monstres \fg{}, nombre qui se calcule avec la même logique que le tableau ci-dessus de droite en remplaçant les niveaux par les dés de vie des monstres.

\bigskip

Dans la suite du document, nous indiquerons de manière explicite si la progression se fait par niveau ou par nombre de niveaux.

\newpage
%- - - - - - - - - - - - - - - - - - - - - - - - - - - SUB SUB SECTION
\phantomsection\subsubsection*{\uline{\textit{Aptitudes psioniques pour les Guerriers, Paladins, Rangers, Voleurs et Assassins}}}
\addcontentsline{toc}{subsubsection}{Aptitudes psioniques pour les Guerriers, Paladins, Rangers, Voleurs et Assassins}

\begin{tabular}{cclcc}
\textbf{1d20}    & \textbf{CAT}   & \textbf{Aptitude} & \textbf{Basique/Supérieure} & \textbf{Coût}  \\
1       & A:1   & Réduction                 & BAS & 0  \\
2       & A:2   & Expansion                 & BAS & Spécial    \\
3       & A:3   & Lévitation                & BAS & 1/tour    \\
4       & B:1   & Domination                & BAS & Spécial   \\
5       & A:4   & Contrôle de l'esprit sur le corps & BAS & 0  \\
6       & A:5   & Invisibilité              & BAS & 2/tour  \\
7       & C:1   & Prémonition               & BAS & Spécial  \\
8       & A:6   & Hibernation               & BAS & 0  \\
9       & A:7   & Changer le poids du corps & BAS & 1/tour \\
10      & C:2   & Clairaudience             & BAS & 2/tour \\
11      & C:3   & Clairvoyance              & BAS & 2/tour \\
12      & A:8   & Corps comme arme          & BAS & 0 \\
13      & A:9   & Contrôle de l'énergie     & SUP & Spécial \\
14      & A:10  & Télékinésie               & SUP & 3/tour\\
15      & D:1   & Marche dimensionnelle     & SUP & Spécial\\
16      & D:2   & Projection astrale        & SUP & 0\\
17      & A:11  & Réarrangement moléculaire & SUP & Spécial\\
18      & A:12  & Manipulation moléculaire  & SUP & 50\\
19      & A:13  & Contrôle du corps         & SUP & 5/tour\\
20      & B:2   & Barrière de l'esprit      & SUP & 0\\
\end{tabular}

\bigskip

\begin{tabular}{ccl}
\multicolumn{3}{c}{OPTION 1 : CHOIX D'APTITUDE PAR CATEGORIE} \\
\textbf{Catégorie} &  \textbf{Nb d'aptitudes} & \multicolumn{1}{c}{\textbf{Jet}} \\
\textbf{A} & 13 & 1d20 : rejouer si 14--20 \\
\textbf{B} & 2 & 1d6 : pair = 1, impair = 2 \\
\textbf{C} & 3 & 1d6 : 1--2=1, 3--4=2, 5--6=3 \\
\textbf{D} & 2 & 1d6 : pair = 1, impair = 2 \\
\end{tabular}

%- - - - - - - - - - - - - - - - - - - - - - - - - - - SUB SUB SECTION
\phantomsection\subsubsection*{\uline{\textit{Explication des aptitudes psioniques pour les Guerriers}}}
\addcontentsline{toc}{subsubsection}{Explication des aptitudes psioniques pour les Guerriers}

%++++++++APTITUDE
\pdfbookmark[4]{Réduction}{custom-guerrier-reduction}\phantomsection\label{guerrier-reduction}\textbf{\uline{Réduction (0)}} : la capacité de rendre le corps plus petit en taille (--1/3m par niveau de maîtrise).

\bigskip

\begin{tabular}{cc}
\textbf{Niveau de maîtrise} & \textbf{Réduction} \\
premier     & --1/3m \\
deuxième    & --2/3m \\
troisième   & --1m \\
quatrième   & --1m 1/3 \\
cinquième   & --1m 2/3 \\
sixième*    & --2m \\
...         & ... \\
\end{tabular}

\medskip

* Après six niveaux de possession, l'individu peut devenir aussi petit qu'un minuscule insecte.

\bigskip

%++++++++APTITUDE
\pdfbookmark[4]{Expansion}{custom-guerrier-expansion}\phantomsection\label{guerrier-expansion}\textbf{\uline{Expansion (spécial)}} : la capacité pour le corps de devenir plus grand en taille (+2/3m par niveau de maîtrise). La croissance en masse et en force est proportionnée, de sorte qu'au maximum, la croissance de la force atteint celle d'un géant des tempêtes (+8m est la limite au niveau 12 de maîtrise).

\bigskip

\begin{tabular}{cc}
\textbf{Niveau de maîtrise} & \textbf{Expansion} \\
premier     & +2/3m \\
deuxième    & +1m 1/3 \\
troisième   & +2m \\
quatrième   & +2m 2/3 \\
cinquième   & +3m 1/3 \\
sixième     & +4m \\
...         & ... \\
douzième    & +8m (maximum) \\
\end{tabular}
\bigskip

Durée : il est possible de rester à sa taille maximale pendant deux tours. Mais si le personnage choisit une expansion d'un niveau inférieur à son niveau de maîtrise, il accroît l'endurance pour un tour par niveau. Exemple : si l'expansion potentielle était de 4m (niveau 6), une expansion de seulement 2m (niveau 3) permettrait à l'individu de rester à cette taille pour cinq (2 + 3) tours de jeu.

\bigskip

%++++++++APTITUDE
\pdfbookmark[4]{Lévitation}{custom-guerrier-levitation}\phantomsection\label{guerrier-levitation}\textbf{\uline{Lévitation (1/tour)}} : De manière similaire à la lévitation magique (voir page \pageref{sort-levitation}), cette aptitude permet à l'individu de léviter un tour par nombre de niveaux de possession de l'aptitude.

\bigskip

\begin{tabular}{cc}
\textbf{Niveau de maîtrise} & \textbf{Durée maximale} \\
premier     & 1 tour \\
deuxième    & 3 tours (1+2) \\
troisième   & 6 tours (1+2+3) \\
quatrième   & 10 tours (1+2+3+4) \\
...         & ... \\
\end{tabular}

\bigskip

%++++++++APTITUDE
\pdfbookmark[4]{Domination}{custom-guerrier-domination}\phantomsection\label{guerrier-domination}\textbf{\uline{Domination (spécial)}} : La capacité de forcer quelqu'un à agir selon votre volonté. L'utilisation de cette aptitude requiert une grande concentration, et elle utilise des points de force psionique à hauteur de un point par niveau de créature dominée par minute de domination. Si la domination requiert le dominé de faire des choses qui sont grandement contre sa volonté, la dépense de points de force psionique est doublée.

\bigskip

%++++++++APTITUDE
\pdfbookmark[4]{Contrôle de l'esprit sur le corps }{custom-controle-ESC}\phantomsection\label{guerrier-controle-ESC}\textbf{\uline{Contrôle de l'esprit sur le corps (0)}} : la capacité de supprimer certains besoins corporels (ou de les satisfaire avec des moyens psioniques) ; nourriture, eau, et sommeil peuvent être complètement ignorés pour deux jours par niveau de possession du pouvoir.

\bigskip

\begin{tabular}{cc}
\textbf{Niveau de maîtrise} & \textbf{Durée maximale} \\
premier     & 2 jours \\
deuxième    & 4 jours \\
troisième   & 6 jours \\
...         & ... \\
\end{tabular}

\bigskip

Plus tard, néanmoins, la personne doit passer un nombre de jours équivalent à se reposer pour restaurer son aptitude : si elle ne le fait pas, cela ne mettra pas à mal son corps, mais l'aptitude ne sera plus utilisable tant qu'un tel repos ne sera pas pris.

\bigskip


%++++++++APTITUDE
\pdfbookmark[4]{Invisibilité}{custom-guerrier-invisibilite}\phantomsection\label{guerrier-invisibilite}\textbf{\uline{Invisibilité (2/tour)}} : cette aptitude permet à l'individu de ne pas être détecté, bien que la personne dans cet état ne puisse pas faire d'actions violentes tant qu'elle est invisible. Pour chaque nombre de niveaux de possession de l'aptitude, elle est capable d'échapper à un nombre de niveaux équivalent de créatures, comme montré dans la table ci-dessous.

\bigskip

\begin{tabular}{cc}
\textbf{Niveau de maîtrise} & \textbf{Nb de niveaux de créatures} \\
premier     & 1 \\
deuxième    & 3 (1+2) \\
troisième   & 6 (1+2+3) \\
quatrième   & 10 (1+2+3+4) \\
cinquième   & 15 (1+2+3+4+5) \\
...         & ... \\
\end{tabular}

\bigskip

%++++++++APTITUDE
\pdfbookmark[4]{Prémonition}{custom-guerrier-premonition}\phantomsection\label{guerrier-premonition}\textbf{\uline{Prémonition (spécial)}} : la capacité d'estimer la meilleure probabilité de déroulé des événements, ou d'estimer le résultat le plus probable d'actions entreprises. Ce pouvoir ne s'applique qu'au futur immédiat.

\bigskip

La difficulté de prédiction dépend des facteurs suivants (voir table ci-dessous\footnote{Nous avons traduit la règle originale. La seule innovation que nous avons ajoutée est les seuils sur le nombre de facteurs inconnus. Nous avons choisi 5 et 10 mais un MD pourrait choisir d'autres seuils.}) :

\bigskip

\begin{itemize}
\item Du moment prévu dans le futur,
\item Du nombre de facteurs inconnus attachés à la prédiction.
\end{itemize}

\bigskip

\begin{tabular}{lccc}
DIFFICULTE DE LA PREDICTION & \multicolumn{3}{c}{\textbf{Moment prévu dans le futur}} \\
                    & \textbf{Proche}        & \textbf{Moyennement proche}    & \textbf{Lointain}  \\
\textbf{Nombre de facteurs}  & \textbf{(1--4 tours)}  & \textbf{(5--30 tours) }  & \textbf{(+30 tours)} \\
Moins de 5          & Faible        & Moyenne               & Haute \\
5--10               & Moyenne       & Haute                 & Haute \\
Plus de 10          & Haute         & Haute                 & Haute \\
\end{tabular}

\bigskip

La chance de prédire dépend de deux paramètres :

\bigskip

\begin{itemize}
\item Du total des scores d'Intelligence et de Sagesse du personnage, qui donne la chance de base par difficulté de prédiction,
\item Du niveau de maîtrise de l'aptitude du personnage, qui donne un bonus à cette chance de base.
\end{itemize}

\bigskip

\begin{tabular}{c>{\centering\arraybackslash}p{3.2cm}>{\centering\arraybackslash}p{3.2cm}>{\centering\arraybackslash}p{3.2cm}}
\multicolumn{4}{c}{CHANCE DE BASE DE PREMONITION} \\
\textbf{Total des scores} & \multicolumn{3}{c}{\textbf{Probabilité de prémonition par difficulté}} \\
\textbf{Intelligence et Sagesse} & \textbf{Faible} & \textbf{Moyenne} & \textbf{Haute} \\
Inférieur à 30 & 40\% & 30\% & 20\% \\
30--33         & 50\% & 35\% & 25\% \\
34--35         & 65\% & 45\% & 35\% \\
36 \& plus     & 70\% & 50\% & 40\% \\
\end{tabular}

\bigskip

\begin{tabular}{cc}
\multicolumn{2}{c}{BONUS A LA CHANCE DE BASE DE PREMONITION} \\
\textbf{Niveau de maîtrise} & \textbf{Bonus} \\
premier     & +0\% \\
deuxième    & +2\% \\
troisième   & +5\% (2+3) \\
quatrième   & +9\% (2+3+4) \\
...         & ... \\
\end{tabular}

\bigskip

Coût : la dépense de force psionique est directement reliée au nombre de facteurs inconnus qui doivent être prédits.

\bigskip

Exemples :

\bigskip

\begin{itemize}
\item S'il existe six facteurs inconnus pouvant être basiquement résolus, le coût est de 6 points.
\item Afin de prédire les résultats d'une mêlée, par exemple, chaque attaque doit être faite et comptée comme inconnue, et, dans une mêlée impliquant plusieurs individus et plusieurs monstres, le coût par round de mêlée pourrait facilement atteindre ou dépasser 10 points.
\end{itemize}

\bigskip

Notes :

\bigskip

\begin{itemize}
\item Le coût n'est pas connu de celui qui prédit jusqu'à ce que la prédiction soit réalisée.
\item Si l'individu ayant des pouvoirs psioniques ne dispose pas de suffisamment de points pour prévoir complètement, alors la prémonition cesse au moment où il n'a plus de force pour continuer.
\end{itemize}

\bigskip

N.B. La prémonition dépend entièrement de l'arbitre, et il doit attacher la plus grande attention à l'usage de cette aptitude.

\bigskip

%++++++++APTITUDE
\pdfbookmark[4]{Hibernation}{custom-guerrier-hibernation}\phantomsection\label{guerrier-hibernation}\textbf{\uline{Hibernation (0)}} : cette aptitude permet de suspendre virtuellement toutes les fonctions vitales du corps.

\bigskip

L'individu qui dispose de cette aptitude est capable de se "régler" pour se réveiller à un moment dans le futur et de redémarrer ses fonctions.

\bigskip

\begin{tabular}{cc}
\textbf{Niveau de maîtrise} & \textbf{Durée maximale} \\
premier     & 1 semaine \\
deuxième    & 3 semaines (1+2) \\
troisième   & 6 semaines (1+2+3) \\
...         & ... \\
\end{tabular}

\bigskip

L'individu hibernant ne peut pas être réveillé avant le moment qu'il a lui-même "réglé" pour son réveil. Pour chaque semaine passée en hibernation, l'individu doit passer une journée d'activité normale avant de pouvoir hiberner de nouveau.

\bigskip

%++++++++APTITUDE
\pdfbookmark[4]{Changer le poids du corps}{custom-guerrier-changer-poids}\phantomsection\label{guerrier-changer-poids}\textbf{\uline{Changer le poids du corps (1/tour)}} : cette aptitude permet à l'individu d'ajuster le poids du corps à la surface sur laquelle il marche, de sorte qu'il ne s'enfonce pas dans elle, par exemple dans l'eau, les sables mouvants, la boue, etc.

\bigskip

\begin{tabular}{cc}
\textbf{Niveau de maîtrise} & \textbf{Durée maximale} \\
premier     & 1h/jour \\
deuxième    & 2h/jour \\
troisième   & 3h/jour \\
...         & ... \\
\end{tabular}

\bigskip

%++++++++APTITUDE
\pdfbookmark[4]{Clairaudience}{custom-guerrier-clairaudienc}\phantomsection\label{guerrier-clairaudience}\textbf{\uline{Clairaudience (2/tour)}} : l'aptitude d'entendre à distance. L'individu possédant ce pouvoir est capable d'entendre ce qui se passe jusqu'à 9 mètres de distance. Le pouvoir est directionnel.

\bigskip

1/3m de pierre équivaut à 3m d'espace vide. Après chaque niveau auquel l'individu a acquis cette aptitude, ce dernier gagne une distance additionnelle de 3m par nombre de niveaux.

\bigskip

\begin{tabular}{cc}
\textbf{Niveau de maîtrise} & \textbf{Portée maximale} \\
premier     & 9m \\
deuxième    & 15m (9+2x3) \\
troisième   & 24m (9+2x3+3x3) \\
...         & ... \\
\end{tabular}

\bigskip

Ce pouvoir peut être utilisé en conjonction avec une boule de cristal (voir page \pageref{objet-boule-cristal}). Il est sujet à des dispositifs d'entraves magiques et non magiques, comme mentionné dans les explications du sort du même nom (voir page \pageref{sort-clairaudience}).

\bigskip

%++++++++APTITUDE
\pdfbookmark[4]{Clairvoyance}{custom-guerrier-clairvoyance}\phantomsection\label{guerrier-clairvoyance}\textbf{\uline{Clairvoyance(2/tour)}} : comme l'aptitude de clairaudience ci-dessus, excepté que la portée est dix fois supérieure. Au septième niveau de possession, la portée devient illimitée en distance.

\bigskip
\begin{tabular}{cc}
\textbf{Niveau de maîtrise} & \textbf{Portée maximale} \\
premier     & 90m \\
deuxième    & 150m \\
troisième   & 240m \\
quatrième   & 360m \\
cinquième   & 510m \\
sixième     & 690m \\
septième    & illimitée \\
\end{tabular}

\bigskip

%++++++++APTITUDE
\pdfbookmark[4]{Corps comme arme}{custom-guerrier-corps-comme-arme}\phantomsection\label{guerrier-corps-comme-arme}\textbf{\uline{Corps comme arme (0)}} : cette aptitude requiert de la personne qui l'a obtenue de renoncer à l'utilisation de toute arme et armure pour que son corps assume leurs fonctions. L'individu altère psioniquement son corps pour l'endurcir dans le but de frapper ou de se défendre.

\bigskip

Selon le niveau de maîtrise du pouvoir, la classe d'armure, les modificateurs à l'attaque (voir page \pageref{custom-combat-alternatif}) et les bonus dommages sont donnés par la table suivante (les dommages sont ceux de l'arme équivalente).

\bigskip

\begin{tabular}{cccc}
\textbf{Niveau de maîtrise} & \textbf{Classe d'armure} & \textbf{Attaque équivalente à} & \textbf{Bonus aux dommages}\\
premier     & 8  & Dague                & 0 \\
deuxième    & 7  & Hache à main         & 0 \\
troisième   & 6  & Masse                & 0 \\
quatrième   & 5  & Hache de bataille    & 0 \\
cinquième   & 4  & Epée                 & 0 \\
sixième     & 3  & Epée                 & +1 \\
septième    & 2  & Epée                 & +2 \\
huitième    & 1  & Epée                 & +3 \\
neuvième    & 0  & Epée                 & +4 \\
dixième     & -1 & Epée                 & +5 \\
\end{tabular}

\bigskip

Note : dans les trois premiers livrets de D\&D, si l'on n'utilise pas les règles de CHAINMAIL, la probabilité de toucher lors d'une attaque ne dépend que du niveau du personnage et de la classe d'armure de son opposant, et pas de son arme. Les dommages sont aussi constants selon les armes (1d6 si pas de modificateurs).

\bigskip

\textit{Corps comme arme} utilise le système de combat alternatif proposé dans le supplément I, \texttt{Greyhawk} (voir pages \pageref{ew-corps-comme-arme} et \pageref{custom-combat-alternatif}). La ligne \textit{Attaque équivalente à} fait référence aux armes listées dans le système de combat alternatif, ce dernier faisant intervenir l'arme du porteur à deux niveaux :

\bigskip

\begin{itemize}
\item Comme modificateur à l'attaque, selon la classe d'armure du défenseur ;
\item Comme arme ayant des dommages particuliers (au lieu du 1d6 des livrets initiaux).
\end{itemize}

\bigskip

Précision : celui qui possède l'aptitude depuis trois niveaux (niveau de maîtrise 3) peut frapper comme une dague, une hache à main ou une masse (prendre le bonus le plus favorable) mais, dans tous les cas, il inflige les dommages qui sont ceux d'une masse (dommages selon son niveau de maîtrise).

\bigskip

Noter que, en ce qui concerne le facteur arme, le \textit{Corps comme arme} est considéré comme ayant une classe de moins que la dague en ce qui concerne le facteur vitesse, mais la même classe en ce qui concerne la longueur. Cela signifie qu'il faut décaler d'une colonne vers la droite dans la table des modificateurs du système de combat alternatif de \texttt{Greyhawk} si la vitesse entre en jeu (voir page \pageref{custom-combat-alternatif}).

\bigskip

%++++++++APTITUDE
\pdfbookmark[4]{Contrôle de l'énergie}{custom-guerrier-controle-energie}\phantomsection\label{guerrier-controle-energie}\textbf{\uline{Contrôle de l'énergie (spécial)}} : cette aptitude permet à l'utilisateur de canaliser l'énergie dirigée vers lui autour de son corps et de la dissiper. Ainsi, si un sort est dirigé sur lui ou sur l'endroit où il se trouve, il peut utiliser son aptitude pour rendre l'énergie du sort inoffensive.

\bigskip

Le coût d'utilisation de cette aptitude de 5 points de force psionique par niveau d'énergie dissipée. Comme règle générale, considérez chaque dé de dommage qui peut être fait par l'énergie comme un niveau, et si aucun dé de dommage n'est applicable, le niveau du sort peut être utilisé comme mesure de niveau.

\bigskip

%++++++++APTITUDE
\pdfbookmark[4]{Télékinésie}{custom-guerrier-telekinesie}\phantomsection\label{guerrier-telekinesie}\textbf{\uline{Télékinésie (3/tour)}} : la capacité de bouger les objets par le pouvoir de l'esprit. Le possesseur est capable de bouger un poids de 50 pièces d'or par nombre de niveaux de maîtrise.

\bigskip

\begin{tabular}{cc}
\textbf{Niveau de maîtrise} & \textbf{Poids maximum (en PO)} \\
premier     & 50 \\
deuxième    & 150 (50+2x50) \\
troisième   & 300 (50+2x50+3x50) \\
...         & ... \\
\end{tabular}

\bigskip

%++++++++APTITUDE
\pdfbookmark[4]{Marche dimensionnelle}{custom-guerrier-marche-dimensionnelle}\phantomsection\label{guerrier-marche-dimensionnelle}\textbf{\uline{Marche dimensionnelle (spécial)}} : la maîtrise de cette aptitude permet à l'individu de se déplacer entre les dimensions pour arriver à un endroit distant en un temps relativement court. Le problème de se perdre en route demeure néanmoins, ce qui implique qu'il faille utiliser la table suivante pour déterminer la durée réelle du déplacement. La durée de base est d'une heure pour 160km de distance :

\bigskip

\begin{tabular}{c>{\centering\arraybackslash}p{2.1cm}>{\centering\arraybackslash}p{2.1cm}>{\centering\arraybackslash}p{2.1cm}>{\centering\arraybackslash}p{2.1cm}>{\centering\arraybackslash}p{2.1cm}}
& \multicolumn{5}{c}{\textbf{Altération du temps par jet de dé (1d12)}} \\
\textbf{Niveau de maîtrise} & \textbf{1--2} & \textbf{3--5} & \textbf{6--8} & \textbf{9-11} & \textbf{12} \\
premier             & +100\% & +50\% & +25\% & +10\% & 0 \\
deuxième--quatrième & +100\% & +25\% & +10\% & 0     & 0 \\
cinquième--septième &  +50\% & +10\% & 0     & 0     & --10\% \\
huitième et au delà &  +25\% &     0 & 0     & --10\% & --50\% \\
\end{tabular}

\bigskip

Les règles ne stipulent pas de coût en points de force psionique. Nous proposons 10 points par heure de voyage. Ainsi, si le voyage s'allonge, le nombre de points consommé augmentera.

\bigskip

%++++++++APTITUDE
\pdfbookmark[4]{Projection astrale}{custom-guerrier-projection-astrale}\phantomsection\label{guerrier-projection-astrale}\textbf{\uline{Projection astrale (0)}} : cette aptitude est similaire à celle du sort du même nom (voir page \pageref{sort-astral}).

\bigskip

Quand elle est projetée astralement, la personne ne peut pas être détectée excepté par quelques rares créatures, et son corps astral n'est pas sujet aux dangers habituels.

\bigskip

Au premier niveau de maîtrise, le possesseur ne peut avancer qu'au rythme de la marche ; au second, il peut courir aussi vite qu'un petit cheval ; au troisième, il est capable de voler aussi vite qu'un rokh (voir page \pageref{monstre-rokh}), et la vitesse, après, double avec chaque niveau de maîtrise ; en plus, au dixième niveau de maîtrise, le possesseur de l'aptitude est capable de se projeter dans l'espace à la vitesse de la lumière.

\bigskip
\begin{tabular}{cc}
\textbf{Niveau de maîtrise} & \textbf{Vitesse de déplacement} \\
premier     & marche \\
deuxième    & petit cheval \\
troisième   & rokh \\
quatrième   & rokh x 2 \\
cinquième   & rokh x 4 \\
...         & ... \\
dixième     & lumière \\
\end{tabular}

\bigskip

Les dangers sont basiquement de deux types :

\bigskip

\begin{itemize}
\item Premièrement, il est possible de rencontrer une créature qui peut opérer dans le plan astral (les démons le font, les méduses et les basilics regardent dedans, etc.).
\item Secondement, le corps astral est attaché au corps physique par un cordon d'argent. Si ce cordon est cassé, alors le corps physique et le corps astral meurent.
\end{itemize}

\bigskip

Lors du déplacement astral, il est nécessaire de tester la présence d'un vent psychique. Lancer 1d100 et consulter la table ci-dessous.

\bigskip

\begin{tabular}{cl}
\textbf{1d100} & \textbf{Existence du vent psychique} \\
01--10 & Un vent psychique souffle dans un rayon de 160km autour du corps physique \\
11--60 & Un vent psychique souffle à plus de 160km autour du corps physique \\
61--90 & Un vent psychique souffle dans l'espace lointain \\
91--100 & Aucun vent psychique ne souffle \\
\end{tabular}

\bigskip

Si un vent psychique souffle à moins de 160km du corps physique, il affecte les personnes projetées astralement comme suit :

\bigskip

\begin{tabular}{ccc}
\textbf{Niveau de maîtrise} & \textbf{Etre emporté} & \textbf{Perte de 1--100 jours} \\
premier            & 08\% & 20\% \\
deuxième           & 07\% & 18\% \\
troisième          & 05\% & 15\% \\
quatrième          & 04\% & 12\% \\
cinquième          & 04\% & 10\% \\
sixième            & 02\% & 07\% \\
septième--neuvième & 01\% & 05\% \\
dixième            & --   & 02\% \\
\end{tabular}

\medskip

Etre emporté casse le lien d'argent, ce qui implique la mort des corps physique et astral.

\bigskip

Perdre entre 1--10 jours se produit lorsque la tentative échoue et que le corps astral est projeté à l'intérieur plutôt qu'à l'extérieur. De 1--100 jours seront perdus en raison de la désorientation due au déchirement de l'esprit.

\bigskip

Il n'y a pas de coût psionique pour cette aptitude.

\bigskip

%++++++++APTITUDE
\pdfbookmark[4]{Réarrangement moléculaire}{custom-guerrier-rearrange-mol}\phantomsection\label{guerrier-rearrange-mol}\textbf{\uline{Réarrangement moléculaire (spécial)}} : avec cette aptitude, le possesseur est capable d'altérer les molécules des substances métalliques en une autre structure, ainsi les transformant en des métaux différents. Cela, en effet, transmute les métaux, mais ne peut être exécuté qu'une fois par mois au coût de 2 points psioniques par poids de pièce d'or changé. Le poids maximal par niveau de maîtrise est de 10 pièces d'or.

\bigskip

%++++++++APTITUDE
\pdfbookmark[4]{Manipulation moléculaire}{custom-guerrier-manip-mol}\phantomsection\label{guerrier-manip-mol}\textbf{\uline{Manipulation moléculaire (50)}} : la capacité de décaler les arrangement moléculaires de façon à créer une substance de faible résistance. Avec chaque niveau de maîtrise, le possesseur devient plus adepte de la manipulation :

\bigskip

\begin{tabular}{cc}
\textbf{Niveau de maîtrise} & \textbf{Capable de manipuler l'équivalent de} \\
premier    & cordelettes fines \\
deuxième   & cordes fines \\
troisième  & cordes épaisses ou lanières de cuir\\
quatrième  & câbles \\
cinquième  & chaînes légères \\
sixième    & chaînes lourdes \\
septième   & fers et menottes\\
huitième   & barres de fer, 2.5cm de diamètre\\
neuvième   & barres d'acier, 2.5cm de diamètre \\
dixième    & murs épais de pierre, 2/3m d'épaisseur, trou de la taille d'un homme \\
\end{tabular}

\bigskip

%++++++++APTITUDE
\pdfbookmark[4]{Contrôle du corps}{custom-guerrier-controle-corp}\phantomsection\label{guerrier-controle-corps}\textbf{\uline{Contrôle du corps (5/tour)}} : la capacité d'adapter le corps à des températures extrêmes ou des éléments destructifs hostiles (fumées empoisonnées, eau, acide).

\bigskip

Cela permet au possesseur de traverser le feu, de respirer sous l'eau, etc., cela pour une durée limitée dépendant du niveau de maîtrise qu'il possède. Comme règle générale, assumer que l'individu est capable de résister à l'équivalent d'un dé de dommages causé par la substance ou l'environnement pendant un tour (dix minutes). Cela implique qu'il pourrait traverser un feu normal ou rester sous l'eau pendant un tour, mais dans un environnement plus hostile, la limite du temps d'exposition serait réduite en conséquence.

\bigskip

\begin{tabular}{cc}
\textbf{Niveau de maîtrise} & \textbf{Résistance et durée} \\
premier     & 1 dé de dommages pendant 1 période \\
deuxième    & 1 dé de dommages pendant 3 périodes (1+2) \\
troisième   & 1 dé de dommages pendant 6 périodes (1+2+3) \\
...         & ... \\
dixième     & 1 dé de dommages pendant 55 périodes (1+2+3+...+10) \\
\end{tabular}

\bigskip

%++++++++APTITUDE
\pdfbookmark[4]{Barrière de l'esprit}{custom-guerrier-barriere-esprit}\phantomsection\label{guerrier-barriere-esprit}\textbf{\uline{Barrière de l'esprit (0)}} : cette aptitude protège le corps physique et l'esprit d'une possession.

\bigskip

Elle peut être utilisée quand le corps est abandonné (comme en cas de projection astrale) ou à d'autres moments pour le protéger de possessions par urnes magiques (voir page \pageref{sort-urne-magique}), démons ou diables.

\bigskip

La chance pour que le possesseur puisse établir la barrière de son esprit avec succès est de 10\% par niveau de maîtrise. Après le dixième niveau de maîtrise, le pourcentage de chances qu'il soit capable de localiser l'urne ou amulette de l'être qui tente de le posséder croît de la même façon.

\bigskip

\begin{tabular}{ccc}
&\textbf{Chance de réussite} & \textbf{Localisation de} \\
\textbf{Niveau de maîtrise} & \textbf{Barrière de l'esprit} & \textbf{l'origine de l'attaque}\\
premier     & 10\%  & 0\% \\
deuxième    & 20\%  & 0\% \\
troisième   & 30\%  & 0\% \\
...         & ...   & ... \\
dixième     & 100\% & 0\% \\
onzième     & 100\% & 10\% \\
douzième    & 100\% & 20\% \\
...         & ...   & ... \\
\end{tabular}


%- - - - - - - - - - - - - - - - - - - - - - - - - - - SUB SUB SECTION








\newpage
%- - - - - - - - - - - - - - - - - - - - - - - - - - - SUB SUB SECTION
\phantomsection\subsubsection*{\uline{\textit{Aptitudes psioniques pour les Magiciens et les Illusionnistes}}}
\addcontentsline{toc}{subsubsection}{Aptitudes psioniques pour les Magiciens et les Illusionnistes}

\begin{tabular}{cclcc}
\textbf{1d20}& \textbf{CAT} & \textbf{Aptitude} &  \textbf{Basique/Supérieure} & \textbf{Coût} \\
1   & C:1 & Détection du Mal/Bien       & BAS & 0 \\
2   & C:2 & Détection de la magie       & BAS & 1/tour \\
3   & B:1 & Perception extrasensorielle & BAS & 1/tour \\
4   & B:2 & Hypnose                     & BAS & spécial \\
5   & A:1 & Lévitation                  & BAS & 1/tour \\
6   & C:3 & Clairaudience               & BAS & 1/tour \\
7   & C:4 & Clairvoyance                & BAS & 1/tour \\
8   & A:2 & Réduction                   & BAS & 0 \\
9   & A:3 & Expansion                   & BAS & spécial \\
10  & A:4 & Agitation moléculaire       & BAS & 2/tour \\
11  & B:3 & Projection télépathique     & SUP & 3/tour \\
12  & C:5 & Prémonition                 & SUP & spécial \\
13  & D:1 & Porte dimensionnelle        & SUP & 10 \\
14  & A:5 & Télékinésie                 & SUP & 3/tour \\
15  & D:2 & Téléportation               & SUP & 20 \\
16  & D:3 & Projection astrale          & SUP & spécial \\
17  & D:4 & Forme éthérée               & SUP & 5/tour \\
18  & A:6 & Altération de la forme      & SUP & spécial \\
19  &     & Relancer 1d20               &  & \\
20  &     & Relancer 1d20               &  & \\
\end{tabular}

\bigskip

\begin{tabular}{ccl}
\multicolumn{3}{c}{OPTION 1 : CHOIX D'APTITUDE PAR CATEGORIE} \\
\textbf{Catégorie} &  \textbf{Nb d'aptitudes} & \multicolumn{1}{c}{\textbf{Jet}} \\
\textbf{A} & 6 & 1d6 \\
\textbf{B} & 3 & 1d6 : 1--2=1, 3--4=2, 5--6=3 \\
\textbf{C} & 5 & 1d6 : rejouer si 6 \\
\textbf{D} & 4 & 1d4 \\
\end{tabular}


%- - - - - - - - - - - - - - - - - - - - - - - - - - - SUB SUB SECTION
\phantomsection\subsubsection*{\uline{\textit{Explication des aptitudes psioniques pour les Magiciens}}}
\addcontentsline{toc}{subsubsection}{Explication des aptitudes psioniques pour les Magiciens}

%++++++++APTITUDE
\pdfbookmark[4]{Détection du Mal/du Bien}{custom-magicien-detection-mal}\phantomsection\label{magicien-detection-mal}\textbf{\uline{Détection du Mal/du Bien (0)}} : cette aptitude est simplement le pouvoir de détecter l'aura qui émane de l'esprit des créatures -- ou qui reste sur les objets ou dans les lieux si l'aura est exceptionnellement forte. Cette aptitude ne fonctionne pas pour les non-psioniques.

\bigskip

Il n'y a pas d'utilisation de points de force psionique pour détecter le mal ou le bien.

\bigskip

%++++++++APTITUDE
\pdfbookmark[4]{Détection de la magie}{custom-magicien-magie}\phantomsection\label{magicien-magie}\textbf{\uline{Détection de la magie (1/tour)}} : bien que cette aptitude soit similaire en nature aux autres types de détection, la magie opère sur un plan différent, ce qui implique que le possesseur de ce pouvoir soit obligé de dépenser 1 point de force psionique pour chaque tour dans lequel il tente de détecter la magie.

\bigskip

Après trois niveaux de maîtrise de l'aptitude, le possesseur possède une chance cumulative de 10\% de déterminer quelle sorte de magie est impliquée (et non seulement que certaines forces magiques sont là), soit au quatrième niveau de progression, il possède une chance de 20\% de déterminer la nature basique du sort qui est en cours ou a été lancé.

\bigskip

\begin{tabular}{cc}
                            &\textbf{Détermination de la} \\
\textbf{Niveau de maîtrise} & \textbf{nature du sort} \\
premier     & 0\%   \\
deuxième    & 0\%   \\
troisième   & 10\%  \\
quatrième   & 20\%  \\
cinquième   & 30\%  \\
...         & ...    \\
\end{tabular}

\bigskip

%++++++++APTITUDE
\pdfbookmark[4]{Perception extrasensorielle}{custom-magicien-ESP}\phantomsection\label{magicien-ESP}\textbf{\uline{Perception extrasensorielle (1/tour)}} : un pouvoir similaire au sort du même nom (voir page \pageref{sort-esp}), excepté que la portée est le double de celle du sort, soit 4m. Noter que cette aptitude permet au possesseur d'être \og à l'écoute \fg{} des pensées, et qu'il y a une différence avec le fait de recevoir et de transmettre des pensées de manière télépathique.

\bigskip

%++++++++APTITUDE
\pdfbookmark[4]{Hypnose}{custom-magicien-hypnose}\phantomsection\label{magicien-hypnose}\textbf{\uline{Hypnose (spécial)}} : l'aptitude ressemble au sort de suggestion des magiciens (voir page \pageref{sort-suggestion}), mais il n'affectera pas les personnes très stupides ou hautement intelligentes.

\bigskip

\begin{tabular}{cc}
                            &\textbf{Nombre de niveaux} \\
\textbf{Niveau de maîtrise} & \textbf{de créatures hypnotisées}\\
premier     & 1   \\
deuxième    & 3 (1+2)   \\
troisième   & 6 (1+2+3)  \\
quatrième   & 10 (1+2+3+4) \\
...         & ...    \\
\end{tabular}

\bigskip

Le coût d'utilisation de cette aptitude est de 1 point de force psionique pour chaque niveau de créature affecté. Si l'intelligence de la créature sur laquelle l'aptitude est utilisée est entre 13 et 16, un jet de sauvegarde contre la magie est autorisé, et s'il est réussi, le pouvoir ne le touche pas.

\bigskip

La suggestion post-hypnose aura une chance cumulative de 5\% par jour de disparaître.

\bigskip

%++++++++APTITUDE
\pdfbookmark[4]{Lévitation}{custom-magicien-levitation}\textbf{\uline{Lévitation (1/tour)}} : identique à l'aptitude des guerriers (voir page \pageref{guerrier-levitation}).

\bigskip

%++++++++APTITUDE
\pdfbookmark[4]{Clairaudience}{custom-magicien-clairaudience}\textbf{\uline{Clairaudience (1/tour)}} : identique à l'aptitude des guerriers (voir page \pageref{guerrier-clairaudience}).

\bigskip

%++++++++APTITUDE
\pdfbookmark[4]{Clairvoyance}{custom-magicien-clairvoyance}\textbf{\uline{Clairvoyance (1/tour)}} : identique à l'aptitude des guerriers (voir page \pageref{guerrier-clairvoyance}).

\bigskip

%++++++++APTITUDE
\pdfbookmark[4]{Réduction}{custom-magicien-reduction}\textbf{\uline{Réduction (0)}} : identique à l'aptitude des guerriers (voir page \pageref{guerrier-reduction}).

\bigskip

%++++++++APTITUDE
\pdfbookmark[4]{Expansion}{custom-magicien-expansion}\textbf{\uline{Expansion (spécial)}} : identique à l'aptitude des guerriers (voir page \pageref{guerrier-expansion}).

\bigskip

%++++++++APTITUDE
\pdfbookmark[4]{Agitation moléculaire}{custom-magicien-agitation-mol}\phantomsection\label{magicien-agitation-mol}\textbf{\uline{Agitation moléculaire (2/tour)}} : cette aptitude permet au possesseur de faire bouger les molécules d'une chose plus rapidement que la normale. Bien que seulement un petit nombre d'entre elles puisse être affecté, si l'agitation dure pendant dix tours, les effets suivants seront constatés :

\medskip

\begin{tabular}{cc}
\textbf{Type de matériau}   & \textbf{Effet} \\
Papier, paille              & feu avec flammes vives \\
bois sec                    & le bois se consume \\
chair                       & boursouflures* \\
métal                       & chaud au toucher** \\
\end{tabular}

\bigskip

* A chaque tour, la créature prendra 1 point de dommages, cumulatif (1 pour le premier tour, 2 pour le deuxième, 3 pour le troisième, etc.) si l'aptitude continue à être utilisée contre lui.

\bigskip

** Devient brûlant comme via le sort "Chauffer le métal" des druides (voir page \pageref{sort-chauffe-metal}), et refroidira à la même vitesse si l'attention de la personne munie de pouvoirs psioniques quitte l'objet.

\bigskip

Même si la quantité de matériau sur laquelle le possesseur de l'aptitude exerce son influence ne change pas avec des niveaux additionnels de maîtrise, le temps requis pour atteindre les effets décrits ci-dessus diminue de un tour tous les niveaux de maîtrise au dessus du premier. Noter que l'objet affecté doit être visible (cela inclut la clairvoyance) de l'individu psionique.

\bigskip

%++++++++APTITUDE
\pdfbookmark[4]{Projection télépathique}{custom-magicien-projection-telepathique}\phantomsection\label{magicien-projection-telepathique}\textbf{\uline{Projection télépathique (3/tour)}} : cette aptitude est assez similaire au pouvoir conféré par le heaume de télépathie (voir page \pageref{objet-heaume-telepathie}).

\bigskip

L'individu avec cette aptitude est capable d'envoyer des messages télépathiques à n'importe quelle personne  disposant de l'aptitude "Perception extrasensorielle" (qu'elle soit psionique ou magique). De plus, le possesseur de l'aptitude est capable d'influencer un niveau de créature par niveau de maîtrise de l'aptitude.

\bigskip

Même les créatures basiquement stupides ou hautement intelligentes peuvent être influencées de manière télépathique. La portée du pouvoir est de 2m plus le niveau de maîtrise de l'individu fois 1/3, de manière cumulative, comme indiqué dans la table ci-dessous. Au dixième niveau, la portée double.

\bigskip

\begin{tabular}{ccc}
                            &\textbf{Nombre de niveaux} &\\
\textbf{Niveau de maîtrise} & \textbf{de créatures influencées} & \textbf{Portée}\\
premier     & 1             & 2m 1/3 (2+1/3)\\
deuxième    & 3 (1+2)       & 3m (2+1/3+2x1/3)\\
troisième   & 6 (1+2+3)     & 4m (2+1/3+2x1/3+3x1/3)\\
quatrième   & 10 (1+2+3+4)  & 5m 1/3 (2+1/3+2x1/3+3x1/3+4x1/3)\\
...         & ...           & ...\\
dixième     & 55            & Environ 20m \\
onzième     & 66            & Environ 40m (double du niveau 10) \\
douzième    & 78            & Environ 80m (double du niveau 11) \\
...         & ...           & ... \\
\end{tabular}

\bigskip

Même les créatures basiquement stupides ou hautement intelligentes peuvent être influencées de manière télépathique. La portée du pouvoir est de 2m plus le niveau de maîtrise de l'individu, de manière cumulative.

\bigskip

Note : un heaume de télépathie (voir page \pageref{objet-heaume-telepathie}) double le pouvoir et la portée de l'aptitude et octroie, en plus, au possesseur les effets d'un bonus de +4 sur son intelligence .

\bigskip

%++++++++APTITUDE
\pdfbookmark[4]{Prémonition}{custom-magicien-premonition}\phantomsection\label{magicien-premonition}\textbf{\uline{Prémonition (spécial)}} : identique à l'aptitude des guerriers (voir page \pageref{guerrier-premonition}).

\bigskip

%++++++++APTITUDE
\pdfbookmark[4]{Porte dimensionnelle}{custom-magicien-porte-dimensionnelle}\phantomsection\label{magicien-porte-dimensionnelle}\textbf{\uline{Porte dimensionnelle (10)}} : cette aptitude est exactement la même que le sort du même nom (voir page \pageref{sort-porte-dimensionnelle}), excepté que l'individu ayant des pouvoirs psioniques dépense des points de force psionique pour accomplir la téléportation limitée.

\bigskip

%++++++++APTITUDE
\pdfbookmark[4]{Télékinésie}{custom-magicien-telekinesie}\textbf{\uline{Télékinésie (3/tour)}} : identique à l'aptitude des guerriers (voir page \pageref{guerrier-telekinesie}).

\bigskip

%++++++++APTITUDE
\pdfbookmark[4]{Téléportation}{custom-magicien-teleportation}\phantomsection\label{magicien-teleportation}\textbf{\uline{Téléportation (20)}} : l'aptitude est exactement la même que le sort du même nom (voir page \pageref{sort-teleporter}), excepté qu'elle coûte de l'énergie psionique à exécuter.

\bigskip

Si de l'énergie psionique additionnelle est dépensée, la chance d'arriver trop bas ou trop haut est altérée proportionnellement ; ainsi, si 10 points additionnels sont dépensés dans la téléportation, les risques d'arriver top bas ou trop haut sont réduits de 5\% chacun.

\bigskip

%++++++++APTITUDE
\pdfbookmark[4]{Projection astrale}{custom-magicien-projection-astrale}\phantomsection\label{magicien-projection-astrale}\textbf{\uline{Projection astrale (spécial)}} : identique au pouvoir disponible pour les guerriers (voir page \pageref{guerrier-projection-astrale}), excepté que des sorts peuvent être utilisés comme détaillé dans la description du sort astral (voir page \pageref{sort-astral}).

\bigskip

%++++++++APTITUDE
\pdfbookmark[4]{Forme éthérée}{custom-magicien-forme-etheree}\phantomsection\label{magicien-forme-etheree}\textbf{\uline{Forme éthérée (5/tour)}} : ce pouvoir confère la même aptitude que la potion magique de forme éthérée (voir page \pageref{objet-huile-etheree}).

\bigskip

En fait, la personne psionique altère les vibrations de son corps pour les aligner avec celles d'un autre plan. Noter que tant que cette aptitude n'est pas maîtrisée depuis plusieurs niveaux, il n'est pas possible de porter beaucoup de choses, car l'état éthéré s'étend uniquement à un poids d'équipement/encombrement de 50 pièces d'or par niveau de maîtrise.

\bigskip

Les individus éthérés sont affectés par le vent psychique (détaillé dans le paragraphe Projection astrale page \pageref{guerrier-projection-astrale}) comme suit : la chance que le vent souffle est de 1\%, et cela doit être testé à tous tours durant lesquels l'individu est dans sa forme éthérée. Si le vent souffle, l'individu éthéré ne sera pas tué, mais la chance qu'il a d'être perdu est \textbf{doublée}. A ce moment, il n'y a plus de dépense de points de force psionique pour rester éthéré, car l'individu est perdu dans le plan et le restera pour un temps décidé par le jet de dés.

\bigskip

%++++++++APTITUDE
\pdfbookmark[4]{Altération de la forme}{custom-magicien-alteration-forme}\phantomsection\label{magicien-alteration-forme}\textbf{\uline{Altération de la forme (spécial)}} : ce pouvoir est assez similaire au sort de métamorphose (voir pa- ge~\pageref{sort-metamorphose}).

\bigskip

Le possesseur est capable d'altérer sa forme en presque n'importe quoi, mais il n'y a pas de gain correspondant aux caractéristiques de la forme assumée -- ni de perte de compétences de la personne qui a altéré sa forme. Le coût basique est de 5 points d'énergie psionique pour changer sa forme, avec des changements extrêmes en taille, masse ou composition moléculaire, coûtant de manière additionnelle :

\bigskip

\begin{tabular}{ll}
\textbf{Exemple d'altération extrême} & \textbf{Coût psionique} \\
Changement de poids de +/- poids de 1000 pièces d'or & 2 points/1000 PO \\
Changement en végétal* & 10 points \\
Changement en minéral & 50 points \\
\end{tabular}

\bigskip

* Le changement dans l'autre sens vers le monde animal est chargé de la même façon en points de force psionique.



\newpage
%- - - - - - - - - - - - - - - - - - - - - - - - - - - SUB SUB SECTION
\phantomsection\subsubsection*{\uline{\textit{Aptitudes psioniques pour les Clercs, les Moines et les Druides}}}
\addcontentsline{toc}{subsubsection}{Aptitudes psioniques pour les Clercs, les Moines et les Druides}

\bigskip

\begin{tabular}{cclcc}
\textbf{1d20} & \textbf{CAT} & \textbf{Aptitude}& Basique/Supérieure & Coût \\
1   & C:1 & Détection du Mal/Bien       & BAS & 0 \\
2   & B:1 & Empathie                    & BAS & 0 \\
3   & A:1 & Lévitation                  & BAS & 1/tour \\
4   & B:2 & Hypnose                     & BAS & 1/tour \\
5   & B:3 & Domination                  & BAS & spécial \\
6   & B:4 & Perception extrasensorielle & BAS & 1/tour \\
7   & A:2 & Ajustement cellulaire       & BAS & spécial \\
8   & A:3 & Contrôle de l'esprit sur le corps & BAS & 0 \\
9   & A:4 & Changer le poids du corps   & BAS & 1/tour \\
10  & B:5 & Télépathie avec les animaux & BAS & 2/tour \\
11  & A:5 & Réarrangement moléculaire   & SUP & 5/tour  \\
12  & B:6 & Altération de l'aura        & SUP & spécial  \\
13  & C:2 & Prémonition                 & SUP & spécial  \\
14  & B:7 & Projection télépathique     & SUP & 3/tour  \\
15  & D:1 & Marche dimensionnelle       & SUP & spécial  \\
16  & D:2 & Projection astrale          & SUP & spécial  \\
17  & B:8 & Domination des masses       & SUP & spécial  \\
18  & D:3 & Voyage probabiliste         & SUP & spécial  \\
19  &     & Relancer 1d20               & & \\
20  &     & Relancer 1d20               & & \\
\end{tabular}

\bigskip

\begin{tabular}{ccl}
\multicolumn{3}{c}{OPTION 1 : CHOIX D'APTITUDE PAR CATEGORIE} \\
\textbf{Catégorie} &  \textbf{Nb d'aptitudes} & \multicolumn{1}{c}{\textbf{Jet}} \\
\textbf{A} & 5 & 1d6 : rejouer si 6 \\
\textbf{B} & 8 & 1d8 \\
\textbf{C} & 2 & 1d6 : pair = 1, impair = 2 \\
\textbf{D} & 3 & 1d4 : rejouer si 4 \\
\end{tabular}

\bigskip

\textbf{Liste des aptitudes}

\bigskip


%- - - - - - - - - - - - - - - - - - - - - - - - - - - SUB SUB SECTION
\phantomsection\subsubsection*{\uline{\textit{Explication des aptitudes psioniques pour les Clercs}}}
\addcontentsline{toc}{subsubsection}{Explication des aptitudes psioniques pour les Clercs}

%++++++++APTITUDE
\pdfbookmark[4]{Détection du Mal/du Bien}{custom-clerc-detection-mal}\textbf{\uline{Détection du Mal/du Bien (0)}} : identique à l'aptitude des magiciens (voir page \pageref{magicien-detection-mal}).

\bigskip

%++++++++APTITUDE
\pdfbookmark[4]{Empathie}{custom-clerc-empathie}\phantomsection\label{clerc-empathie}\textbf{\uline{Empathie (0)}} : cette aptitude permet au possesseur de ressentir les émotions basiques ou les besoins de n'importe quelle créature consciente. C'est-à-dire qu'il peut sentir l'amour, la haine, l'hostilité, la bienveillance, la rage, la peur, la curiosité, le doute, la faim, la soif, et ainsi de suite.

\bigskip

La portée de cette aptitude est seulement de 2/3m au premier niveau de maîtrise, mais avec chaque niveau de progression, le possesseur est capable d'étendre son aptitude de 2/3m, si bien qu'au troisième niveau de maîtrise, il peut être empathique dans un rayon d'environ 2m.

\bigskip

\begin{tabular}{cc}
\textbf{Niveau de maîtrise} & \textbf{Portée}\\
premier     & 2/3m   \\
deuxième    & 1m 1/3   \\
troisième   & 2m  \\
quatrième   & 2m 2/3 \\
...         & ...    \\
\end{tabular}

\bigskip

%++++++++APTITUDE
\pdfbookmark[4]{Lévitation}{custom-clerc-levitation}\textbf{\uline{Lévitation (1/tour)}} : identique à l'aptitude des guerriers (voir page \pageref{guerrier-levitation}).

\bigskip

%++++++++APTITUDE
\pdfbookmark[4]{Hypnose}{custom-clerc-hypnose}\textbf{\uline{Hypnose (1/tour)}} : identique à l'aptitude des magiciens (voir page \pageref{magicien-hypnose}).

\bigskip

%++++++++APTITUDE
\pdfbookmark[4]{Domination}{custom-clerc-domination}\textbf{\uline{Domination (spécial)}} : identique à l'aptitude des guerriers. (voir page \pageref{guerrier-domination})

\bigskip

%++++++++APTITUDE
\pdfbookmark[4]{Perception extrasensorielle}{custom-clerc-esp}\textbf{\uline{Perception extrasensorielle (1/tour)}} : identique à l'aptitude des magiciens (voir page \pageref{magicien-ESP}).

\bigskip

%++++++++APTITUDE
\pdfbookmark[4]{Ajustement cellulaire}{custom-clerc-ajustement-cellulaire}\phantomsection\label{clerc-ajustement-cellulaire}\textbf{\uline{Ajustement cellulaire (spécial)}} : cette aptitude permet au possesseur de soigner les blessures et les maladies.

\bigskip

Le coût en points de force psionique pour soigner les blessures est de 2 points pour un point de dommages. Le coût pour soigner les maladies est une base de 20 points pour les maladies mineures, et doit être ajusté à la hausse par l'arbitre pour les maladies sévères ou les cas avancés.

\bigskip

Le nombre de points de dommages qui peuvent être soignés pendant une période de 24 heures par le possesseur de cette aptitude est dicté par le niveau de maîtrise qu'il possède : voir table ci-dessous.

\bigskip

\begin{tabular}{cc}
& \textbf{Points de dommages} \\
\textbf{Niveau de maîtrise} & \textbf{soignables en 24h} \\
premier     & 10 \\
deuxième    & 20 \\
troisième   & 30 \\
quatrième   & 40 \\
...         & ...\\
\end{tabular}

\bigskip

%++++++++APTITUDE
\pdfbookmark[4]{Contrôle de l'esprit sur le corps}{custom-clerc-cec}\textbf{\uline{Contrôle de l'esprit sur le corps (0)}} : identique à l'aptitude des guerriers (voir page \pageref{guerrier-controle-ESC}).

\bigskip

%++++++++APTITUDE
\pdfbookmark[4]{Changer le poids du corps}{custom-clerc-cpc}\textbf{\uline{Changer le poids du corps (1/tour)}} : identique à l'aptitude des guerriers (voir page \pageref{guerrier-changer-poids}).

\bigskip

%++++++++APTITUDE
\pdfbookmark[4]{Télépathie avec les animaux}{custom-clerc-telepathie-animale}\phantomsection\label{clerc-telepathie-animale}\textbf{\uline{Télépathie avec les animaux (2/tour)}} : cette aptitude donne au possesseur le pouvoir de communiquer avec des créatures conscientes par contact mental direct, mais elle ne permet pas de commander ou d'influencer la créature avec qui est établie la communication.

\bigskip

L'aptitude dépend du niveau de maîtrise de la personne dotée du pouvoir :

\bigskip

\begin{tabular}{cc}
\textbf{Niveau de maîtrise} & \textbf{Peut communiquer avec} \\
premier     & mammifères \\
deuxième    & oiseaux \\
troisième   & reptiles \& amphibiens \\
quatrième   & poissons et créatures similaires \\
cinquième   & insectes \\
sixième     & animaux \og monstrueux \fg \\
septième    & plantes \\
\end{tabular}

\bigskip

%++++++++APTITUDE
\pdfbookmark[4]{Réarrangement moléculaire}{custom-clerc-rearrange-mol}\textbf{\uline{Réarrangement moléculaire (5/tour)}} : identique à l'aptitude des guerriers (voir page \pageref{guerrier-rearrange-mol}).

\bigskip

%++++++++APTITUDE
\pdfbookmark[4]{Altération de l'aura}{custom-clerc-alteration-aura}\phantomsection\label{clerc-alteration-aura}\textbf{\uline{Altération de l'aura (spécial)}} : cette aptitude est étroitement reliée au sort \textit{Délivrance des malédictions} (voir page \pageref{sort-delivrance-malediction}) en ce qu'une malédiction placée sur quelque chose ou quelqu'un se distingue facilement par son aura.

\bigskip

L'individu possédant cette aptitude est capable de reconnaître l'aura défavorable et de l'altérer, mais le coût de la reconnaissance est de 1 point de force psionique par niveau de malédiction, et l'altération ne peut être faite qu'au coût additionnel de 5 points de force par niveau de malédiction.

\bigskip

%++++++++APTITUDE
\pdfbookmark[4]{Prémonition}{custom-clerc-premonition}\textbf{\uline{Prémonition (spécial)}} : identique à l'aptitude des magiciens (voir page \pageref{magicien-premonition}).

\bigskip

%++++++++APTITUDE
\pdfbookmark[4]{Projection télépathique}{custom-clerc-proj-telepat}\textbf{\uline{Projection télépathique (3/tour)}} : cette aptitude est étroitement reliée à la projection télépathique des magiciens (voir page \pageref{magicien-projection-telepathique}), excepté que le possesseur est capable d'envoyer des suggestions d'émotions basiques à deux fois le nombre de niveaux de créatures possible aux magiciens.

\bigskip

\begin{tabular}{cc}
                                &\textbf{Nombre de niveaux} \\
\textbf{Niveau de maîtrise}      & \textbf{de créatures influencées}\\
premier     & 2 (2x1)            \\
deuxième    & 6 (2x(1+2)))       \\
troisième   & 12 (2x(1+2+3))     \\
quatrième   & 20 (2x(1+2+3+4))   \\
...         & ...           \\
\end{tabular}

\bigskip

Note : un heaume de télépathie (voir page \pageref{objet-heaume-telepathie}) double le pouvoir et la portée de l'aptitude et octroie, en plus, au possesseur les effets d'un bonus de +4 sur son intelligence (comme pour les magiciens).

\bigskip

%++++++++APTITUDE
\pdfbookmark[4]{Marche dimensionnelle}{custom-clerc-marche-dim}\textbf{\uline{Marche dimensionnelle (spécial)}} : identique à l'aptitude des guerriers (voir page \pageref{guerrier-marche-dimensionnelle}).

\bigskip

%++++++++APTITUDE
\pdfbookmark[4]{Projection astrale}{custom-clerce-proj-astr}\textbf{\uline{Projection astrale (spécial)}} : identique à l'aptitude des magiciens (voir page \pageref{magicien-projection-astrale}).

\bigskip

%++++++++APTITUDE
\pdfbookmark[4]{Domination des masse}{custom-clerc-dom-masses}\textbf{\uline{Domination des masses (spécial)}} : cette aptitude permet au possesseur d'utiliser sa capacité de domination sur de multiples individus.

\bigskip

Le coût en points de force psionique est le même que celui utilisé pour \textit{Domination} (voir page \pageref{guerrier-domination}), mais cette aptitude permet au possesseur d'exercer sa dominance de manière continue après la dépense initiale, si bien qu'une dépense continuelle n'est pas nécessaire.

\bigskip

La domination de masse ne causera jamais d'acte entièrement contre la volonté collective dans tous les cas de figure.

\bigskip

Les possibles niveaux influencés et la durée de cette domination dépendent tous deux du niveau de maîtrise de l'individu qui possède cette aptitude.

\bigskip

Pour chaque niveau de maîtrise possédé, l'individu est capable de dominer 5 niveaux de créatures pour 2 tours, et au septième niveau de maîtrise, la période de domination devient une semaine entière, et ensuite, la période est étendue d'une semaine par niveau additionnel de maîtrise. Noter que les créatures extrêmement intelligentes ne peuvent être dominées, tout comme celles avec des personnalités très fortes ne peuvent pas être dominées avec succès pour une durée quelconque.

\bigskip

\begin{tabular}{ccc}
                                 &\textbf{Nombre de niveaux} &\\
\textbf{Niveau de maîtrise}      & \textbf{de créatures dominés} & \textbf{Durée} \\
premier             & 5     & 2 tours (20min)      \\
\textit{deuxième}            & \textit{10}    & \textit{2h}                    \\
\textit{troisième}           & \textit{20}    & \textit{6h}                    \\
\textit{quatrième}           & \textit{40}    & \textit{12h}                   \\
\textit{cinquième}           & \textit{80}    & \textit{1 journée}                \\
\textit{sixième}             & \textit{160}   & \textit{3.5 jours} \\
septième   & \textit{320}   & 1 semaine \\
huitième            & \textit{640}   & 2 semaines \\
neuvième            & \textit{1280}  & 3 semaines \\
... & ... & ...
\end{tabular}

\bigskip

Dans la table ci-dessus, les éléments en italique sont une proposition pour compléter les éléments incomplets fournis par les règles.

\bigskip

%++++++++APTITUDE
\pdfbookmark[4]{Voyage probabiliste}{custom-clerc-voyage-proba}\textbf{\uline{Voyage probabiliste (spécial)}} : au moyen de cette aptitude, le possesseur est capable de pénétrer dans des mondes parallèles et entrer dans les plans différents. Cela est extrêmement dangereux néanmoins, car cela correspond à de la projection astrale avec l'enveloppe corporelle emportée avec soi.

\bigskip

Le vent psychique affecte le voyageur probabiliste, comme s'il se projetait dans l'espace. Pour chaque probabilité ou plan croisé, 10 points d'énergie sont dépensés de manière psionique.

\bigskip

Le voyageur est capable d'entrer en communion avec des pouvoirs amicaux, par exemple -- ou risquer d'entrer dans des plans hostiles à son alignement, ou tenter d'explorer les probabilités suivant une ligne de conduite faite sienne.

\newpage
%- - - - - - - - - - - - - - - - - - - - - - - - - - - SUB SUB SECTION
\phantomsection\subsubsection*{\uline{\textit{Conséquences du gain d'aptitude psionique}}}
\addcontentsline{toc}{subsubsection}{Conséquences du gain d'aptitude psionique}

Acquérir une aptitude psionique peut avoir 2 impacts :
\begin{itemize}
\item Un impact sur les modes d'attaques et les modes de défenses (voir la section sur le combat psionique),
\item Un impact sur la baisse des caractéristiques ou avantages du personnage.
\end{itemize}

%- - - - - - - - - - - - - - - - - - - - - - - - - - - SUB SUB SECTION
\phantomsection\subsubsection*{\uline{\textit{Pertes de caractéristiques}}}
\addcontentsline{toc}{subsubsection}{Pertes de caractéristiques}

Au fur et à mesure que le personnage acquiert des aptitudes psioniques, un certain nombre de désavantages vont lui être imposés, comme des pertes de points de caractéristiques ou des pertes de niveaux de sorts.

\bigskip

La table suivante recense ces pertes par classe de personnage.

\bigskip

\begin{tabular}{lcp{12cm}}
& \textbf{Fréquence} & \\
\textbf{Classe} & \textbf{de la perte} & \textbf{Nature de la perte}\\
Guerrier & 4 aptitudes & 1 pt de Force \\
& 1 aptitude & 1 suivant est perdu \\
Magicien & 1 aptitude & Des niveaux de sort correspondant au nombre d'aptitudes psioniques* sont perdus, soit : \\
&& --- Première aptitude : 1 niveau de sorts (un sort de niveau 1), \\
&& --- Deuxième aptitude : 2 niveaux de sorts (un sort de niveau 2 ou deux sorts de niveau 1), \\
&& --- Etc. \\
Clerc & 1 aptitude & Des niveaux de sort correspondant au nombre d'aptitudes psioniques sont perdus (comme les magiciens) + un rang perdu dans la table de retournement des morts-vivants \\
&& Exemple : un clerc de niveau 5 avec une aptitude psionique repoussera les morts vivants comme un clerc de niveau 4. \\
Voleur & 4 aptitudes & 1 pt de Force + 1 point de Dextérité \\
& 1 aptitude & 1 suivant est perdu \\
\end{tabular}

\bigskip

* Jamais le magicien ne doit se souvenir de plus de sorts de haut niveau que de sorts de bas niveau, et s'il est capable d'utiliser des sorts du sixième niveau, il doit être capable de se souvenir d'au moins un sort de tous les autres cinq niveaux.

%- - - - - - - - - - - - - - - - - - - - - - - - - - - SUB SUB SECTION
\phantomsection\subsubsection*{\uline{\textit{Liens entre le monde psionique et la magie}}}
\addcontentsline{toc}{subsubsection}{Liens entre le monde psionique et la magie}
\label{custom-liens-aptitudes-sorts}

La table ci-dessous présente la liste des aptitudes psioniques reliées avec un sort. Cette table est utile dans le cadre des attaques psioniques sur créature non psionique (voir page \pageref{custom-attaque-non-psionique}).

\bigskip

\begin{tabular}{lcl}
\textbf{Aptitude psionique}                                     & \textbf{Livre}    & \textbf{Sort relié} \\
Altération de l'aura (p.\pageref{clerc-alteration-aura})        & OD\&D vol. 1       & Délivrance des malédictions (p. \pageref{sort-delivrance-malediction}) \\
Altération de la forme (p. \pageref{magicien-alteration-forme}) & OD\&D vol. 1       & Métamorphose (p. \pageref{sort-metamorphose}) \\
Agitation moléculaire (p. \pageref{magicien-agitation-mol})     & Eldritch Wizardry & Chauffer le métal (p. \pageref{sort-chauffe-metal}) \\
Clairaudience (p. \pageref{guerrier-clairaudience})             & OD\&D vol. 1       & Clairaudience (p. \pageref{sort-clairaudience}) \\
Clairvoyance (p. \pageref{guerrier-clairvoyance})               & OD\&D vol. 1       & Clairvoyance (p. \pageref{sort-clairvoyance}) \\
Hypnose (p. \pageref{magicien-hypnose})                         & Greyhawk          & Suggestion (p. \pageref{sort-suggestion}) \\
Lévitation (p. \pageref{guerrier-levitation})                   & OD\&D vol. 1       & Léviter (voir p. \pageref{sort-levitation}) \\
Perception extrasensorielle (p. \pageref{magicien-ESP})         & OD\&D vol. 1       & Perception extrasensorielle (p. \pageref{sort-esp}) \\
Porte dimensionnelle (p. \pageref{magicien-porte-dimensionnelle}) & OD\&D vol. 1     & Porte dimensionnelle (p. \pageref{sort-porte-dimensionnelle}) \\
Projection astrale (p. \pageref{guerrier-projection-astrale})   & Greyhawk          & Sort astral (p. \pageref{sort-astral}) \\
Téléportation (p.\pageref{magicien-teleportation})              & OD\&D vol. 1       & Téléporter (p. \pageref{sort-teleporter}) \\
\end{tabular}

\bigskip

Cette table est aussi utile pour savoir si une créature psionique détecte l'utilisation de sorts liés au monde psionique.

\bigskip

De la même façon, les objets magiques liés au monde psionique sont donnés ci-dessous.

\bigskip
\begin{tabular}{cll}
\textbf{Livre}  & \textbf{Objet magique}    & \textbf{Aptitude liée} \\
OD\&D vol. 2     & Amulette de captation (p. \pageref{objet-amulette-captation}) & Perception extrasensorielle (p. \pageref{magicien-ESP}) \\
OD\&D vol. 2     & Boules de cristal (p. \pageref{objet-boule-cristal}) & Perception extrasensorielle (p. \pageref{magicien-ESP}) \\
OD\&D vol. 2     & Epées                     & Multiples \\
OD\&D vol. 2     & Heaume de télépathie (p. \pageref{objet-heaume-telepathie})   & Télépathie (p. \pageref{objet-heaume-telepathie})  \\
Greyhawk        & Huile éthérée (p. \pageref{objet-huile-etheree}) & Forme éthérée (p. \pageref{magicien-forme-etheree}) \\
OD\&D vol. 2     & Médaillon de percep. extrasensorielle (p. \pageref{objet-medaillon-esp}) & Perception extrasensorielle (p. \pageref{magicien-ESP}) \\
\end{tabular}


%----------------------------------------------------- SUB SECTION
\phantomsection\subsection*{COMBAT PSIONIQUE}
\addcontentsline{toc}{subsection}{COMBAT PSIONIQUE}

Il y a basiquement deux types d'attaques psioniques :

\bigskip

\begin{enumerate}
\item Le type dans lequel il n'y a pas d'attaque en retour ;
\item Le type qui est un échange d'attaques et de défenses où deux créatures avec des aptitudes psioniques sont impliquées.
\end{enumerate}

\bigskip

Certains dispositifs magiques ou aptitudes psioniques limitées peuvent avoir des impacts sur le premier cas ci-dessus. Il est aussi possible que certaines créatures dotées d'aptitudes psioniques aient une forme d'attaque qui affectera uniquement les autres formes de vie dotées de capacités psioniques. Quand le combat psionique se produit, aucune autre action ne peut être effectuée.

\bigskip

De la même façon que le personnage psionique peut obtenir des aptitudes, il va aussi gagner des modes d'attaques et de défense psioniques pour pouvoir mener un combat psionique.

%- - - - - - - - - - - - - - - - - - - - - - - - - - - SUB SUB SECTION
\phantomsection\subsubsection*{\uline{\textit{Modes d'attaques et de défense psioniques}}}
\addcontentsline{toc}{subsubsection}{Modes d'attaques et de défense psioniques}
\label{custom-attaques}

La liste des modes d'attaques et de défense psionique est donnée ci-dessous.

\bigskip

\begin{tabular}{clccclc}
\multicolumn{2}{l}{\textbf{modes d'attaques, toutes classes}} & \textbf{Coût} && \multicolumn{2}{l}{\textbf{modes de défenses, toutes classes}} & \textbf{Coût} \\
A. & Onde de choc psionique   & 20  && F. & Esprit vide  & 1 \\
B. & Poussée de l'esprit  	  & 10  && G. & Bouclier de pensées  & 2 \\
C. & Coup de fouet sur l'ego  & 15  && H. & Barrières mentales & 4 \\
D. & Imposition d'identité    & 10  && I. & Forteresse intellectuelle  & 7 \\
E. & Écrasement psychique     & 25* && J. & Tour de volonté de fer  & 10 \\
\end{tabular}

\bigskip

*Si le joueur ne possède pas assez de points, altérer la probabilité de succès en \% en conséquence.

%- - - - - - - - - - - - - - - - - - - - - - - - - - - SUB SUB SECTION
\phantomsection\subsubsection*{\uline{\textit{Portée des modes d'attaques et de défense psioniques}}}
\addcontentsline{toc}{subsubsection}{Portée des modes d'attaques et de défense psioniques}
\label{custom-portee}

La portée des différentes attaques psioniques est donnée dans la table ci-dessous. Les attaques à portée moyenne font seulement 80\% des dommages précisés. Les attaques à longue portée font seulement 50\% des dommages précisés.

\bigskip

\begin{tabular}{lccc}
&\multicolumn{3}{c}{\textbf{Portée/Dommages}} \\
\textbf{Mode d'attaque} & \textbf{Courte} & \textbf{Moyenne} & \textbf{Longue} \\
Onde de choc psionique  & 1m -- 100\% & 2.5m -- 80\% &  4m -- 50\% \\
Poussée de l'esprit     & 3m -- 100\% &   6m -- 80\% &  9m -- 50\% \\
Coup de fouet sur l'ego & 2m -- 100\% &   4m -- 80\% &  6m -- 50\% \\
Imposition d'identité   & 4m -- 100\% &   8m -- 80\% & 12m -- 50\% \\
Écrasement psychique    & 2m -- 100\% &   -- -- 80\% & --  -- 50\% \\
\end{tabular}

\bigskip

La portée courte augmente de 1/3m (et les autres portées augmentent de la même façon proportionnellement) avec chaque niveau de maîtrise d'une capacité d'attaque\footnote{Voir page \pageref{exemple-complet} pour un exemple de mise en œuvre.}.

\bigskip

\begin{tabular}{ll}
\textbf{Mode de défense} & \textbf{Protection maximale pour} \\
Esprit vide & Individu seul \\
Bouclier de pensées & Individu seul \\
Barrière mentale & Individu seul \\
Forteresse intellectuelle & Cercle de 3m autour de l'individu \\
Tour de volonté de fer & Cercle de 1m autour de l'individu \\
\end{tabular}

\bigskip

%- - - - - - - - - - - - - - - - - - - - - - - - - - - SUB SUB SECTION
\phantomsection\subsubsection*{\uline{\textit{Gain de modes d'attaques et de défenses psioniques}}}
\addcontentsline{toc}{subsubsection}{Gain d'aptitudes}

Le personnage gagne des modes d'attaques et de défenses psioniques dans un des deux cas suivants :

\bigskip

\begin{itemize}
\item Il vient de calculer son potentiel psychique et vient de gagner sa première aptitude psionique : il gagne \textit{Onde de choc psionique} automatiquement ;
\item Il vient de gagner une ou deux nouvelles aptitudes psioniques à l'occasion d'un changement de niveau (voir page \pageref{aptitudes-gain}) : il doit consulter la table ci-dessous.
\end{itemize}

\bigskip

Si le personnage n'a pas assez d'aptitudes pour gagner un nouveau mode d'attaque ou de défense, il devra attendre le prochain changement de niveau pour réestimer sa capacité à en gagner. En revanche sa progression en niveau (relativement à sa classe de personnage) augmentera son niveau de maîtrise des aptitudes et des modes d'attaques et de défenses.


\bigskip

\begin{tabular}{lcc}
& \textbf{Nombre d'aptitudes} & \textbf{Nombre d'aptitudes} \\
\textbf{Classe de personnage} & \textbf{pour gain de mode d'attaque} & \textbf{pour gain de mode de défense} \\
Guerrier et Voleur & 5 & 4 \\
Magicien & 4 & 3 \\
Clerc & 4 & 3 \\
\end{tabular}

\bigskip

A noter que les modes d'attaques et de défenses sont acquis dans l'ordre des lettres du tableau, soit respectivement de A à E, et de F à J.

%- - - - - - - - - - - - - - - - - - - - - - - - - - - SUB SUB SECTION
\phantomsection\subsubsection*{\uline{\textit{Forces d'attaque et de défense psioniques}}}
\addcontentsline{toc}{subsubsection}{Forces d'attaque et de défense psioniques}

La force d'attaque psionique se calcule en additionnant :

\bigskip

\begin{itemize}
\item Le potentiel psychique du personnage,
\item Le nombre d'aptitudes psioniques multiplié par 2,
\item Le nombre d'attaques et de défenses psioniques multiplié par 5.
\end{itemize}

\bigskip

La force de défense psionique est égale à la force d'attaque psionique.

\bigskip

Les forces d'attaque psionique des monstres sont exposées dans les paragraphes traitant des monstres dotés de pouvoirs psioniques.

%- - - - - - - - - - - - - - - - - - - - - - - - - - - SUB SUB SECTION
\phantomsection\subsubsection*{\uline{\textit{Force psionique}}}
\addcontentsline{toc}{subsubsection}{Force psionique}

La force psionique est égale à deux fois la force d'attaque psionique (ou à la somme des forces d'attaque et de défense).

\bigskip

A noter que le heaume de télépathie accroît la force psionique totale de 40 points (utile pour les combats psioniques, voir plus loin), tout en conservant le pouvoir décrit dans la section \textit{Divers objets magiques} (page \pageref{objet-heaume-telepathie}).


%- - - - - - - - - - - - - - - - - - - - - - - - - - - SUB SUB SECTION
\phantomsection\subsubsection*{\uline{\textit{Utilisation des pouvoirs psioniques}}}
\addcontentsline{toc}{subsubsection}{Utilisation des pouvoirs psioniques}
\label{custom-utilisation-pouvoirs}

Utiliser des pouvoirs psioniques (aptitudes, attaques et défenses) consomme généralement des points de force psionique. Tous les pouvoirs sont donnés avec leur coût.

\bigskip

Le comptage des points consommés se fait de la manière suivante :

\bigskip

\begin{tabular}{lccc}
& \textbf{Points de force} & \textbf{Points de force} & \textbf{Points de force} \\
& \textbf{psionique} & \textbf{psionique d'attaque} & \textbf{psionique de défense} \\
Aptitude psionique & 100\%  & 50\%  & 50\%  \\
Attaque psionique & 100\%   & 100\% & 0\% \\
Défense psionique & 100\%   & 0\%   & 100\% \\
\end{tabular}

\bigskip

Cette table est applicable aux points de dommage psychiques encaissés lors des combats psioniques (voir page \pageref{custom-combat}), les points de dommages affectant la force psionique de la même façon qu'une aptitude (50/50)\footnote{Une autre façon de voir les choses serait, dans le combat psionique, de ne décrémenter que les points de force psionique de défense. Mais comme la table d'attaque psionique utilise la totalité de la force psionique, il nous paraît cohérent de décrémenter la force psionique totale du défenseur avec un prorata de 50/50. Mais le choix est laissé au MD d'avoir une autre interprétation des règles.}.

\bigskip

Note : nous avons toujours à tout moment, points de force psionique = points de force psionique d'attaque + points de force psionique de défense.

%- - - - - - - - - - - - - - - - - - - - - - - - - - - SUB SUB SECTION
\phantomsection\subsubsection*{\uline{\textit{Conséquence de l'utilisation des pouvoirs psioniques}}}
\addcontentsline{toc}{subsubsection}{Conséquence de l'utilisation des pouvoirs psioniques}
\label{custom-alerte-pouvoirs}

L'utilisation des pouvoirs psioniques alertera toute créature douée de pouvoirs psioniques dans la portée du pouvoir utilisé.

\bigskip

Si le pouvoir est utilisé de manière continue, les probabilités d'identifier la direction et le pouvoir eux-mêmes augmentent. La chance de base est de 10\% pour chaque pouvoir, et la chance augmente de 10\% pour chaque tour d'utilisation de la même aptitude. L'usage d'une aptitude différente rendra l'identification impossible mais pas la direction. Quand la direction est trouvée, la force relative du pouvoir peut être déterminée au tour suivant.

\bigskip

Les aptitudes supérieures alertent les autres créatures psioniques sur une portée double de celle de l'aptitude.

\bigskip

Le combat psionique (modes d'attaques et de défenses) alertent les créatures psioniques sur une portée triple de la capacité psionique (exception faite de \textit{Poussée de l'esprit} et \textit{Imposition d'identité} où la détection ne se fait au maximum qu'à la portée de la capacité).

\bigskip

Noter que les sorts qui dupliquent les pouvoirs psioniques ou y sont similaires attireront de même l'attention des créatures psioniques (voir table page \pageref{custom-liens-aptitudes-sorts}).

\bigskip

Cela inclut aussi les objets magiques qui tombent dans cette catégorie.

%- - - - - - - - - - - - - - - - - - - - - - - - - - - SUB SUB SECTION
\phantomsection\subsubsection*{\uline{\textit{Attaques sur les non psioniques}}}
\addcontentsline{toc}{subsubsection}{Attaques sur les non psioniques}
\label{custom-attaque-non-psionique}

Les attaques psioniques sur les créatures non-psioniques ne peuvent être faites que si l'attaquant a une force d'attaque psionique de plus de 120.

\bigskip

Une créature non psionique attaquée par une attaque psionique doit faire un jet de sauvegarde dépendant :

\bigskip

\begin{itemize}
\item De son intelligence,
\item De sa classe de personnage,
\item De sa race,
\item Du port d'un éventuel heaume télépathique (voir page \pageref{objet-heaume-telepathie}),
\item De son état, potentiellement le fruit d'une attaque précédente.
\end{itemize}

\bigskip

\begin{tabular}{c>{\centering\arraybackslash}p{2.6cm}>{\centering\arraybackslash}p{2.6cm}>{\centering\arraybackslash}p{2.6cm}l}
\textbf{Intelligence} & \multicolumn{3}{c}{\textbf{Jet de sauvegarde par portée d'attaque}} & \textbf{EFFET SI SAUVE-} \\
\textbf{du défenseur} & \textbf{Courte} & \textbf{Moyenne} & \textbf{Longue} & \textbf{GARDE ECHOUEE} \\
3--4    & 19 & 18 & 17 & Mort \\
5--7    & 17 & 16 & 15 & Coma 1--4 jours \\
8--10   & 15 & 14 & 13 & Sommeil 20--120 min. \\
11-12   & 13 & 12 & 11 & Etourdi 1--4 tours \\
13-14   & 11 & 10 &  9 & Confus 1--6 tours \\
15-16   &  9 &  8 &  7 & Furieux 1--8 tours \\
17      &  7 &  6 &  5 & Esprit affaibli \\
18      &  5 &  4 &  3 & Folie permanente \\
19      &  3 &  2 &  1 & Folie 1--4 semaines \\
20 \& + &  1 &  0 & --1 & Folie 2--12 jours \\
\end{tabular}

\bigskip

La table ci-dessous propose des ajustements au jet de sauvegarde.

\bigskip

\begin{tabular}{lcclc}
\multicolumn{2}{c}{\textbf{Bonus}} && \multicolumn{2}{c}{\textbf{Malus}} \\
\textbf{Classe}     &       && \textbf{Objet}   & \\
Magicien            & +1    && Médaillon de perception extrasensorielle (voir p. \pageref{objet-medaillon-esp})     & --5 \\
Clerc               & +2    && Sort relié aux pouvoirs psioniques** & --4 \\
\textbf{Races}      &       && \textbf{Etat}     & \\
Elfe                & +2    && Etourdi           & --3 \\
Nain                & +4    && Confus            & --2 \\
Halfling            & +4    && Enragé            & --1 \\
\textbf{Objets}     &       && Esprit affaibli   & *** \\
Heaume télépathique* & +4    && Fou             & **** \\
\end{tabular}

\bigskip

\begin{tabular}{rp{15cm}}
\multicolumn{1}{r}{*}       & Un heaume de télépathie porté par le défenseur étourdira l'attaquant pour trois tours si le défenseur réussit son jet de sauvegarde. \\
\multicolumn{1}{r}{**}      & Voir la liste page \pageref{custom-liens-aptitudes-sorts}. \\
\multicolumn{1}{r}{***}     & Traiter un esprit affaibli comme une personne à l'intelligence de 3--4 \\
\multicolumn{1}{r}{****}    & Les individus fous ne peuvent être attaqués psioniquement que par l'\textit{Imposition d'identité} (voir la section sur les attaques psioniques page \pageref{custom-attaques}).
\end{tabular}

%- - - - - - - - - - - - - - - - - - - - - - - - - - - SUB SUB SECTION
\phantomsection\subsubsection*{\uline{\textit{Le combat pas à pas}}}
\addcontentsline{toc}{subsubsection}{Le combat pas à pas}
\label{custom-combat}

Le combat se déroule de la manière suivante :

\bigskip

\begin{enumerate}
\item Déterminer si un des participants est surpris ;
\begin{itemize}
\item Le livret original ne proposant pas de règle spéciale, vous pouvez utiliser les règles originales de \texttt{OD\&D} en page \pageref{dd3-surprise} ;
\item Si l'un des participants est surpris, utiliser la table dans la section \textit{Surprise} ci-dessous pour résoudre le combat.
\end{itemize}
\item Déterminer l'initiative ;
\begin{itemize}
\item Voir ci-dessous ;
\end{itemize}
\item L'attaquant choisit son mode d'attaque et le défenseur son mode défense ;
\begin{itemize}
\item Si le défenseur n'a pas de modes de défenses ou n'a plus de points de force psionique, la table de combat dans la section \textit{Surprise} est utilisée ;
\item Dans l'autre cas, les dommages de la section \textit{Combat psionique complet avec dommages} sont enregistrés chez le défenseur (en prenant en compte le modificateur de portée de l'attaque psionique) ;
\end{itemize}\item Recommencer au point 2.
\end{enumerate}

\bigskip

Note : le défenseur devrait toujours choisir le mode défense le plus efficace pour contrer l'attaque. Mais il est possible d'envisager que, pour plus de réalisme, le mode de défense choisi le soit sans avoir connaissance du mode d'attaque. En effet, l'utilisation de tel ou tel mode d'attaque n'est pas visible chez l'attaquant. Le MD doit déterminer une façon de joueur et s'y tenir mais tous les choix sont possibles.

%- - - - - - - - - - - - - - - - - - - - - - - - - - - SUB SUB SECTION
\phantomsection\subsubsection*{\uline{\textit{Surprise}}}
\addcontentsline{toc}{subsubsection}{Surprise}
\label{custom-surprise}

Si un des individus participant au combat est surpris, l'attaque psionique est gérée dans la table ci-dessous.

\medskip

\begin{tabular}{c>{\centering\arraybackslash}p{1.6cm}>{\centering\arraybackslash}p{1.6cm}>{\centering\arraybackslash}p{1.6cm}>{\centering\arraybackslash}p{1.6cm}>{\centering\arraybackslash}p{1.6cm}>{\centering\arraybackslash}p{1.6cm}>{\centering\arraybackslash}p{1.6cm}}
\textbf{Force} &&&&&& \\
\textbf{d'attaque} & \multicolumn{7}{c}{\textbf{Potentiel psionique du défenseur}} \\
\textbf{psionique} & \textbf{01--10} & \textbf{11--25} & \textbf{26--50} & \textbf{51--75} & \textbf{76--90} & \textbf{91--99} & \textbf{00} \\
01--20      & E & E & 40 & 30 & 20 & 10 & 5 \\
21--40      & E & E & E  & 40 & 30 & 20 & 10 \\
41--60      & B & E & E  & E  & 40 & 30 & 20 \\
61--80      & B & E & E  & E  & E  & 40 & 30 \\
81--90      & H & B & E  & E  & E  & E  & 40 \\
91--00      & H & H & B  & E  & E  & E  & E \\
101--110    & M & H & H  & B  & E  & E  & E \\
111--120    & M & M & H  & H  & B  & B  & E \\
121 et plus & M & M & M  & H  & H  & H  & B \\
\end{tabular}

\bigskip

\begin{tabular}{lp{14.5cm}}
E = & Etourdi pour 5--20 tours, pas d'attaque psionique \\
B = & Blessure psychique, récupération en 1--6 mois, pas d'attaque psionique \\
H = & Handicapé psionique de manière permanente, perd toutes ses aptitudes \\
M = & Mort \\
5--40 = & Nombre de points d'attaque psionique perdus -- récupérés en 1--6 jours \\
Note : & L'attaque \textit{Coup de fouet sur l'ego} qui donne un résultat "M" veut dire stupidité et le résultat "H" doit être considéré comme "B". L'attaque \textit{Imposition psychique} avec un résultat de "B", "H", ou "M" signifie que le défenseur est sous le contrôle de l'attaquant jusqu'à ce qu'il soit libéré. \\
\end{tabular}

%- - - - - - - - - - - - - - - - - - - - - - - - - - - SUB SUB SECTION
\phantomsection\subsubsection*{\uline{\textit{Initiative}}}
\addcontentsline{toc}{subsubsection}{Initiative}
\label{custom-initiative}

Si aucun individu n'est surpris et que les opposants annoncent simultanément qu'ils attaquent de manière psionique (ou dans le cas où le monstre le fait automatiquement et le personnage annonce qu'il le fait), la séquence d'attaque est déterminée comme suit : chaque opposant fait un jet de pourcentage et ajoute le résultat à sa force d'attaque psionique. Le plus haut score attaque en premier.

%- - - - - - - - - - - - - - - - - - - - - - - - - - - SUB SUB SECTION
\phantomsection\subsubsection*{\uline{\textit{Combat psionique complet avec dommages}}}
\addcontentsline{toc}{subsubsection}{Combat psionique complet avec dommages}
\label{custom-combat}

La table ci-dessous propose la table complète de combat entre psioniques. Les chiffres montrés sont les points de dommages encaissés par la force psionique de l'adversaire (voir page \pageref{custom-utilisation-pouvoirs}).

\bigskip

Attention à bien prendre en compte les modificateurs de distance décrits en page \pageref{custom-portee} (100\% à portée courte, 80\% à portée moyenne et 50\% à portée longue).


\bigskip

En ce qui concerne l'utilisation du pouvoir \textit{Ecrasement psychique}, l'attaquant doit faire un jet de 1d100 sous le pourcentage indiqué (inférieur ou égal). Si le jet réussi, l'attaque tue instantanément le défenseur.

\bigskip

A noter que cette table utilise la force psionique totale de l'attaquant et non pas seulement la force psionique d'attaque.

\bigskip

\begin{tabular}{cl>{\centering\arraybackslash}p{2cm}>{\centering\arraybackslash}p{2cm}>{\centering\arraybackslash}p{2cm}>{\centering\arraybackslash}p{2cm}>{\centering\arraybackslash}p{2cm}}
\small\textbf{Force} & & \multicolumn{5}{c}{\small\textbf{Mode défensif}} \\
\small\textbf{psionique} & \small\textbf{Mode} & \small\textbf{Esprit} & \small\textbf{Bouclier} & \small\textbf{Barrière} & \small\textbf{Forteresse} & \small\textbf{Tour de vo-} \\
\small\textbf{totale} & \small\textbf{offensif} & \small\textbf{vide} & \small\textbf{de pensées} & \small\textbf{mentale} & \small\textbf{intellectuelle} & \small\textbf{lonté de fer} \\

01 & Onde de choc       &    2 & 3    & 3 & 1 & 0 \\
   & Poussée    &   10 & 3    & 0 & 0 & 1 \\
à  & Coup de fouet      &    6 & 2    & 0 & 0 & 0 \\
   & Imposition &    1 & 4    & 6 & 0 & 1 \\
20 & Écrasement & 01\% & -    & - & - & - \\

21 & Onde de choc       &    3 & 7    & 4 & 2 & 0 \\
   & Poussée    &   12 & 5    & 1 & 0 & 3 \\
à  & Coup de fouet      &    8 & 4    & 0 & 0 & 0 \\
   & Imposition &    2 & 5    & 8 & 1 & 2 \\
40 & Écrasement & 02\% & 01\% & - & - & - \\

41 & Onde de choc       &    4 & 9    & 5    & 3 & 0 \\
   & Poussée    &   14 & 7    & 02   & 1 & 4 \\
à  & Coup de fouet      &   10 & 6    & 0    & 0 & 0 \\
   & Imposition &    3 & 7    & 10   & 3 & 4 \\
60 & Écrasement & 04\% & 02\% & 01\% & - & - \\

61 & Onde de choc       &    6 & 11   & 7    & 4    & 0 \\
   & Poussée    &   16 & 9    & 4    & 2    & 5 \\
à  & Coup de fouet      &   13 & 9    & 1    & 0    & 1 \\
   & Imposition &    4 & 9    & 13   & 5    & 7 \\
80 & Écrasement & 08\% & 04\% & 02\% & 01\% & - \\

81 & Onde de choc       &    9 & 14   & 9    & 5    & 0  \\
   & Poussée    &   18 & 11   & 6    & 3    & 6  \\
à  & Coup de fouet      &   17 & 13   & 2    & 0    & 2  \\
   & Imposition &    6 & 11   & 16   & 8    & 10 \\
90 & Écrasement & 10\% & 06\% & 04\% & 01\% & -  \\

91  & Onde de choc       &   13 & 17   & 11   & 7    & 1    \\
    & Poussée    &   20 & 13   & 8    & 4    & 7    \\
à   & Coup de fouet      &   22 & 17   & 4    & 1    & 3    \\
    & Imposition &    8 & 14   & 19   & 11   & 13   \\
100 & Écrasement & 12\% & 08\% & 06\% & 02\% & 01\% \\

101 & Onde de choc       &   18 & 20   & 13   & 9    & 2    \\
    & Poussée    &   23 & 15   & 10   & 5    & 8    \\
à   & Coup de fouet      &   28 & 21   & 6    & 2    & 4 \\
    & Imposition &   10 & 17   & 23   & 15   & 18   \\
110 & Écrasement & 15\% & 10\% & 08\% & 03\% & 02\% \\

111 & Onde de choc       &   24 & 23   & 15   & 11   & 3    \\
    & Poussée    &   26 & 18   & 13   & 7    & 10   \\
à   & Coup de fouet      &   35 & 27   & 8    & 3    & 6    \\
    & Imposition &   13 & 21   & 27   & 19   & 24   \\
120 & Écrasement & 20\% & 14\% & 10\% & 05\% & 03\% \\

\end{tabular}

\begin{tabular}{cl>{\centering\arraybackslash}p{2cm}>{\centering\arraybackslash}p{2cm}>{\centering\arraybackslash}p{2cm}>{\centering\arraybackslash}p{2cm}>{\centering\arraybackslash}p{2cm}}
\small\textbf{Force} & & \multicolumn{5}{c}{\small\textbf{Mode défensif}} \\
\small\textbf{psionique} & \small\textbf{Mode} & \small\textbf{Esprit} & \small\textbf{Bouclier} & \small\textbf{Barrière} & \small\textbf{Forteresse} & \small\textbf{Tour de vo-} \\
\small\textbf{totale} & \small\textbf{offensif} & \small\textbf{vide} & \small\textbf{de pensées} & \small\textbf{mentale} & \small\textbf{intellectuelle} & \small\textbf{lonté de fer} \\

121  & Onde de choc       &   30 & 27   & 18   & 14   & 5    \\
     & Poussée    &   29 & 22   & 17   & 10   & 12   \\
à    & Coup de fouet      &   43 & 33   & 11   & 5    & 8    \\
     & Imposition &   17 & 25   & 31   & 23   & 30   \\
plus & Écrasement & 25\% & 18\% & 13\% & 07\% & 04\% \\
\end{tabular}

\bigskip

Quand l'un des combattants en est réduit à ne plus avoir de capacités défensives, toutes les attaques sur lui sont considérées comme devant utiliser la table proposée pour les combattants surpris (page \pageref{custom-surprise}.

%----------------------------------------------------- SUB SECTION
\phantomsection\subsection*{RESTAURATION DE L'ENERGIE PSIONIQUE}
\addcontentsline{toc}{subsection}{RESTAURATION DE L'ENERGIE PSIONIQUE}

Les points de force psionique dépensés peuvent être restaurés par l'arrêt total de toute activité psionique. La vitesse de restauration dépend du type d'activité non-psionique que le personnage psionique pratiquera :

\bigskip

\begin{tabular}{lp{0.3cm}c}
\textbf{Activité}                       && \textbf{Récupération de points de force psionique} \\
Marcher, parler \& activités identiques && 6 points/heure \\
Se reposer tranquillement               && 12 points/heure \\
Dormir                                  && 24 points/heure \\
\end{tabular}

\newpage
%----------------------------------------------------- SUB SECTION
\phantomsection\subsection*{EXEMPLE COMPLET}
\addcontentsline{toc}{subsection}{EXEMPLE COMPLET}
\label{exemple-complet}

%- - - - - - - - - - - - - - - - - - - - - - - - - - - SUB SUB SECTION
\phantomsection\subsubsection*{\uline{\textit{Accession aux pouvoirs psioniques}}}
\addcontentsline{toc}{subsubsection}{Accession aux pouvoirs psioniques}

Rustik, guerrier de niveau 4.
\bigskip

\begin{tabular}{cccccc}
\multicolumn{3}{c}{\textbf{Classe} : Guerrier} & \multicolumn{3}{c}{\textbf{Niveau} : 4} \\
\textbf{FOR} & \textbf{CON} & \textbf{DEX} & \textbf{INT} & \textbf{SAG} & \textbf{CHA} \\
16 & 13 & 10 & \textbf{15} & 9 & 8 \\
\end{tabular}

\bigskip

Il a réussi son jet de test psionique en tirant 92 sur 1d100. Il est humain et son intelligence est de 15. Il peut donc calculer son Potentiel psychique. Il tire un nouveau d100 et obtient 77.

\bigskip

\begin{tabular}{cc}
\textbf{Potentiel psychique} & \textbf{Chance de gagner une aptitude}\\
77 & Niveau x 11\% \\
\end{tabular}

\bigskip

Il dispose donc de 4 x 11\% = 44\% d'avoir une aptitude psionique.

\bigskip

Le joueur fait en séquence les actions suivantes :

\bigskip

Il lance 1d100 et obtient 31. Il lance 1d20 sur la table des aptitudes psioniques et obtient un 16 ce qui lui donne l'aptitude \textit{Projection astrale}. Il ne peut pas la prendre car c'est une aptitude supérieure et il n'a encore aucune aptitude. Il rejoue et obtient 1, ce qui lui donne \textit{Réduction} qui est bien une aptitude basique.

\bigskip

Il relance 1d100 sous son potentiel psychique cette fois, et obtient un 15. Son MD lui propose d'utiliser l'option 2 avec 1d50. Il fait 8 et 2 aux d10 ce qui lui donne 32. Il obtient A:13, \textit{Contrôle de l'esprit sur le corps}. Il peut la prendre dans la mesure où il n'a pas plus d'aptitudes supérieures que d'aptitudes inférieures.

\bigskip

Avec sa première aptitude psionique, Rustik gagne son premier mode d'attaque psionique : Onde de choc psionique.

\bigskip

\begin{tabular}{lcccc}
\textbf{Aptitudes psioniques} & \textbf{Niveau acquis} & \textbf{Maîtrise}  & \textbf{BAS/SUP} & \textbf{Malus} \\
Réduction                         & 4 & 1 & BAS & - \\
Contrôle de l'esprit sur le corps & 4 & 1 & SUP & - \\
\end{tabular}

\bigskip

\begin{tabular}{lc}
\textbf{Modes d'attaques \& de défenses} & \textbf{Onde de choc psionique} \\
Type                                & Attaque \\
Niveau acquisition                  & 4 \\
Niveau de maîtrise                  & 1 \\
Bonus portée                        & -- \\
Coût en points de force psionique   & 20 \\
Portée courte/Dommages              &   1m -- 100\% \\
Portée moyenne/Dommages             & 2.5m -- 80\% \\
Portée longue/Dommages              &   4m -- 50\%  \\
Protection (défense)                & -- \\
\end{tabular}

\bigskip

Il est temps maintenant pour Rustik de calculer sa force psionique d'attaque. Il doit additionner :

\bigskip

\begin{itemize}
\item Son potentiel psychique : 77,
\item Deux fois le nombre de ses aptitudes psioniques : 2 x 2 = 4,
\item Cinq fois le nombre de ses modes d'attaques et de défenses : 5 x 1 = 5.
\end{itemize}

\bigskip

Sa force d'attaque psionique est donc de 77 + 4 + 5 = 86, tout comme sa force de défense. Sa force psionique est le total des deux.

\bigskip

\begin{tabular}{cccc}
& \textbf{Force psionique} & \textbf{Force d'attaque} & \textbf{Force de défense}  \\
Normale & 172 & 86 & 86 \\
Courante & 172 & 86 & 86 \\
\end{tabular}


%\bigskip

%- - - - - - - - - - - - - - - - - - - - - - - - - - - SUB SUB SECTION
\phantomsection\subsubsection*{\uline{\textit{Progression sans gain d'aptitude}}}
\addcontentsline{toc}{subsubsection}{Progression sans gain d'aptitude}

Plus tard, Rustik passe au niveau 5. Sa chance d'obtenir une aptitude est maintenant de 5 x 11 = 55 \%. Il joue et rate. Il n'obtient aucune nouvelle aptitude psychique et donc ne progresse pas à ce niveau, mais son niveau de maîtrise augmente sur ce qu'il possède déjà.

\bigskip

\begin{tabular}{lcccc}
\textbf{Aptitudes psioniques} & \textbf{Niveau acquis} & \textbf{Maîtrise}  & \textbf{BAS/SUP} & \textbf{Malus} \\
Réduction                         & 4 & \textbf{2} & BAS & - \\
Contrôle de l'esprit sur le corps & 4 & \textbf{2} & SUP & - \\
\end{tabular}

\bigskip

\begin{tabular}{lc}
\textbf{Modes d'attaques \& de défenses} & \textbf{Onde de choc psionique} \\
Type                                & Attaque \\
Niveau acquisition                  & 4 \\
Niveau de maîtrise                  & \textbf{2} \\
Bonus portée                        & +1/3m \\
Coût en points de force psionique   & 20 \\
Portée courte/Dommages              &   1m 1/3 -- 100\% \\
Portée moyenne/Dommages             & 3.5m -- 80\% \\
Portée longue/Dommages              &   5m 1/3 -- 50\%  \\
Protection (défense)                & -- \\
\end{tabular}

\bigskip

N'ayant pas changé son nombre d'aptitudes, il n'a pas à recalculer sa force psionique.

%- - - - - - - - - - - - - - - - - - - - - - - - - - - SUB SUB SECTION
\phantomsection\subsubsection*{\uline{\textit{Progression avec gain d'aptitude}}}
\addcontentsline{toc}{subsubsection}{Progression avec gain d'aptitude}

Plus tard encore, Rustik passe au niveau 6.

\bigskip

\begin{tabular}{cc}
\textbf{Potentiel psychique} & \textbf{Chance de gagner une aptitude}\\
77 & 6 x 11\% = 66\% \\
\end{tabular}

\bigskip

Il lance 1d100 et obtient 23 ce qui est inférieur ou égal à 66. Il choisit de déterminer son aptitude avec 1d20. Il tire un 18 mais ne peut pas l'avoir car il aurait plus d'aptitudes supérieures que d'aptitudes basiques. Il rejoue et fait 11, ce qui octroie le pouvoir \textit{Clairvoyance}.

\bigskip

Il relance 1d100 et fait 30 ce qui inférieur à son Potentiel psychique. Son MD lui propose l'option 1. Il accepte et fait un 10, ce qui lui donne A:10, Télékinésie.

\bigskip

Il obtient sa quatrième aptitude. Il doit perdre un point de force.

\bigskip

A la quatrième aptitude psionique, étant un guerrier, il gagne son premier mode défense psionique : Esprit vide.

\bigskip

\begin{tabular}{lcccc}
\textbf{Aptitudes psioniques} & \textbf{Niveau acquis} & \textbf{Maîtrise}  & \textbf{BAS/SUP} & \textbf{Malus} \\
Réduction                         & 4 & 3 & BAS & - \\
Contrôle de l'esprit sur le corps & 4 & 3 & SUP & - \\
Clairvoyance                      & 6 & 1 & BAS & - \\
Télékinésie                       & 6 & 1 & SUP & -1 FOR\\
\end{tabular}

\bigskip

\begin{tabular}{lcc}
\textbf{Modes d'attaques \& de défenses} & \textbf{Onde de choc psionique} & \textbf{Esprit vide}\\
Type                                & Attaque           & Défense\\
Niveau acquisition                  & 4                 & 6 \\
Niveau de maîtrise                  & 3                 & 1\\
Bonus portée                        & +1m               & --\\
Coût en points de force psionique   & 20                & 1 \\
Portée courte/Dommages              & 2m -- 100\%       & --\\
Portée moyenne/Dommages             & 4.5m -- 80\%      & -- \\
Portée longue/Dommages              & 6m -- 50\%        & --\\
Protection (défense)                & --                & Individu seul \\
\end{tabular}

\bigskip

Le tableau de ses caractéristique est modifié en conséquence.

\bigskip

\begin{tabular}{cccccc}
\multicolumn{3}{c}{\textbf{Classe} : Guerrier} & \multicolumn{3}{c}{\textbf{Niveau} : 6} \\
\textbf{FOR} & \textbf{CON} & \textbf{DEX} & \textbf{INT} & \textbf{SAG} & \textbf{CHA} \\
\textbf{16-1=15} & 13 & 10 & 15 & 9 & 8 \\
\end{tabular}

\bigskip

La force psionique est aussi modifiée car elle dépend du profil psionique du personnage. Pour recalculer sa force psionique d'attaque, nous devons additionner :

\bigskip

\begin{itemize}
\item Son potentiel psychique : 77,
\item Deux fois le nombre de ses aptitudes psioniques : 2 x 4 = 8,
\item Cinq fois le nombre de ses modes d'attaques et de défenses : 5 x 2 = 10.
\end{itemize}

\bigskip

Sa force d'attaque psionique est donc de 77 + 8 + 10 = 95, tout comme sa force de défense.

\bigskip

\begin{tabular}{cccc}
& \textbf{Force psionique} & \textbf{Force d'attaque} & \textbf{Force de défense}  \\
Normale     & 190 & 95 & 95 \\
Courante    & 190 & 95 & 95 \\
\end{tabular}

\newpage
%----------------------------------------------------- SUB SECTION
\phantomsection\subsection*{Système de combat alternatif complet}
\addcontentsline{toc}{subsection}{Système de combat alternatif complet}
\label{custom-combat-alternatif}

Ce système est basé sur les capacités offensives et défensives des combattants ; des choses comme la vitesse, la férocité ou les armes employées par les monstres attaquant sont incluses dans les matrices. Deux tableaux sont proposés : un pour les hommes contre les hommes ou les monstres, et un pour les monstres (incluant les kobolds, les gobelins, les orcs, et.) contre les hommes.

\bigskip

Nous proposons un tableau consolidé des matrices d'attaque pour le cas des personnages joueurs attaquant avec une arme de mêlée ou de jet. L'attaquant doit jeter 1d20, ajouter les modificateurs et comparer avec le score à atteindre pour toucher la classe d'armure correspondante.

\bigskip

\begin{tabular}{cccccccccccc}
\textbf{Niv. } & \textbf{Niv.} & \textbf{Niv.} & \multicolumn{8}{c}{\textbf{Classe d'armure du défenseur}} \\
\textbf{guerrier}   & \textbf{magicien}   & \textbf{clerc}   & \textbf{9} & \textbf{8} & \textbf{7} & \textbf{6} & \textbf{5} & \textbf{4} & \textbf{3} & \textbf{2} \\
1--3   & 1--5   & 1--4   & 10 & 11 & 12 & 13 & 14 & 15 & 16 & 17 \\
4--6   & 6--10  & 5--8   &  8 &  9 & 10 & 11 & 12 & 13 & 14 & 15 \\
7--9   & 11--15 & 9--12  &  5 &  6 &  7 &  8 &  9 & 10 & 11 & 12 \\
10--12 & 16--20 & 13--16 &  3 &  4 &  5 &  6 &  7 &  8 &  9 & 10 \\
13--15 & 21--25 & 17--20 &  1 &  2 &  3 &  4 &  5 &  6 &  7 &  8 \\
16\&+  & 26\&+  & 21\&+  &  1 &  1 &  1 &  1 &  2 &  3 &  4 &  5 \\
\multicolumn{11}{l}{\textbf{Modificateurs d'attaques dus aux armes de mêlée (à ajouter au d20)}} \\
\multicolumn{3}{l}{Dague*}            & -3 & -3 & -1 & -1 &  0 &  0 & +1 & +2 \\
\multicolumn{3}{l}{Hache à main}      & -3 & -2 & -1 & -1 &  0 &  0 & +1 & +1 \\
\multicolumn{3}{l}{Masse}             &  0 & +1 &  0 &  0 &  0 &  0 &  0 &  0 \\
\multicolumn{3}{l}{Marteau}           &  0 & +1 &  0 & +1 &  0 &  0 &  0 &  0 \\
\multicolumn{3}{l}{Epée*}             & -2 & -1 &  0 &  0 &  0 &  0 &  0 & +1 \\
\multicolumn{3}{l}{Piolet}            & +2 & +3 & +2 & +3 &  0 &  0 &  0 &  0 \\
\multicolumn{3}{l}{Hache de bataille} & -1 &  0 & +1 & +1 &  0 &  0 &  0 &  0 \\
\multicolumn{3}{l}{Masse à pointes}   &  0 &  0 & +1 & +2 & +1 & +1 & +2 & +2 \\
\multicolumn{3}{l}{Fléau d'armes}     & +2 & +2 & +1 & +2 & +1 & +1 & +1 & +1 \\
\multicolumn{3}{l}{Lance*}            & -2 & -1 & -1 & -1 &  0 &  0 &  0 &  0 \\
\multicolumn{3}{l}{Arme d'hast*}      & -1 &  0 &  0 & +1 & +1 & +2 & +2 & +2 \\
\multicolumn{3}{l}{Hallebarde*}       &  0 & +1 & +1 & +2 & +1 &  0 &  0 &  0 \\
\multicolumn{3}{l}{Epée à deux mains} & +1 & +2 & +3 & +3 & +2 & +2 & +2 & +2 \\
\multicolumn{3}{l}{Lance de tournoi}  &  0 &  0 & +1 & +2 & +3 & +3 & +3 & +3 \\
\multicolumn{3}{l}{Pique}             & -1 &  0 &  0 &  0 &  0 &  0 &  0 &  0 \\
\multicolumn{11}{l}{\textbf{Modificateurs d'attaques dus aux armes de jet (à ajouter au d20)}} \\
\multicolumn{3}{l}{Arc court (15)}       & \footnotesize-3-5-7 & \footnotesize-2-3-5 & \footnotesize0-1-2 & \footnotesize0 0-1 & \footnotesize+1 0 0 & \footnotesize+2+1 0 & \footnotesize+2+1 0 & \footnotesize+2+1 0 \\
\multicolumn{3}{l}{Arc monté (18)}       & \footnotesize-3-4-7 & \footnotesize-2-3-5 & \footnotesize0-1-2 & \footnotesize0 0-1 & \footnotesize+1 0 0 & \footnotesize+2+1 0 & \footnotesize+2+1 0 & \footnotesize+3+2+1 \\
\multicolumn{3}{l}{Petite arbalète (18)} & \footnotesize-3-5-7 & \footnotesize-2-3-5 & \footnotesize0-1-4 & \footnotesize0 0-1 & \footnotesize+2+1 0 & \footnotesize+3+1 0 & \footnotesize+3+1 0 & \footnotesize+3+2+1 \\
\multicolumn{3}{l}{Arc long (21)}        & \footnotesize-2-3-5 & \footnotesize 0-2-4 & \footnotesize0 0-1 &\footnotesize+2+1 0 & \footnotesize+3+2+1 & \footnotesize+3+2+1 & \footnotesize+3+2+1 & \footnotesize+3+2+1 \\
\multicolumn{3}{l}{Arc composite (24)}   & \footnotesize-3-4-5 & \footnotesize 0-3-4 & \footnotesize0-1-2 &\footnotesize+2 0-1 & \footnotesize+3+1 0 & \footnotesize+3+2+1 & \footnotesize+3+2+1 & \footnotesize+3+2+1 \\
\multicolumn{3}{l}{Arbalète lourde (24)} & \footnotesize-1-2-3 & \footnotesize 0-1-3 &\footnotesize+1 0-1 &\footnotesize+2 0 0 & \footnotesize+3+1 0 & \footnotesize+4+2+1 & \footnotesize+4+2+1 & \footnotesize+4+3+2 \\
\multicolumn{3}{l}{Arquebuse (18)}       &  \footnotesize0-1-3 & \footnotesize+1 0-1 &\footnotesize+2 0 0 &\footnotesize+2+1 0 & \footnotesize+3+2 0 & \footnotesize+3+2 0 & \footnotesize+3+2 0 & \footnotesize+3+2 0 \\
\end{tabular}

\newpage

Notes :

\bigskip

\begin{itemize}
\item Les nombres entre parenthèses sur les armes de jet sont les portées maximales ; les nombres montrés dans le tableau sont les modificateurs dus aux armes dans le cas des portées courte (premier tiers de la portée), moyenne (tiers suivant de la portée) et longue (dernier tiers de la portée).
\item Les hommes normaux équivalent à des guerriers de niveau 1.
\item Les voleurs sont dans le même schéma que les clercs.
\end{itemize}

\bigskip

