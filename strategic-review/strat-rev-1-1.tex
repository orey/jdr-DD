%==========================================================================SECTION
\newpage
\phantomsection\section*{Volume 1 numéro 1}
\addcontentsline{toc}{section}{Volume 1 numéro 1, printemps 1975}
\label{sr11}

\bigskip

\begin{center}
{\LARGE \textbf{Printemps 1975}}
\end{center}

\bigskip\bigskip

\begin{center}
\underline{\underline{CHRONIQUE SPECIALE NUMERO UN}}
\end{center}

\bigskip

\begin{center}
\underline{AVENTURES EN DONJONS EN SOLO}
\end{center}

\bigskip

par gary Gygax, avec des remerciements spéciaux à George A. Lord

Tests préliminaires : Robert Kuntz et Ernest Gygax

\medskip

Bien qu'il ait été possible pour les enthousiastes de jouer des parties de DONJONS \& DRAGONS en solo par l'utilisation des \textit{Aventures dans les régions sauvages}\footnote{Troisième livret de OD\&D (NdT).}, il n'a pas été publié de méthode uniforme d'exploration de donjons, malgré le fait qu'on ait requis de l'arbitre de la campagne, jusqu'à présent, de concevoir les niveaux de donjons. Au travers de la série de tables suivante (et d'un nombre considérable de jets de dés), il est maintenant possible de s'aventurer seul au travers d'infinis labyrinthes de donjons ! Au bout d'un certain temps, je suis certain qu'il y aura, malgré tout, des ressemblances dans cette aventure, et pour cette raison, un système d'échange d'enveloppes scellées pour des pièces spéciales et des tours/pièges est vivement conseillé. Ces enveloppes peuvent venir de n'importe quel joueur et contient des monstres et des trésors, un complexe complet de pièces (déroulé petit à petit), des artéfacts anciens et ainsi de suite. Toutes les enveloppes devraient indiquer à quel niveau les contenus sont destinés et pour quel endroit, par exemple\footnote{Gary Gygax utilise "i.e." parfois dans le sens de "id est", soit "c'est-à-dire", mais aussi souvent dans le sens de par exemple, ce qui est le cas ici (NdT).} une chambre, une pièce, des couloirs larges de 7m, etc. Maintenant, libérez votre copie de D \& D, vos dés et plein de papier graphe et amusez-vous !

%reprendre ici


The upper level above the dungeon in which your solo adventures
are to take place should be completely planned out, and it is a good
idea to use the outdoor encounter matrix to see what lives where (a
staircase discovered later just might lead right into the midst of what-
ever it is). The stairway down to the first level of the dungeon should
be situated in the approximate middle of the upper ruins (or whatever
you have as upper works).




%TODO réimporter la traduction du combat D&D