%==========================================================================SECTION
\newpage
\phantomsection\section*{Volume 1 numéro 2}
\addcontentsline{toc}{section}{Volume 1 numéro 2, été 1975}
\label{sr12}

\bigskip

\begin{center}
{\LARGE \textbf{Printemps 1975}}
\end{center}

\bigskip\bigskip

\begin{center}
QUESTIONS LES PLUS FREQUEMMENT POSEES A PROPOS DES REGLES DE DONJONS \& DRAGONS
\end{center}

La limitation en termes d'espace imprimable (et il a été difficile de faire tenir tout le contenu que nous avions dans trois livrets !) nous a forcé à passer rapidement sur certaines sections, en espérant que cela ne causerait pas de problèmes excessifs aux lecteurs. Tandis que le nombre de lettres avec des questions sur D \& D indique que notre assomption était correcte, même un ou deux pourcents des lecteurs représente une proportion trop grande d'acheteurs insatisfaits, ce qui fait que nous offrons, par le présent article, un peu plus de détails dans les sections où les questions sont le plus fréquemment posées. De plus, quelques erreurs ont été corrigées au moyen de feuilles additionnelles dans le dernier tirage de D \& D. Ceux d'entre vous qui disposent d'un ensemble de règles qui ne contiennent pas ces corrections, peuvent en obtenir un exemplaire en envoyant simplement une enveloppe réponse timbrée à TSR demandant les "Corrections D \& D".

\medskip

\uline{Combat} : CHAINMAIL est principalement un système pour des combats 1:20\footnote{
    Dans les wargames de l'époque, les conventions de mise à l'échelle sont souvent notées "x:y", par exemple le 1:20 de \texttt{Chainmail}. Cette façon d'écrire correspond à un nombre x de pouces sur le plan représentant un nombre y de pieds dans la réalité (\textit{inch-to-the-foot}). 1:20 correspond donc à 1cm pour 2.4m (NdT).
}, bien qu'il fournisse aussi une représentation basique du combat homme à homme. Les sections "Homme-à-Homme" et "Supplément Fantastique" de Chainmail fournissent des systèmes pour des actions de petite taille sur table. Le système standard de Chaimail est conçu pour des actions plus grandes ou des types humanoïdes sont impliqués, soit des kobolds, des gobelins, des nains, des orcs, des elfes, des hommes, des hobgbelins, etc. Il est suggéré que le système alternatif de D \& D soit utilisé pour résoudre les mêlées importantes où les personnages principaux sont concernés, ainsi que celles qui impliquent les monstres les plus forts.

\medskip

Quand le combat fantastique a lieu, il y a normalement un seul échange d'attaques par round et, à moins que les règles ne disent le contraire, un dé à six faces est utilisé pour déterminer le nombre de points de dommages encaissé quand une attaque réussit. Le type d'armes n'est pas considéré, sauf pour le cas des armes magiques. Un super-héro, par exemple, attaquerait huit fois seulement s'il combattait contre des hommes normaux (ou des créatures ayant basiquement la même force, soit des kobolds, des gobelins, des gnomes, des nains, et ainsi de suite).

\medskip

Les considérations comme les tables de types d'armes, de dommages par type d'arme, et de dommages par attaque de monstre, apparaissent dans le premier livret ajouté à la série D \& D -- SUPPLEMENT I, GREYHAWK, qui devrait être disponible au moment où cette publication paraît, ou peu de temps après.

\medskip

L'initiative est toujours vérifiée. La surprise permet naturellement une première attaque dans beaucoup de cas. Pour continuer le propos, l'initiative est simplement le fait de jeter deux dés (en assumant que c'est le nombre de combattants) avec le plus haut score gagnant la première attaque ce round. Les scores aux dés sont ajustés par la dextérité et ainsi de suite\footnote{
    Chaque combattant jette son propre dé et, suite à ajustements (de caractéristique et de circonstances), le plus haut score gagne la première attaque. L'ajustement de dextérité est donné dans le livret 1, \texttt{Men \& Magic} de \texttt{OD\&D} page 11, pour les armes à distance uniquement. (NdT).
}.

\medskip

\uline{Exemple de combat} :

\medskip

10 ORCS\footnote{
    Page 3 du livret 2 de \texttt{OD\&D}, \texttt{Montsres \& Treasures} (NdT).
} surprennent un Héros\footnote{
    Un Héros est un guerrier de niveau 4, page 16 du manuel \texttt{Men \& Magic} de \texttt{OD\&D} (Ndt).
} solitaire, errant perdu dans les donjons, mais le jet de dés révèle qu'ils sont à une distance de 10m au moment de la détermination de la surprise ; donc ils utilisent leur initiative pour se rapprocher et être à distance de mêlée\footnote{
    Ils utilisent le round de surprise pour se rapprocher (NdT).
}. Le jet d'initiative est maintenant fait\footnote{
    Pour le deuxième round (NdT).
}. Le Héros obtient un 3, plus 1 pour sa haute \uline{dextérité}, ce qui donne 4\footnote{
    On peut considérer que Gygax interprète doublement les règles de \texttt{OD\&D} : tout d'abord, il utilise la description de la caractéristique Dextérité de la page 11 de \texttt{Men \& Magic} "vitesse dans les actions par exemple tirer le premier" (\textit{speed with actions such as firing first}) et le tableau des bonus et malus de la même page pour avoir le bonus de +1 dû à la haute dextérité du Héros ; ensuite, il applique ce bonus au jet d'initiative qui, de fait, n'est jamais expliqué dans \texttt{OD\&D} mais qui est expliqué dans les premières pages de \texttt{Chainmail} : chaque camp lance 1d6 et le plus haut score gagne l'initiative. La correction de l'initiative par le modificateur de dextérité apparaîtra plus tard officiellement dans les règles de la version 3 de \texttt{D\&D}, sachant que, dans cette version, le d20 remplace le d6, car le jet l'initiative est en fait un jet de dextérité (NdT).
}. Les Orcs font 6 et même si l'on compte -1 pour leur manque de dextérité (optionnel), ils obtiennent tout de même la première attaque. Comme ils surpassent en nombre de très loin leur opposant, il est probable qu'ils tenteront de le maîtriser au lieu de le tuer, ce qui signifie que chaque toucher réussi sera compté comme une tentative d'agripper le Héros :

\medskip

\begin{itemize}
\item Armure supposée pour le Héros : cotte de maille \& bouclier -- CA 4.
\item Score requis pour toucher une CA 4 -- 15 (pour des monstres avec 1 DV)\footnote{
    Page 20 de \texttt{OD\&D Men \& Magic} (NdT).
}.
\item \uline{Seulement 5 Orcs peuvent attaquer}, comme ils n'ont pas eu le temps d'encercler.
\end{itemize}

\medskip

Supposons les scores suivants pour les attaques des Orcs :

\medskip

Orc n°1 - 06 ; n°2 - 10 ; n°3 - 18 ; n°4 - 20 ; n°5 - 03.

\medskip

Deux des Orcs ont agrippé le Héros, et si son score avec 4 dés est inférieur à leur score avec 2 dés, il a été maîtrisé sans aucune chance de se libérer\footnote{
    Cette façon de compter est conforme à une phrase qui fait beaucoup débat dans le courant OSR, page 5 de \texttt{Monsters \& Treasures} :  "Les capacités d'attaque/de défense contre un homme normal doivent être vus comme simplement autoriser un jet de dés comme pour un "type homme" pour chaque dé de vie, avec tous les bonus n'étant donnés qu'à une des attaques, par exemple un Troll attaquerait 6 fois, une fois avec un bonus de +3 au jet d'attaque." (\textit{Attack/Defense capabilities versus normal men are simply a matter of allowing one roll as a man-type for every hit die, with any bonuses being given to only one of the attacks, i.e. a Troll would attack six times, once with a +3 added to the die roll.}). Elle est issue de la partie combat Homme-à-Homme (Man-to-Man combat) de \texttt{Chainmail}.

    Extrapolons. Comme le Héros dispose de 4 DV, il pourrait avoir la possibilité de répartir ses quatre attaques et ses quatre dés de dommages (rappelons que dans \texttt{OD\&D}, toutes les armes font 1d6 de dommages) comme il le souhaite (1 attaque à 4 dés de dommages, 2 à 2 dés, 4 à 1 dés, etc., entre les différents adversaires). Cette relation étroite entre niveau, dés de vie et dés de dommages des personnages et dés de vie, dés de dommages des monstres montre un système de jeu très différent de ce qu'il fut ensuite, car disposer de 4 attaques par round pour un Héros (niveau 4) et pouvoir les répartir sur plusieurs adversaires, permet de gérer des mêlées bien plus complexes et ayant un côté tactique très amusant. Cette méthode permet aussi d'équilibrer facilement les rencontres. A noter que dans le  combat Homme-à-Homme, les progressions des différentes classes de personnages en termes de nombre d'attaques sont aussi très intéressantes (NdT).
}. Si les scores sont ex-aequo, ils se bagarrent avec le Héros toujours debout, mais ce dernier sera incapable de se défendre avec son arme. Si le Héros fait plus que les Orcs, utilisez la différence positive pour se débarrasser de ses attaquants, par exemple, le Héros fait 15 et les Orcs n'ont fait que 8, donc le Héros en a jeté deux sur le côté, les \uline{assommant} pour 7 tours à répartir entre eux.

\medskip

\begin{itemize}
\item Round 2 : l'initiative va au Héros.
\item Score requis pour battre les Orcs -- 11 (guerrier du 4\textsuperscript{ème} niveau contre CA 6).
\end{itemize}

\medskip

Supposons le score suivant pour le Héros. Noter qu'il a droit à une attaque pour chaque niveau de combat, car le ration de un Orc contre le Héros est de \foreignlanguage{english}{1:4}, donc tout cela est traité comme une mêlée normale (non fantastique), comme tous les combats dans lesquels le score d'un côté est basé sur un dé de dommages ou moins\footnote{
    Gygax fait encore référence au combat Homme-à-Homme de \texttt{Chainmail} avec des Orcs à 1 DV (NdT).
}.

\medskip

Héros : 19 ; 01 ; 16 ; 09. Deux coups ratés sur 4 coups exécutés. 8 Orcs peuvent possiblement être atteints. Un dé à 8 faces est utilisé pour déterminer ceux qui ont été touchés. Supposons qu'un 3 et un 8 aient été tirés. Pour les Orcs numéros 3 et 8, les dés sont jetés pour déterminés leurs points de dommages, et ils obtiennent 3 et 4 respectivement. L'Orc numéro 3 prend 6 points de dommages et est tué. L'Orc numéro 8 prend 1 point de dommages et est capable de combattre\footnote{
    L'Orc numéro 3 prend 6 = 3 + 3 de dommages, le premier 3 est le résultat du d6 de dommages et le second 3 serait le bonus aux dommages d'un guerrier de force 18 page 7 de \texttt{Greyhawk}. L'Orc ayant 1DV au maximum, il est tué. L'orc numéro 8 prend 1 = 4 - 3 de dommages 4 étant le résultat du jet du d6, et -3 étant un malus dont nous ne connaissons pas l'origine (NdT).
}.

\medskip

\begin{itemize}
\item Tous les 7 Orcs survivants/non-assommés sont maintenant capables d'attaquer.
\end{itemize}

\medskip

Les tentatives de maîtriser le Héros continuent, et pas moins de quatre Orcs sont capables d'attaquer le Héros depuis des positions pour lesquelles le bouclier ne peut pas être utilisé, et donc sa classe d'armure est considérée comme étant de 5, et les Orcs qui attaquent pas derrière ajoutent +2 à leurs dés pour toucher. Dans ce cas, il est vraisemblable que les Orcs captureront le Héros.

\medskip

Les jets de sauvegarde pour les monstres sont les mêmes que pour les types et niveaux d'hom-mes\footnote{
    Le premier type d'homme est un homme normal (1DV) ce qui correspond à un niveau 0, le niveau 1 des personnages étant de 1DV+1 (NdT).
}, ce qui signifie qu'un balrog gagnerait le jet de sauvegarde le plus favorable pour lui, soit celui d'un guerrier du 10ème niveau, soit celui d'un magicien du 12ème niveau (la deuxième solution s'appuyant sur la résistance magique du balrog). Un troll serait l'égal d'un guerrier du 7ème niveau comme il a 6 dés +3 de vie, donc virtuellement sept dés.

\medskip

Moral : ce facteur est rarement considéré. Les joueurs, représentant basiquement seulement leur propre personnage et quelques autres, ont leur moral personnel en réalité. Les monstres stupides combattent jusqu'à la mort. Occasionnellement cependant, il est nécessaire de vérifier soit les troupes servant le groupe (à n'importe quel propos), soit le moral des monstres stupides. C'est une décision dépendant strictement de l'arbitre. Le système utilisé est aussi à la discrétion de l'arbitre, bien qu'il y en ait un dans CHAINMAIL qui puisse être utilisé, ou l'arbitre peut simplement jeter 2 dés -- un 2 étant un moral au plus bas et un 12 étant un très bon moral. Avec des ajustements de situations, ce score pourra servir comme une indication sur les actions que prendront la partie dont le moral aura été vérifié.

\medskip

Expérience : une faible valeur devrait être attribuée aux objets magiques en ce qui concerne l'expérience, comme ces objets seront très utiles pour gagner de nouveaux trésors. Ainsi, dans la campagne Greyhawk, une flèche magique (+1) vaut au maximum 100 points d'expérience, une épée magique +1 sans caractéristiques spéciales vaut au maximum 1000 points, un rouleau de parchemin de sorts vaudra soit 500 points, soit 100 points par niveau et par sort (soit un sort du sixième niveau vaudra au maximum 600 points d'expérience), une potion vaut entre 250 et 500 points, et même un anneau avec un génie ne vaudra pas plus d'environ 5000 points. Les métaux de valeur et les pierres cependant sont récompensés en points d'expérience à hauteur du ratio 1 pièce d'or vaut 1 point d'expérience, ajusté par les circonstances -- comme expliqué dans D \& D, un guerrier du 10\textsuperscript{ème} niveau ne peut pas expédier une bande de kobolds et ne s'attendre à gagner qu'environ 1/10\textsuperscript{ème} d'expérience, à moins que le nombre de kobolds et les circonstances du combat soient telles qu'ils aient sérieusement mis en difficulté ledit guerrier en mettant véritablement sa vie en danger. Pour des raisons de détermination d'expérience, le niveau du monstre est équivalent à ses dés de vie ; et des compétences additionnelles s'ajoutent au niveau dans ce cas\footnote{
    Passage un peu alambiqué (NdT).
}. Une gorgone vaut certainement environ 10 facteur de niveaux, une tête de balrog moins de 12, le plus gros des dragons rouges, pas moins de 16 ou 17, et ainsi de suite. Le jugement de l'arbitre doit être utilisé pour statuer sur ce sujets, mais avec les exemples ci-dessus, cela ne devrait pas poser de difficultés.

\medskip

\uline{Sorts} : un magicien ne peut utiliser un sort qu'une seule fois durant une journée quelconque, même s'il transporte ses livres avec lui. Cela ne veut pas dire qu'il ne peut pas s'équiper d'une multiplicité du même sort, de sorte à pouvoir l'utiliser plus qu'une seule fois. Par conséquent, un magicien pourrait, par exemple, s'équiper de trois sorts de sommeil, chacun d'entre eux ne serait utilisable qu'une seule fois. Il pourrait aussi avoir un rouleau de parchemin de, disons, deux sorts, chacun d'entre eux étant aussi des sorts des sommeil. A mesure que les sorts seraient lus dans le parchemins, ils disparaîtraient, ce qui veut dire qu'au total, ce magicien aurait un maximum de cinq sorts de sommeil à utiliser le jour en question. S'ils n'avait pas de livres avec lui, il ne pourrait pas renouveler les sorts le jour suivant, comme le jeu suppose que le magicien gagne ses sorts via une préparation comme la mémorisation des incantations, et une fois que le sort a été prononcé, le schéma dans la mémoire a complètement disparu. De la même manière, les sorts sont inscrits sur un rouleau de parchemin et, dès que les mots sont proférés, ils s'effacent du parchemin.

