% A compiler avec xelatex
\documentclass[12pt]{article}

\usepackage[utf8]{inputenc}
\usepackage[french]{babel}

% Pour inclure la page de garde
\usepackage{pdfpages}

%geometry of the page
\usepackage[vmargin=0.6in,hmargin=1in]{geometry}

\setlength{\parindent}{1cm}

%========================================= FONTS
\usepackage{fontspec}

% Police de caractères OD&D Saloon Girl Inline
%\newcommand{\ODDtitlefont}{\fontsize{38}{40}\fontspec{QuentinCaps}\selectfont}
\newcommand{\ODDtitlefont}{\fontsize{60}{40}\fontspec{Saloon Girl Inline}\selectfont}

% Police de caractères OD&D OPTIChisel-Normal:style=Regular
\newcommand{\ODDtitlebisfont}{\fontsize{52}{70}\fontspec{OPTIChisel-Normal:style=Regular}\selectfont}
\newcommand{\ODDsectionfont}{\fontsize{42}{50}\fontspec{OPTIChisel-Normal:style=Regular}\selectfont}

\newcommand{\ODDtimes}{\fontspec{Times New Roman}\selectfont}

%- \usepackage{xcolor}
\usepackage{titlesec}
\defaultfontfeatures{Ligatures=TeX}
% Set sans serif font to Calibri
%- \setsansfont{Calibri}

% Set main font
\setmainfont{Futura Std}

% Define light and dark Microsoft blue colours
%- \definecolor{MSBlue}{rgb}{.204,.353,.541}
%- \definecolor{MSLightBlue}{rgb}{.31,.506,.741}
% Define a new fontfamily for the subsubsection font
% Don't use \fontspec directly to change the font
%- \newfontfamily\subsubsectionfont[Color=MSLightBlue]{Times New Roman}
% Set formats for each heading level
%\titleformat*{\section}{\normalfont\fontsize{12}\ttfamily{QuentinCaps}}

%\titleformat{\section}{\fontspec{QuentinCaps}\selectfont}{\thesection}{1em}{}

\titleformat{\section}{\centering\ODDsectionfont}{\thesection}{1em}{}
\titleformat{\subsection}{\large\bfseries}{\thesubsection}{1em}{}


%- \titleformat*{\subsection}{\large\bfseries\sffamily\color{MSLightBlue}}
%- \titleformat*{\subsubsection}{\itshape\subsubsectionfont}

% pour le (R)
\usepackage{fontspec}

% Pour les images
\usepackage{graphicx}
\graphicspath{{./yed/}{./images/}{./maps/}}

%Macro pour réduire l'espace sous le titre
\newcommand{\myunderline}[1]{\underline{\smash{#1}}}

%++++++++++++++++++++++++++++++++++++++++++++++++++++++++++++++++++++++++++
%                               DOCUMENT
%++++++++++++++++++++++++++++++++++++++++++++++++++++++++++++++++++++++++++
\begin{document}

\includepdf{pagedegarde.pdf}
\newpage

{\color{white}a}

\newpage

\thispagestyle{empty}
\begin{center}
{\Huge \ODDtitlefont{DONJONS \& DRAGONS}}{\normalsize \textsuperscript{\sffamily\textregistered}}

\vspace{1.8cm}

{\Large \textbf{Supplément III}}

\vspace{1.3cm}

{\Huge \ODDtitlebisfont{ELDRITCH}}

\vspace{0.3cm}

{\Huge \ODDtitlebisfont{WIZARDRY}}

\vspace{2.0cm}

{\Large \textbf{MAGIE ANCIENNE ET PUISSANTE}}

\vspace{0.8cm}

{\large PAR}

\vspace{0.1cm}

{\large GARY GYGAX \& BRIAN BLUME}

\vspace{3cm}

Remerciements spéciaux à l'aîné Steve Marsh, Dennis Sustare (le Super Druide),

Jim Ward \& Tim Kask pour leurs suggestions et contributions !

\vspace{0.8cm}

Illustrations de Dave Sutherland, Tracy Lesch \& Gary Kwapisz

Couverture par Deborah Larson

\vspace{1cm}

{\small \textbf{\ODDtimes{2023}}}

\vspace{0.5cm}

{\small \ODDtimes{\textbf{\textcopyright\ 1976 - TSR GAMES}

\textbf{9\textsuperscript{ème} impression, novembre 1979}

\textbf{Imprimé aux U.S.A}}}
\end{center}

\vfill

{\small \noindent LES QUESTIONS SUR LES REGLES DOIVENT ETRE ACCOMPAGNEES D'UNE ENVELOPPE TIMBREE A L'ADRESSE DE L'EXPEDITEUR ET ENVOYEES A TACTICAL STUDIES RULES, POB 756, LAKE GENEVA, WISCONSIN 53247}


\newpage
%====================================================
{\color{white}a}

\vfill

\noindent {\scriptsize \textit{Ce fascicule est une adaptation par Rouboudou (https://rouboudou.itch.io) de la partie pouvoirs psioniques du livret original. Cette adaptation est une œuvre de fan, gratuite et ne pouvant être vendue.}}

\newpage

\section*{Préface}

Le livre que vous tenez maintenant dans vos mains présente de nouvelles dimensions à un système de jeu déjà fascinant. Il s'agit du troisième supplément à DONJONS \& DRAGONS, produit comme conséquence à une demande toujours croissante de nouveau matériel.

Ce livre présente aussi une nouvelle mode dans l'art subtil d'être Maître de Donjon. Fidèle à sa conception d'origine, D\&D n'était limité dans son périmètre que par l'imagination et la dévotion des Maîtres de Donjons où qu'ils soient. Les suppléments ont répondu au besoin d'idées nouvelles et de mécanismes de simulation additionnels. Mais progressivement, D\&D a perdu un peu de sa saveur, et a commencé à devenir prévisible. Cela était dû à la prolifération d'ensembles de règles ; alors que c'était très bien pour nous en tant de compagnie, c'était compliqué pour le MD. Quand tous les joueurs avaient toutes les règles en face d'eux, il devenait presque qu'impossible de les séduire à affronter le danger ou les pièges.

Le nouveau concept innovant présenté dans ces pages devrait faire long feu en réintroduisant un peu de mystère, d'incertitude et de danger, qui refont de D\&D le défi sans équivalent qu'il a toujours été. La légende retrouve sa magie inestimable originale. On ne verra plus d'aventurier imprudent descendre dans un donjon , trouver quelque chose et savoir immédiatement ce que cela fait et comment cela fonctionne. De même, les joueurs ne pourront plus envoyer un de leurs infortunés serviteurs à une mort précoce en le forçant à expérimenter à sa place.

L'introduction du combat psionique est destiné à revivifier des parties devenues stagnantes. Il ouvre de nouvelles possibilités à la fois aux joueurs et au MD, tout en intégrant un des sujets favoris des auteurs de science-fiction et de fantastique : les pouvoirs inconnus de l'esprit.

Comme pour les deux précédents suppléments, le matériel contenu dans ce livret propose le même format que les trois livrets originaux de D\&D. Les corrections et les ajouts sont indiqués dans le texte de sorte qu'ils puissent être intégrés facilement dans les règles originales.

Comme vous pourrez le noter sur la page de titre, ce supplément est le fruit de plusieurs contributeurs. Telle est la nature de la chose que vous tenez entre vos mains. D\&D a été conçu pour être un jeu libre, lié aux règles de manière souple. Nous pensons que ELDRITCH WIZARDRY favorise les principes originaux de danger, d'excitation et d'incertitude. Que vous réussissiez toujours vos jets de sauvegarde.

\vspace{1cm}

\noindent Timothy J. Kask

\noindent TSR Publications Editor

\noindent Lake Geneva, Wisconsin

\noindent 23 avril 1976

\newpage

\section*{Introduction}

Le terme anglais \texttt{psionic} a été utilisé la première fois en 1951 dans une nouvelle de science-fiction écrite par Jack Williamson, \texttt{The Greatest Invention}, publié dans le magazine \texttt{Astounding Science Fiction}. Il est la compression de deux deux termes, \texttt{psi} dans le sens de phénomène psychique et \texttt{electronics}. \texttt{Psionics} devient un terme décrivant la discipline qui étudie les phénomènes psychiques avec les moyens de l'ingénierie moderne de l'époque, soit l'électronique. Malgré la promotion de personnes comme John W. Campbell, le terme restera utilisé uniquement dans le monde de la science-fiction, avant d'être intégré dans le monde des jeux de rôles.

La version originale de Donjons \& Dragons, dite OD\&D, est publié en 1974 sous la forme de trois livrets à la couverture marron. On trouve dans le second livret, \texttt{Monstres et Trésors}, traduit en français par Porphyre et disponible sur le site de \texttt{La Forge de Papier}(la-forge-de-papier.over-blog.com), on trouve les premières références à des pouvoirs psychiques, dans le chapitre sur les épées magiques. A l'époque, le terme \texttt{psionic} n'est pas utilisé.

Ensuite, en 1976, dans le deuxième supplément \texttt{Eldritch Wizardry} cosigné par Gary Gygax et Brian Blume, les pouvoirs psychiques arrivent dans le monde des PJs et des PNJs. Les règles sont présentée de manière assez chaotique, ce qui générera vite la réputation d'un système injouable dans le monde des joueurs. Derrière la juste critique sur la présentation, beaucoup de joueurs semblent avoir rejeté le supplément en raison du fait même de proposer en extension à un jeu médiéval-fantastique des règles plus proches de la science-fiction.

Nous proposons ici une nouvelle présentation des règles psioniques du supplément \texttt{Eldritch Wizardry}. Les règles présentées sont exhaustives, précisées et réorganisées. La consultation de certains sites en américain a été nécessaire pour s'assurer de la bonne compréhension de certaines règles ambiguës qui, encore aujourd'hui, provoquent des commentaires et des incompréhensions.

Une fois éclairci, le système se montre très intéressant, non seulement parce qu'il est très "gygaxien" (on le voit notamment au travers de l'utilisation de règles gigognes), mais aussi parce qu'il est le premier système complet de psioniques proposant une façon nouvelle de voir les pouvoirs, très différente de la magie de OD\&D, façon qui inspirera beaucoup d'autres systèmes de pouvoirs dont l'utilisation est basée sur la consommation de points.

\vspace{1cm}

\noindent Rouboudou

\noindent https://orey.github.io/blog

\noindent Juillet 2023

\newpage

\section*{Index}

\newpage

\section*{Hommes \& Magie}


\newpage

\section*{Monstres \& Trésors}

\subsection*{\underline{EPEES}}
{\parindent0pt

Parmi les armes magiques, seules les épées possèdent certains attributs humains (et surhumains) ; les épées ont un \myunderline{Alignement} (Loyal, Neutre ou Chaotique), un facteur d'\myunderline{Intelligence}, et un score d'\myunderline{Egoïsme} (ainsi qu'une détermination optionnelle de leur \myunderline{origine/objectif}). Ces déterminations sont faites de la manière suivante :

\bigskip

\textbf{Alignement} : lancez un dé de pourcentage pour déterminer l'Alignement.

\bigskip

{\parindent2.5cm
\begin{tabular}{p{2.8cm}l}
01 - 65
& L'épée est Loyale \\
66 - 90
& L'épée est Neutre \\
91 - 00
& L'épée est Chaotique \\
\end{tabular}}

\medskip

Notez que les pourcentages ci-dessus sont inversés pour l'épée ayant la capacité de drainer un niveau d'énergie vitale (83 sur la Table des Epées\footnote
{Se référer au second au manuel \texttt{Monsters \& Treasure} des règles originales de OD\&D. Le mécanisme est expliqué dans la description des \texttt{Wights} (esprits) : il s'agit essentiellement de perdre temporairement un niveau et un dé de vie (NdT).}). Si l'épée est Chaotique, elle affecte les créatures notées entre parenthèses (clercs, chevaux ailés, hippogriffes, rockhs, ents) au lieu de ceux indiqués en premier\footnote{Dans la Table des Epées.} (trolls et morts-vivants).

\medskip

Si un personnage se saisit d'une épée qui n'est pas du même Alignement que lui-même, il subit les dommages suivants :

\medskip

{\parindent2cm Loi -- Chaos : 2 Dés (2-12 points)

Neutralité - Loi/Chaos : 1 Dé (1--6 Points)}

\medskip

Si on ordonne à un PNJ de prendre une épée, les dommages ne seront que de la moitié de ceux indiqués ci-dessus, car la personne n'agit pas de manière libre. De plus, l'épée pourrait libérer celui qui l'a prise d'un sortilège, le faire changer d'Alignement, ou alors lui faire gagner des pouvoirs, ce qui les enlèverait au précédent maître de l'épée.

\medskip

De plus, si l‘Intelligence/Egoïsme de l'épée (voir ci-dessous) est supérieur de 6 points ou plus à celle du personnage qui l'a prise, l'épée contrôlera la personne, le faisant même prendre l'Alignement de l'épée, et agir en conséquence. Cela pourrait vouloir dire qu'un mercenaire d'un PJ Loyal à qui on ordonnerait de prendre une épée Neutre, et qui se ferait dominer par l'épée, mentirait délibérément à propos de ses pouvoirs ; si l'épée était Chaotique, il attaquerait.

\medskip

Après avoir déterminé l'Alignement, il faut définir l'Intelligence de l'épée.

\medskip

\myunderline{Intelligence} : l'Intelligence conditionne deux facteurs : les pouvoirs mentaux et la capacité à communiquer. Tous deux sont déterminés par un seul jet de dé.

\medskip

\begin{tabular}{c l c}
\textbf{Intelligence} & &\myunderline{\textbf{Capacité de}} \\
\textbf{\myunderline{(Jet de dé)}} & \myunderline{\textbf{Pouvoirs mentaux}} & \myunderline{\textbf{communication}} \\
1--6 & Aucun & Aucune* \\
7 & Un Pouvoir Primaire & Empathie \\
8 & Deux Pouvoirs Primaires & Empathie \\
9 & Trois Pouvoirs Primaires & Empathie \\
10 & Trois Pouvoirs Primaires et l'usage de langues** & Parole \\
11 & Comme 10 plus Lecture de la Magie & Parole \\
12 & Comme 11 plus une Capacité Extraordinaire & Télépathie \\
\end{tabular}

\medskip

* Même incapable de communiquer, l'épée confère au porteur ses pouvoirs, mais ceux-ci devront être découverts par l'utilisateur.

\medskip

** Le nombre de langues parlées \myunderline{en plus de la langue d'alignement de l'épée} est déterminé par un jet de dé.

\medskip

\begin{tabular}[t]{ll}

\begin{tabular}[t]{lp{7cm}}
\multicolumn{2}{l}{\myunderline{\textbf{Pouvoirs Primaires}}} \\
\textbf{\myunderline{Jet de dés}} & \myunderline{\textbf{Pouvoir}} \\
01--15 & Remarquer des parois \& salles coulissantes \\
16--30 & Détecter des passages inclinés \\
31--40 & Localiser les portes secrètes \\
41--50 & Détecter les pièges \\
51--60 & Voir les objets invisibles \\
61--70 & Détecter le mal et/ou l'or \\
71--80 & Détecter la nourriture et son type \\
81--90 & Détecter la magie \\
91--95 & Détecter les bijoux (nombre et taille) \\
96--99 & Faire deux jets en ignorant les scores au dessus de 95 excepté 00 \\
\hspace{0.4cm}00 & Faire un jet sur la table des Capacités Extraordinaires au lieu de celle-ci \\
\end{tabular}

\begin{tabular}[t]{lp{3cm}}
\multicolumn{2}{l}{\myunderline{\textbf{Langues parlées}}} \\
\textbf{\myunderline{Jet de dés}} & \myunderline{\textbf{Nb. langues}} \\
01--50 & Un \\
51--70 & Deux \\
71--85 & Trois \\
86--90 & Quatre \\
90--99 & Cinq \\
\hspace{0.4cm}00 & Faire deux jets en ignorant 00 s'il est tiré \\
\end{tabular}
\end{tabular}

\begin{center}
\includegraphics[scale=0.042]{./images/medusa.jpg}
\end{center}

\newpage

\begin{tabular}[t]{p{3cm}p{12cm}}
\multicolumn{2}{l}{\myunderline{\textbf{Table des Capacités Extraordinaires}}} \\
\textbf{\myunderline{Jet de dé}} & \myunderline{\textbf{Capacité}} \\
01--10 & Clairaudience \\
11--20 & Clairvoyance \\
21--30 & Perception extra-sensorielle (ESP) \\
31--40 & Télépathie \\
41--50 & Télékinésie \\
51--59 & Téléportation \\
60--68 & Vision rayons X \\
69--77 & Génération d'illusions \\
78--82 & Lévitation \\
83--87 & Vol \\
88--92 & Soins (1 point par 6 tours ou 6 points par jour) \\
93--97 & 1--4 fois la force normale pendant 1--10 tours, peut être employé une fois par jour \\
98-99 & Refaire deux jets en ignorant les jets au-dessus de 97 \\
\hspace{0.4cm}00 & Refaire trois jets en ignorant les jets au-dessus de 97 \\
\end{tabular}

\medskip

Tous les Pouvoirs Primaires et Capacités Extraordinaires sont transmis à l'utilisateur de l'épée. Si vous tirez la même capacité deux fois , cela signifie que la capacité est doublée en termes de force, de portée, de précision, et.

\bigskip

\myunderline{\textbf{Egoïsme}} : seules les épées avec une Intelligence de 7 ou plus ont un score d'Egoïsme. L'Egoïsme est compris dans l'intervalle 1--12, plus le nombre est grand et plus grand est l'Egoïsme de l'épée. L'Egoïsme de l'épée peut lui faire faire les choses suivantes :

\begin{enumerate}
\item Obliger l'utilisateur à se défaire de meilleures armes,
\item Mettre l'utilisateur dans des situations très dangereuses afin d'exalter son rôle dans le combat,
\item Se laisser capturer par un personnage ou une créature de niveau plus élevé qui est plus proche de l'esprit de l'épée,
\item Se laisser capturer par un personnage ou une créature de niveau plus faible dans le but d'exercer un plus grand contrôle sur son utilisateur, et
\item Exiger qu'une partie des trésors conquis lui soit consacré, sous la forme de meilleurs fourreaux, d'incrustations de joyaux, ou de dispositifs magiques pour la garder lorsque personne ne l'utilise.
\end{enumerate}

A chaque moment où une situation se présente durant laquelle une des possibilités listées ci-dessus existe, l'Egoïsme de l'épée intervient. Ce dernier influencera toujours sa relation à l'utilisateur, bien que de vrais rapports puissent voir le jour si l'alignement et les buts du personnage/utilisateur coïncident avec l'\myunderline{origine/objectife} de l'épée. La détermination de chaque facteur est décrite ci-après:

\begin{center}
\begin{minipage}{0.8\linewidth}
\textbf{Influence de l'Egoïsme dans des Situations Clef} : l'arbitre additionne l'Intelligence et l'Egoïsme de l'épée (entre 8 et 24) et ajoute 1 pour chaque Capacité Extraordinaire (entre 1 et 4 si applicable). Le total (dans l'intervalle 8--28) est comparé à la somme de l'Intelligence et de la Force du personnage (6--36) modifié par une variable basée sur l'état physique de l'utilisateur. Si le personnage est frais et relativement exempt de dommages (moins de 10\% de dommages), il faut \myunderline{ajouter} 1--6 points à son total (pour arriver à un intervalle de 7--42). S'il est mentalement ou physiquement fatigué, ou s'il a subi des dommages compris entre 10\% et 50\%, il faut \myunderline{déduire} 1--4 points de son total (pour arriver à un intervalle de 2--35). Si les dommages subis dépassent les 50\%, ou le personnage a été sous le coup d'une tension mentale sévère venant d'une forme de magie, il faut \myunderline{déduire} 2--8 points (pour arriver à un intervalle de 0--34).

\bigskip

{\parindent0.5cm \begin{tabular}{ll}
\textbf{\myunderline{Différence}} & \myunderline{\textbf{Résultat}} \\
6 ou plus & Le plus haut score l'emporte \\
2--5 & 75\% de chances que le plus haut score l'emporte \\
0--1 & 50\% de chances pour chacun \\
\end{tabular}}

\bigskip

\textbf{L'Egoïsme dans les relations au long cours avec l'utilisateur} : cette détermination est assez simple, car elle est basée sur la comparaison du score d'Egoïsme de l'épée (1--12) avec le niveau du Combattant l'utilisant. Consultez la table des \myunderline{Situations Clef} ci-dessus. Si l'une des parties a une différence positive de 6 ou plus, celle-ci l'emportera toujours et plus aucun test (incluant les Situations Clef) ne sera nécessaire. Une différence positive de 2--5 indiquera que celle ayant le plus haut score l'emporte généralement, et les tests ne devront être effectués que dans les Situations Clef. Une différence de 0--1 indique un combat continu entre l'épée et son utilisateur, et durant les situations stressantes, les deux devraient être testés afin de déterminer qui l'emporte.

\end{minipage}
\end{center}

\myunderline{\textbf{Origine/Objectif}} : naturellement, l'origine de chaque épée est la Loi, la Neutralité ou le Chaos, mais certaines de ces armes sont forgés par des forces plus puissantes pour un objectif particulier. Pour déterminer si une épée possède un tel objectif, faire un jet de pourcentage et un jet de 91 ou plus indique que l'épée possède une mission spéciale. Les épées avec des objectifs spéciaux voient automatiquement leur score l'Intelligence et d'Egoïsme poussés au maximum et elles gagnent une capacité additionnelle :

\bigskip

{\parindent1cm \textbf{Loi}: capacité de paralyser les opposants chaotiques,

\textbf{Neutralité} : ajoute +1 à tous les jets de sauvegarde,

\textbf{Chaos} : capacité de désintégrer les opposants loyaux.}

\bigskip

La capacité spéciale ne sera applicable que pour ceux que l'épée a été chargée de détruire, ou ceux qui les servent.

\myunderline{\textbf{Objectifs}}:

\medskip

{\parindent1.5cm\begin{tabular}{p{4.6cm}p{4.6cm}p{4.6cm}}
Tuer les magiciens & Tuer les combattants & Vaincre la Loi \\
Tuer les clercs & Tuer les monstres & Vaincre le Chaos \\
\end{tabular}}

\bigskip

Ainsi, une épée au service de la Loi ayant pour but de tuer les magiciens (chaotiques) les paralysera ainsi que leurs sbires, mais elle n'utilisera pas ses pouvoirs de paralysie contre un géant errant. Néanmoins, les épées ayant un objectif large utiliseraient leurs pouvoirs pour vaincre tous les opposants de nature Loyale ou Chaotique. Les épées ayant un objectif spécial de Neutralité agiront contre la Loi et le Chaos de la même manière. Les épées ayant un objectif spécial chercheront toujours à l'accomplir, et chaque tentative des utilisateurs de le contrer se soldera immédiatement par un test d'influence.

\bigskip

\myunderline{\textbf{EPEES, BONUS AUX DOMMAGES}} : Les épées reçoivent toutes des bonus pour ce qui est de la probabilité de toucher un opposant, mais certaine sd'entre elles reçoivent aussi un bonus aux dommages quand elles touchent. Ces épées sont celles qui ont un +2 ou +3 contre certaines créatures, mais pas celles qui ont un bonus général de +2 ou +3.
}% parindent

\begin{verbatim}
---
tags:
    - Ambre
    - D&D
    - Zelazni
---

# Vocabulaire

Alors que l'on dit plutôt en français, "pouvoirs psychiques", le terme américain "psionic" est entré progressivement dans le vocabulaire des jeux de rôles, souvent en étant francisé en "psionique". Dans cet article, nous nous limiterons à l'usage du terme "psychique" en français, le terme "psionic" étant réservé à la citation d'ouvrages en anglais.

# 1976 - OD&D au début de tout (comme souvent)

## Introduction

![Alt text](../images/eldritch-wizardry.png)

En 1976 sort le livre VI de *Dungeons & Dragons* avec le titre *Eldritch Wizardry* cosigné de [Gary Gygax](https://github.com/orey/DandD/tree/master/GaryGygax) et de Brain Blume. Ce livre d'une soixantaine de pages contient essentiellement des suppléments sur les pouvoirs psychiques, sur le druide en tant que classe de personnage et sur les démons.

Ledit supplément est disponible sur [archive.org](https://archive.org/details/tsr02005supplement3eldritchwizardy). Il n'a pas été traduit en français à ma connaissance alors que les règles de OD&D l'ont été : voir sur le très bon site [La Forge de papier](http://la-forge-de-papier.over-blog.com/2017/09/donjons-dragons-la-boite-blanche-elle-aussi-en-francais.html).

Il semble que ce soit la première apparition des pouvoirs psychiques dans le monde du JDR. Leur apparition au sein de *D&D* a quelque chose d'étrange, car on pourrait y voir une collision de diverses influences dans la fantasy: le médiéval-fantastique qui favorise la magie et les pouvoirs psychiques qui sent plus le JDR contemporain ou futuriste.

D&D propose les pouvoirs psy comme un genre de supplément optionnel pour "pimenter" les parties. L'introduction de 1976 de [Tim Kask](https://en.wikipedia.org/wiki/Tim_Kask) dudit livre est amusante :

*L'introduction du combat psionique s'attache à redynamiser des jeux devenus ternes. Cela ouvre des possibilités inédites à la fois pour les joueurs et le MD et, ce faisant, cela permet de se reconnecter à l'un des thèmes favoris des écrivains de science-fiction et de fantasy : les pouvoirs inconnus de l'esprit.*

Pour autant, en lisant entre les lignes, l'introduction des démons dans le jeu semble pousser Gygax à créer les règles des pouvoirs psychiques, mieux adaptés, selon lui, à représenter les pouvoirs de certains démons.

Comme souvent dans OD&D, les règles sont confuses, incomplètes, éclatées dans le tout le livret, entrecoupées de choses qui n'ont rien à voir (le Druide, un système de combat alternatif basé sur la **Dextérité Ajustée**, etc.). Au niveau édition, *Eldritch Wizardry* est un des pires exemples de l'époque.

## Obtention des Pouvoirs Psychiques (PoP)

A la création du PJ, un score de 15 ou plus dans INTelligence, WISdom ou CHArisma donne la possibilité de tester si le personnage dispose d'un pouvoir psy (si le personnage est humain). Il est nécessaire de faire 91 ou plus sur 1d100 pour que ce soit le cas. Les moines et les druides ne peuvent pas avoir de pouvoirs psychiques.

Si le personnage est éligible, il faut faire un nouveau jet de 1d100 pour déterminer le **potentiel psychique** (PP, purement aléatoire). Suivant le PP, le PJ obtient un malus ou un bonus pour avoir des **pouvoirs psychiques** (PoP, de -6% à +3%).

| PP (1d100) | Bonus/Malus au jet d'obtention de PoP (BMPoP) |
|------------|-----------------------------------------------|
| 01–10      | -6% par niveau (cumulatif)                    |
| 11–25      | -5% par niveau (cumulatif)                    |
| 26–50      | -4% par niveau (cumulatif)                    |
| 51–75      | Aucun                                         |
| 76–90      | +1% par niveau (cumulatif)                    |
| 91–99      | +2% par niveau (cumulatif)                    |
| 00         | +3% par niveau (cumulatif)                    |

La chance de base pour avoir un PoP est de 10% + BMPoP par niveau. Ainsi, un perso du 4ème niveau avec un PP de 90 aura une base de 10 + 1 = 11 soit 11 x 4 = 44% de chances d'avoir un PoP. Cela veut dire aussi qu'un personnage de niveau 10 aura 100% d'avoir un PoP.

Notons enfin que donc, ce jet se produit à chaque changement de niveau.

Cerise sur le gâteau : si le PJ obtient un PoP, alors s'il réussit un jet de 1d100 sous son PP, il en a automatiquement un second !

Nous sommes en plein dans le monde Gygaxien des **règles gigognes** :

* R1 : INT ou WIS ou CHA > 15,
* R2 : Si R1 OK, jet de 1d100 pour PP et BMPoP,
* R3 : Quand R2 OK, jet de BMPoP pour obtenir un PoP,
* R4 : Si R3 OK, si succès, alors jet de PP pour obtenir un second PoP.

Nous verrons apparaître ce genre de règles gigognes dans les autres éditions de D&D signées de Gygax.

## Pouvoirs psychiques (PoP)

### Liste des pouvoirs

La liste originale est présentée ci-dessous.

![Listedescompetencespsioniques](../images/od&d-psionics.png)

La liste des PoP est fournie ci-dessous, retriée par type (interprétation personnelle) et par classe de personnages.

| Pouvoir Psychique (PoP)           | Type | Guerriers & Voleurs | Magiciens  | Clercs     |
|-----------------------------------|------|---------------------|------------|------------|
| Réduction                         | A    | 1 Oui (B)           | 1 Oui (B)  |            |
| Expansion                         | A    | 2 Oui (B)           | 2 Oui (B)  |            |
| Lévitation                        | A    | 3 Oui (B)           | 3 Oui (B)  | 1 Oui (B)  |
| Changer le poids du corps         | A    | 4 Oui (B)           |            | 2 Oui (B)  |
| Corps comme arme                  | A    | 5 Oui (B)           |            |            |
| Réarrangement moléculaire         | A    | 6 Oui (S)           |            | 3 Oui (S)  |
| Manipulation moléculaire          | A    | 7 Oui (S)           |            |            |
| Contrôle du corps                 | A    | 8 Oui (S)           |            |            |
| Barrière mentale                  | A    | 9 Oui (S)           |            |            |
| Agitation moléculaire             | A    |                     | 4 Oui (B)  |            |
| Altération de la forme            | A    |                     | 5 Oui (S)  |            |
| Contrôle cellulaire               | A    |                     |            | 4 Oui (B)  |
| Contrôle de l'esprit sur le corps | A    | 10 Oui (B)          |            | 5 Oui (B)  |
| Invisibilité                      | A    | 11 Oui (B)          |            |            |
| Hibernation                       | A    | 12 Oui (B)          |            |            |
| Télékinésie                       | A    | 13 Oui (S)          | 6 Oui (S)  |            |
| Contrôle de l'énergie             | A    | 14 Oui (S)          |            |            |
| Domination                        | B    | 15 Oui (B)          |            | 6 Oui (B)  |
| Hypnose                           | B    |                     | 7 Oui (B)  | 7 Oui (B)  |
| Projection télépathique           | B    |                     | 8 Oui (S)  | 8 Oui (S)  |
| Altération de l'aura              | B    |                     |            | 9 Oui (S)  |
| Domination des masses             | B    |                     |            | 10 Oui (S) |
| ESP                               | B    |                     | 9 Oui (B)  | 11 Oui (B) |
| Empathie                          | B    |                     |            | 12 Oui (S) |
| Télépathie animale                | B    |                     |            | 13 Oui (B) |
| Prémonition                       | C    | 16 Oui (B)          | 10 Oui (S) | 14 Oui (S) |
| Clairaudience                     | C    | 17 Oui (B)          | 11 Oui (B) |            |
| Clairvoyance                      | C    | 18 Oui (B)          | 12 Oui (B) |            |
| Détection du mal/du bien          | C    |                     | 13 Oui (B) | 15 Oui (B) |
| Détection de la magie             | C    |                     | 14 Oui (B) |            |
| Marche dimensionnelle             | D    | 19 Oui (S)          |            | 16 Oui (S) |
| Projection astrale                | D    | 20 Oui (S)          | 15 Oui (S) | 17 Oui (S) |
| Porte dimensionnelle              | D    |                     | 16 Oui (S) |            |
| Téléportation                     | D    |                     | 17 Oui (S) |            |
| Substance éthérée                 | D    |                     | 18 Oui (S) |            |
| Voyage probabiliste               | D    |                     |            | 18 Oui (S) |

Nous pouvons classer ces pouvoirs en plusieurs catégories (ce n'est pas dans le livre original) :

* A : Contrôle de la matière au niveau moléculaire, voire atomique,
* B : Contrôle via l'esprit de l'esprit,
* C : Sens,
* D : Voyage psychique.

Les pouvoirs sont présents chez les types de personnages soient au niveau Basique (B) soit au niveau Supérieur (S). Le fait qu'un PoP soit Supérieur ne se traduit (dans les règles) que par le fait qu'elle soit détectable par un autre personnage ayant des pouvoirs psy au double de la portée du pouvoir (voir plus bas détection des pouvoirs psychiques).

Les PJ ne peuvent pas avoir plus de PoP Supérieurs que de PoP Basiques.

Cette distinction est peu développée et on imagine que le but était d'avoir la possibilité de deux puissances différentes dans les mêmes pouvoirs. Par contre, les pouvoirs eux-mêmes n'utilisent pas cette distinction.

La détermination des PoPs est aléatoire mais il est suggéré de restreindre le choix à des types de PoP cohérents.

### A - Contrôle de la matière et de l'énergie

On retrouve dans cette dimension pas mal de pouvoirs :

* Réduction ;
* Expansion ;
* Lévitation, qui est un moyen de changer le poids du corps ;
* Changer le poids du corps (*Body Equilibrium*), notamment pour marcher sur l'eau ;
* Corps comme arme (*Body Weaponry*) ;
* Réarrangement moléculaire, qui transforme les métaux ;
* Manipulation moléculaire, qui permet de rendre fragile différentes matières dures ;
* Contrôle du corps, qui permet d'adapter le corps à des conditions extrêmes comme le froid extrême, le feu, les fumées empoisonnées, etc. ;
* Barrière mentale ;
* Agitation moléculaire, agit un peu comme un four à micro-ondes ;
* Altération de la forme, pour se changer en quelqu'un d'autre ;
* Contrôle de l'esprit sur le corps, pour se passer de manger, de boire et de dormir ;
* Invisibilité ;
* Hibernation (*Suspend Animation*), ou comment suspendre ses fonctions vitales ;
* Télékinésie, contrôle du mouvement des objets par la pensée ;
* Contrôle de l'énergie, ou comment dissiper des énergies agressives ;
* Contrôle cellulaire (*Cellular Adjustment*), qui permet de soigner les blessures et les maladies.

### B - Contrôle via l'esprit de l'esprit

On retrouve les pouvoirs suivants :

* Domination, pour forcer une personne à faire ce que le PJ souhaite ;
* Hypnose, contrôle des esprits faibles ;
* Projection télépathique, envoyer des messages ou influence une personne ;
* Altération de l'aura ;
* Domination des masses ;
* ESP ;
* Empathie ;
* Télépathie animale.

### C - Sens

On retrouve les pouvoirs suivants :

* Prémonition ;
* Clairaudience ;
* Clairvoyance ;
* Détection du mal/du bien ;
* Détection de la magie.

### D - Voyage psychique

* Marche dimensionnelle ;
* Projection astrale ;
* Porte dimensionnelle
* Téléportation
* Substance éthérée ;
* Voyage probabiliste, une projection astrale avec le corps permettant de passer à travers les plans et les mondes parallèles (comme dans la *Saga des Princes d'Ambres* de Zelazni).

## Modes d'Attaque et Modes de Défense

Une fois que le PJ a son premier PoP, il gagne son premier **Mode d'Attaque** psychique (MA) : *Explosion psionique* (voir tableau ci-dessous).

Les autres MA sont gagnés tous les 4 PoP (5 pour les Guerriers).

Cette dernière consigne est un peu vague. Elle sous-entend que, lors du passage de niveau, le PJ ayant des pouvoirs psychiques va refaire un test pour avoir un nouveau PoP (sachant qu'il peut en obtenir au maximum deux par niveau). Il gagnera un mode d'attaque alors en fonction du nombre de PoP dont il dispose.

Concernant les **Modes de Défense** (MD), encore une fois les règles ne sont pas claires. Il est dit qu'ils sont acquis à raison de un tous les 3 PoP acquis (4 pour les Guerriers).

Comme les PJs ayant des pouvoirs psychiques sont capables de combattre psychiquement, il faut supposer que les MDs marchent comme pour les MAs et que le PJ gagne son premier MD avec son premier PoP, puis ensuite tous les 3 (ou 4) PoP acquis.

Une autre interprétation possible serait qu'il faut vraiment avoir 3 ou 4 PoP pour gagner son premier MD. Je trouve cette interprétation un peu bizarre au regard de la suite qui présuppose, dans le combat psionique, que l'attaquant psy a un MA et que le défenseur a un MD. Un défenseur ayant des pouvoirs psychiques sans MD ouvre une brèche dans le système.

Il n'est pas expliqué comment les MA/MD sont attribués : aléatoirement (D6 en rejouant le 6) ou séquentiellement, en utilisant les lettres, ou par simple choix. Comme la puissance de ces attaques est diverse et non linéaire, je pense pas que les lettres soient très importantes pour les MA. Pour les MD, il y a une certaine progression. Au DM de choisir son mode d'attribution.

La liste des modes d'attaque et de défense est présentée ci-dessous avec leur coût en **Force Psionique** entre parenthèses.

| Modes d'Attaque (MA), toutes classes     | Modes de Défense (MD), toutes classes |
|------------------------------------------|---------------------------------------|
| A. Explosion psionique (20)[x3]          | F. Esprit vide (1)                    |
| B. Poussée de l'esprit (10)[x1]          | G. Bouclier de pensée (2)             |
| C. Coup de fouet sur l'ego (15)[x3]      | H. Barrières mentales (4)             |
| D. Imposition d'identité (10)[x1]        | I. Forteresse intellectuelle (7)      |
| E. Ecrasement psychique (25&spades;)[x3] | J. Tour de volonté de fer (10)        |

&spades; Si le PJ possède moins de 25 points de FP, il est demandé au MJ de "modifier la probabilité de succès de manière *ad hoc*".

L'utilisation des MAs peut être détectée par un PJ ou PNJ ayant des pouvoirs psychiques. Le facteur entre crochets donne le multiplicateur de portée du MA dans lequel le PJ ou PNJ peut détecter son usage. Pour mémoire, les pouvoirs

## Force Psionique (FP), Force Psionique d'Attaque (FPA) et de Défense (FPD)

Les formules sont les suivantes :

* FPA = PP + 2 x nombre(PoP) + 5 x nombre(MA) + 5 x nombre(MD)
* FPD = FPA
* FP = FPA + FPD = 2 x FPA

Ces scores serviront de réservoirs de points pour les attaques et les défenses psychiques en combat psychique, mais aussi pour les divers PoP que les PJs auront accumulés et dont le coût est expliqué dans les descriptions.

A chaque fois que de la FP est consommée, il faut répartir la consommation à 50%-50% entre la FPA et la FPD, cela pour toutes les consommations. En gros, la FP est un réservoir de points divisés en deux avec une consommation égale dans les deux sous-réservoirs (cela vaut aussi pour la consommation des points en combat, soit avec les MA et MD).

La FP se récupère assez vite en cessant toute activité psychique et en suivant les indications de la table suivante.

| Activité                            | Gain en FP          |
|-------------------------------------|---------------------|
| Marcher, parler et autres activités | 6 points par heure  |
| Se reposer tranquillement           | 12 points par heure |
| Dormir                              | 24 points par heure |

## Détection des PoP et des MA

La table ci-dessous présente les règles de détection des PoP et MA (nous avons utilisé les lettres pour désigner les MA) de PJs ou PNJs ayant des pouvoirs psychiques et détectant dans leur voisinage des créatures utilisant des pouvoirs.

| Type                     | Niveau/Nom | Portée de la détection | % de base | Cumulatif par tour |
|--------------------------|------------|------------------------|-----------|--------------------|
| PoP                      | Basique    | Portée du pouvoir      | 10%       | Oui (+10%)         |
| PoP                      | Supérieur  | 2 x Portée du pouvoir  | 10%       | Oui (+10%)         |
| MA                       | A, C, E    | 3 x Portée du pouvoir  | 10%       | Oui (+10%)         |
| MA                       | B, D       | Portée du pouvoir      | 10%       | Oui (+10%)         |
| Objet psy                | -          | Portée du pouvoir      | 10%       | Oui (+10%)         |
| Sort (idem psy)          | -          | Portée du sort         | 10%       | Oui (+10%)         |
| Objet magique (idem psy) | -          | Portée du sort         | 10%       | Oui (+10%)         |

Si le pouvoir est utilisé en continu, alors les chances de détection dans le cadre de la portée de détection grandissent de 10% par tour.

La créature qui détecte ne pourra pas détecter un pouvoir qu'elle n'a pas, mais pourra, au premier tour détecter dans quelle direction le pouvoir est utilisé, et au second tour avec quelle puissance "relative" ce pouvoir est utilisé.

Les sorts qui sont similaires aux pouvoirs psychiques seront détectés de la même façon que lesdits pouvoirs. Les objets magiques qui ont les mêmes fonctions que les objets psychiques seront, eux-aussi, détectés de la même façon.

## Le PJ psionique selon OD&D

Le PJ psionique a donc les caractéristiques suivantes :

* PP avec BMPoP,
* Liste des PoPs,
* FP = FPA + FPD,
* Listes des MAs et MDs.

## Combat psychique




*En cours*

## Liens



<div class="mydate">03-20 juin 2023</div>


\end{verbatim}




\end{document}
