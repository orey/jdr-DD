%+=+=+=+=+=+=+=+=+=+=+=+=+=+=+=+=+=+=+=+=+=+=+=+=+=+=+=+=+=+=+=+=+=+=+=+= PART
%+=+=+=+=+=+=+=+=+=+=+=+=+=+=+=+=+=+=+=+=+=+=+=+=+=+=+=+=+=+=+=+=+=+=+=+= PART
%+=+=+=+=+=+=+=+=+=+=+=+=+=+=+=+=+=+=+=+=+=+=+=+=+=+=+=+=+=+=+=+=+=+=+=+= PART
%+=+=+=+=+=+=+=+=+=+=+=+=+=+=+=+=+=+=+=+=+=+=+=+=+=+=+=+=+=+=+=+=+=+=+=+= PART
%+=+=+=+=+=+=+=+=+=+=+=+=+=+=+=+=+=+=+=+=+=+=+=+=+=+=+=+=+=+=+=+=+=+=+=+= PART
\newpage
\phantomsection\addcontentsline{toc}{section}{D\&D VOLUME III -- AVENTURES DANS LE MONDE SOUTERRAIN \& DANS LES REGIONS SAUVAGES}\begin{center}
{\Huge \ODDtitlefont{DONJONS \& DRAGONS}}{\normalsize \textsuperscript{\sffamily\textregistered}}

\vspace{1.8cm}

{\Large \textbf{Volume III}}

\vspace{1.3cm}

{\Huge \ODDtitlebisfont{AVENTURES}}

\vspace{0.3cm}

{\Huge \ODDtitlebisfont{DANS LE MONDE}}

\vspace{0.3cm}

{\Huge \ODDtitlebisfont{SOUTERRAIN}}

\vspace{0.3cm}

{\Huge \ODDtitlebisfont{ \& DANS LES}}

\vspace{0.3cm}

{\Huge \ODDtitlebisfont{REGIONS}}

\vspace{0.3cm}

{\Huge \ODDtitlebisfont{SAUVAGES}}

\vspace{3cm}

{\large PAR

\vspace{0.1cm}

GARY GYGAX \& DAVE ARNESON}
\end{center}

\newpage
%======================Blank page
\phantom{-}
\newpage

%==========================================================================SECTION
\newpage
\phantomsection\section*{Aventures dans le monde souterrain \& dans les régions sauvages}
\addcontentsline{toc}{section}{Aventures dans le monde souterrain \& dans les régions sauvages}

\begin{center}
\textbf{[SELECTION]}
\end{center}

%----------------------------------------------------- SUB SECTION
\phantomsection\subsection*{LES MONSTRES DU MONDE SOUTERRAIN}
\addcontentsline{toc}{subsection}{LES MONSTRES DU MONDE SOUTERRAIN}

\pdfbookmark[3]{Surprise}{pdf-action-surprise}\phantomsection\label{dd3-surprise}\textbf{\uline{Surprise}} : une situation de surprise ne peut exister que dans la mesure où l'une ou les deux parties ne sont pas au courant de la présence de l'autre. Des choses comme la perception extrasensorielle, la lumière ou le bruit rendent impossible la surprise. Si la possibilité d'une surprise existe, jeter un dé à six faces pour chaque partie concernée. Un jet de 1 ou 2 indique que le groupe est surpris. La distance est alors de 3--9m.

\bigskip

La surprise donne l'avantage d'un segment de mouvement libre, utilisable pour fuir, jeter un sort ou s'engager dans le combat. Si les monstres gagnent l'effet de surprise, soit ils se rapprocheront des personnages (à moins qu'ils soient intelligents et que leur proie est manifestement trop forte pour qu'ils l'attaquent), soit ils attaqueront.

\bigskip

Par exemple, une vouivre surprend un groupe de quatre personnages quand ils débouchent dans un endroit vaste et ouvert après avoir pris un coude. Elle attaque dans sa distance de combat, comme indiqué par la détermination de la distance de surprise, le résultat du jet étant de 2, la distance entre eux est de 3m. L'arbitre lance une paire de dés à six faces pour la vouivre et obtient un score de 6, ce qui implique qu'elle ne piquera pas. Elle mord et frappe. La vouivre peut attaquer une fois encore avant que les aventuriers ne frappent à leur tour.

\bigskip

\pdfbookmark[3]{Actions aléatoires par monstre}{pdf-action-monstres}\phantomsection\label{dd3-actions-monstres}\textbf{Actions aléatoires par monstre} : dans les autres situations qu'en poursuite, les plus intelligents des monstres agiront de manière aléatoire, en accord avec les résultats obtenus en faisant un jet de deux dés (à six faces) :

\bigskip

{\parindent6cm \begin{tabular}{cl}
2--5 & réaction négative \\
6--8 & réaction incertaine \\
9--12 & réaction positive \\
\end{tabular}}

\medskip

Le score aux dés doit être modifié par des additions ou des soustractions pour des choses telles que des pots-de-vin offerts, la peur, l'alignement des parties concernées, etc.

