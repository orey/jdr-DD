%+=+=+=+=+=+=+=+=+=+=+=+=+=+=+=+=+=+=+=+=+=+=+=+=+=+=+=+=+=+=+=+=+=+=+=+= PART
%+=+=+=+=+=+=+=+=+=+=+=+=+=+=+=+=+=+=+=+=+=+=+=+=+=+=+=+=+=+=+=+=+=+=+=+= PART
%+=+=+=+=+=+=+=+=+=+=+=+=+=+=+=+=+=+=+=+=+=+=+=+=+=+=+=+=+=+=+=+=+=+=+=+= PART
%+=+=+=+=+=+=+=+=+=+=+=+=+=+=+=+=+=+=+=+=+=+=+=+=+=+=+=+=+=+=+=+=+=+=+=+= PART
%+=+=+=+=+=+=+=+=+=+=+=+=+=+=+=+=+=+=+=+=+=+=+=+=+=+=+=+=+=+=+=+=+=+=+=+= PART
\newpage
\phantomsection\addcontentsline{toc}{section}{D\&D VOLUME I -- HOMMES \& MAGIE}\begin{center}
{\Huge \ODDtitlefont{DONJONS \& DRAGONS}}{\normalsize \textsuperscript{\sffamily\textregistered}}

\vspace{1.8cm}

{\Large \textbf{Volume I}}

\vspace{1.3cm}

{\Huge \ODDtitlebisfont{HOMMES \&} MAGIE}

\vspace{5cm}

{\large PAR

\vspace{0.1cm}

GARY GYGAX \& DAVE ARNESON}
\end{center}

\newpage

%==========================================================================SECTION
\phantomsection\section*{Hommes \& Magie}
\addcontentsline{toc}{section}{Hommes \& Magie}

\begin{center}
\textbf{[SELECTION]}
\end{center}

%----------------------------------------------------- SUB SECTION
\phantomsection\subsection*{SYSTEME DE COMBAT ALTERNATIF}
\addcontentsline{toc}{subsection}{SYSTEME DE COMBAT ALTERNATIF}

Ce système est basé sur les capacités offensives et défensives des combattants ; des choses comme la vitesse, la férocité ou les armes employées par les monstres attaquant sont incluses dans les matrices. Deux tableaux sont proposés : un pour les hommes contre les hommes ou les monstres, et un pour les monstres (incluant les kobolds, les gobelins, les orcs, et.) contre les hommes.

\bigskip

\textbf{MATRICE D'ATTAQUE 1 : HOMMES ATTAQUANT}

\bigskip

\begin{tabular}{cccccccccccc}
\textbf{Niveau du } & \textbf{Niveau du } & \textbf{Niveau du } & \multicolumn{8}{c}{\textbf{Classe d'armure du défenseur}} \\
\textbf{guerrier}   & \textbf{magicien}   & \textbf{clerc}   & \textbf{9} & \textbf{8} & \textbf{7} & \textbf{6} & \textbf{5} & \textbf{4} & \textbf{3} & \textbf{2} \\
\textbf{1--3}   & \textbf{1--5}   & \textbf{1--4}   & 10 & 11 & 12 & 13 & 14 & 15 & 16 & 17 \\
\textbf{4--6}   & \textbf{6--10}  & \textbf{5--8}   &  8 &  9 & 10 & 11 & 12 & 13 & 14 & 15 \\
\textbf{7--9}   & \textbf{11--15} & \textbf{9--12}  &  5 &  6 &  7 &  8 &  9 & 10 & 11 & 12 \\
\textbf{10--12} & \textbf{16--20} & \textbf{13--16} &  3 &  4 &  5 &  6 &  7 &  8 &  9 & 10 \\
\textbf{13--15} & \textbf{21--25} & \textbf{17--20} &  1 &  2 &  3 &  4 &  5 &  6 &  7 &  8 \\
\textbf{16\&+}  & \textbf{26\&+}  & \textbf{21\&+}  &  1 &  1 &  1 &  1 &  2 &  3 &  4 &  5 \\
\end{tabular}

\bigskip

Les hommes normaux équivalent à des guerriers de niveau 1.

\bigskip

Toutes les attaques font 1d6 points de dommages sauf indication contraire.

%----------------------------------------------------- SUB SECTION
\phantomsection\subsection*{EXPLICATION DES SORTS}
\addcontentsline{toc}{subsection}{EXPLICATION DES SORTS}

%- - - - - - - - - - - - - - - - - - - - - - - - - - - SUB SUB SECTION
\phantomsection\subsubsection*{Magiciens}
\addcontentsline{toc}{subsubsection}{Magiciens}

\textbf{Deuxième niveau :}

\bigskip

\label{sort-esp}\textbf{Perception extrasensorielle} (ESP) : un sort qui permet à l'utilisateur de détecter les pensées (s'il y en a) de tout ce qui rode derrière les portes ou dans les ténèbres. Le sort peut pénétrer de la roche jusqu'à environ 60cm d'épaisseur mais une fine couche de plomb empêchera sa pénétration. Durée : 12 tours. Portée : 2m\footnote{Extrait du premier livret de OD\&D, \texttt{Men and Magic} (NdT).}.

\bigskip

\textbf{Troisième niveau :}

\bigskip

\label{sort-clairvoyance}\textbf{Clairvoyance} : comme le sort Perception extrasensorielle excepté que le lanceur du sort peut visualiser au lieu de détecter des pensées\footnote{Extrait du premier livret de OD\&D, \texttt{Men and Magic} (NdT).}.

\bigskip

\label{sort-clairaudience}\textbf{Clairaudience} : comme le sort de Clairvoyance excepté qu'il permet d'entendre plutôt que de voir. C'est un des rares sorts qui peut être lancé au travers d'une boule de cristal\footnote{Extrait du premier livret de OD\&D, \texttt{Men and Magic} (NdT).}.

\bigskip

\textbf{Quatrième niveau :}

\bigskip

\label{sort-metamorphose}\textbf{Métamorphose} : Un sort permettant à l'utilisateur de prendre la forme de \underline{tout} ce qu'il souhaite, mais il ne pourra pas acquérir les compétences de combat de la chose dans laquelle il s'est métamorphosé. C'est-à-dire, si l'utilisateur se transformait dans un dragon d'un type particulier, il ne gagnerait pas les aptitudes de combat ou de souffle, mais il pourrait voler. Durée : 6 tours + le niveau du magicien qui emploie le sort\footnote{Extrait du premier livret de OD\&D, \texttt{Men and Magic} (NdT).}.

\bigskip

\label{sort-delivrance-malediction}\textbf{Délivrance des malédictions} : un sort pour supprimer n'importe quelle malédiction ou envoi maléfique. Noter qu'utiliser ce sort sur une \og épée maudite \fg{}, par exemple, transformerait l'arme en une épée ordinaire et non dans un type de lame enchantée. Portée : adjacent à l'objet\footnote{Extrait du premier livret de OD\&D, \texttt{Men and Magic} (NdT).}.

\bigskip

\label{sort-porte-dimensionnelle}\textbf{Porte dimensionnelle} : un sort de téléportation limitée qui permet à un objet d'être instantanément transporté jusqu'à 11m dans toutes les directions (incluant en haut et en bas). Il n'y a pas de risque de mauvaise évaluation quand on utilise une Porte Dimensionnelle, ainsi l'utilisateur arrive toujours où il souhaite, par exemple 4m au dessus, 11m à l'est, etc. Portée : 30cm\footnote{Extrait du premier livret de OD\&D, \texttt{Men and Magic} (NdT).}.

\bigskip

\textbf{Cinquième niveau :}

\bigskip

\label{sort-teleporter}\textbf{Téléporter} : transport instantané d'un endroit à un autre, quelque soit la distance impliquée, pourvu que l'utilisateur connaisse l'endroit où il veut aller (la topographie de la zone d'arrivée). Sans un savoir certain de la destination, la téléportation est incertaine à 75\%, ce qui implique qu'un score inférieur à 75\% sur un jet de pourcentage implique la mort. Si l'utilisateur connaît la topographie générale de sa destination, mais ne l'a pas étudiée avec attention, le facteur d'incertitude est de 10\% que la téléportation soit trop basse et de 10\% qu'elle soit trop haute. Un score bas (1--10\%) veut dire la mort si un matériau solide est impliqué. Un score élevé (91--100\%) indique une chute de 3 à 30 mètres, ce qui peut aussi causer la mort. Si une étude attentive de la destination a été faite auparavant, alors le magicien possède un risque de 1\% de se téléporter trop bas, et un risque de 4\% de se téléporter trop haut (3--12 mètres)\footnote{Extrait du premier livret de OD\&D, \texttt{Men and Magic} (NdT).}.

\bigskip

