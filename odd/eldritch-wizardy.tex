%+=+=+=+=+=+=+=+=+=+=+=+=+=+=+=+=+=+=+=+=+=+=+=+=+=+=+=+=+=+=+=+=+=+=+=+= PART
%+=+=+=+=+=+=+=+=+=+=+=+=+=+=+=+=+=+=+=+=+=+=+=+=+=+=+=+=+=+=+=+=+=+=+=+= PART
%+=+=+=+=+=+=+=+=+=+=+=+=+=+=+=+=+=+=+=+=+=+=+=+=+=+=+=+=+=+=+=+=+=+=+=+= PART
%+=+=+=+=+=+=+=+=+=+=+=+=+=+=+=+=+=+=+=+=+=+=+=+=+=+=+=+=+=+=+=+=+=+=+=+= PART
\newpage
\phantomsection\addcontentsline{toc}{section}{SUPPLEMENT III -- ELDRITCH WIZARDRY}\begin{center}
{\Huge \ODDtitlefont{DONJONS \& DRAGONS}}{\normalsize \textsuperscript{\sffamily\textregistered}}

\vspace{1.8cm}

{\Large \textbf{Supplément III}}

\vspace{1.3cm}

{\Huge \ODDtitlebisfont{ELDRITCH}}

\vspace{0.3cm}

{\Huge \ODDtitlebisfont{WIZARDRY}}

\vspace{2.0cm}

{\Large \textbf{MAGIE ANCIENNE ET PUISSANTE}}

\vspace{0.5cm}

{\Large \textbf{POUVOIRS PSIONIQUES}}

\vspace{1cm}

{\large PAR

\vspace{0.1cm}

GARY GYGAX \& BRIAN BLUME}
\end{center}

\newpage
\phantom{-}
\newpage
%==========================================================================SECTION
\phantomsection\section*{Hommes \& Magie}
\addcontentsline{toc}{section}{Hommes \& Magie}

\begin{center}
\textbf{[POUVOIRS PSIONIQUES]}
\end{center}

%----------------------------------------------------- SUB SECTION
\phantomsection\subsection*{\normalsize PERSONNAGES :}
\addcontentsline{toc}{subsection}{PERSONNAGES}

{\parindent0pt

Il existe une catégorie spéciale de personnages qui traverse les quatre classes majeures de person-nages-joueurs. Ceux qui possèdent des \textbf{capacités psioniques} peuvent être trouvés parmi les guerriers, les magiciens, les clercs et même les voleurs.

\bigskip

Plus de détails concernant les capacités psioniques et comment déterminer si ce potentiel existe seront trouvés dans la section \textbf{DETERMINATION DES APTITUDES}.

%\bigskip

%Il est important de garder à l'esprit ce qu'est un \og monstre \fg{}. Pour le jeu D\&D, un monstre est toute entité contrôlée par le MD. Les personnages-joueurs et les personnages-non-joueurs contrôlés par les joueurs ne sont pas des monstres : tout le reste, par contre, l'est. Un monstre à D\&D peut être n'importe quoi depuis un démon de Type VI jusqu'à un clerc gentiment Loyal Bon.

\bigskip

Tous les joueurs avec des aptitude psioniques doivent être humains.

\bigskip

Les \textbf{guerriers} sont essentiellement sensibles aux pouvoirs communément connus sous le nom de Yoga. Il y a 20 \og dévotions \fg{} possibles qu'ils peuvent accomplir (les 18 Siddhis et les 2 Sciences) s'ils suivent le développement de leurs talents mentaux. Cependant, pour \textbf{chaque} aptitude qu'ils gagnent, ils doivent perdre un de leurs suivants et un point de Force est perdu de manière permanente pour chaque ensemble de \textbf{quatre} aptitudes. (De plus, ils deviennent sensibles à certains types de monstres et aux attaques de monstres que les personnages sans capacités psychiques ne subissent pas, comme cela sera détaillé plus tard).

\bigskip

Les \textbf{magiciens} qui ont des aptitudes psioniques verront que cela élimine la nécessité d'apprendre certains sorts qui leur donnent fondamentalement les mêmes pouvoirs pour une durée limitée. C'est une chance, car pour chaque aptitude psionique gagnée, le magicien perdra la capacité de se souvenir d'un sort. Cela implique qu'avec le gain de la première aptitude, le magicien sera capable d'utiliser un sort de premier niveau de moins ; quand la deuxième aptitude sera gagnée, il perdra deux niveaux de sorts \textbf{de plus} (soit deux sorts de premier niveau ou un sort de deuxième niveau), et ainsi de suite. Jamais le magicien ne doit se souvenir de plus de sorts de haut niveau que de sorts de bas niveau, et s'il est capable d'utiliser des sorts du sixième niveau, il doit être capable de se souvenir d'au moins un sort de tous les autres cinq niveaux. Les attaques des créatures psioniques seront aussi subies par les magiciens qui développent ce talent.

\bigskip

Les \textbf{clercs} ayant des aptitudes psioniques gagnent aussi l'avantage de pouvoir employer plusieurs pouvoirs \og magiques \fg{}, mais pour chaque aptitude psionique gagnée, le clerc perdra deux de ses autres avantages : primo, il perdra un sort, de la même façon que le magicien ; secundo, le clerc perd la capacité de retourner les morts-vivants à mesure qu'il gagne des pouvoirs  psioniques, de sorte que pour chaque aptitude psionique gagnée, le clerc se place un niveau plus bas dans la capacité de retourner les morts-vivants. Ainsi, un clerc de niveau 10 avec quatre aptitudes psioniques aura perdu 10 niveaux de sorts\footnote{Les pertes de niveaux de sorts sont cumulatives. Pour le gain de 4 aptitudes psioniques : 1 (niveau 1) + 2 (niveau 2) + 3 (niveau 3) + 4 (niveau 4) = 10 niveaux de sorts seront perdus (NdT).} et retournera les morts-vivants comme un clerc de niveau 6. Gagner des aptitudes psioniques rend aussi la personne disposant de ces capacités sensible aux attaques des créatures psioniques.

\bigskip

\textbf{Les moines \& les druides n'ont pas de potentiel psychique : il leur est donc interdit de devenir des personnes aux pouvoirs psychiques.}

\bigskip

Les \textbf{voleurs} qui ont un potentiel psychique avéré sont sujets aux mêmes avantages que ceux gagnés par les guerriers. Néanmoins, en plus des malus notés pour les guerriers, les voleurs perdent aussi 1 point de dextérité toutes les quatre aptitudes gagnées.

%----------------------------------------------------- SUB SECTION
\phantomsection\subsection*{\normalsize DETERMINATION DES APTITUDES :}
\addcontentsline{toc}{subsection}{DETERMINATION DES APTITUDES}

\medskip

Après que les six caractéristiques normales ont été tirées, et que le joueur a choisi un type de personnage, les personnages-joueurs qui disposent d'un score non modifié de 15 ou plus en Intelligence, Sagesse ou Charisme peuvent choisir, en plus, de tester leur capacité psionique, s'ils ont fait le choix d'être humain. La capacité psionique est déterminée en faisant un jet de pourcentage. Un score de 91 ou plus indique que le personnage a cette capacité.

\bigskip

\textbf{Bonus et malus à l'avancement dus aux aptitudes :}

\bigskip

Les personnages-joueurs avancent en niveaux comme indiqué par leur classe et leur caractéristique principale. La capacité psionique, néanmoins, est affectée par le potentiel psychique. Un second jet de dés doit être fait pour déterminer ce niveau, ainsi que les bonus et malus afférents :

\bigskip

{\parindent3cm POTENTIEL PSYCHIQUE

\bigskip

\begin{tabular}{p{3cm}l}
\textbf{Score} & \textbf{Bonus ou malus} \\
\textbf{des dés} & \textbf{Chance de gagner une aptitude} \\
01--10 & --6\%/niveau cumulatif \\
11--25 & --5\%/niveau cumulatif \\
26--50 & --4\%/niveau cumulatif \\
51--75 & Aucun \\
76--90 & +1\%/niveau cumulatif \\
91--99 & +2\%/niveau cumulatif \\
\hspace{0.4cm}00 & +3\%/niveau cumulatif \\
\end{tabular}}

\bigskip

Si un personnage a un malus de --4\%, sa chance de base de gagner une aptitude sera de 6\% au lieu de 10\%\footnote{La chance de base est de 10\% par niveau, de manière cumulative (NdT).}. De même, si son bonus est de 3\%, sa chance de base par niveau sera de 13\%, si bien qu'au niveau 3, la chance de gagner une aptitude psionique sera de 39\%.

\bigskip

\textbf{Bonus} : si une aptitude psionique est gagnée, la chance de gagner immédiatement une seconde aptitude est égale au potentiel psychique du personnage.

%----------------------------------------------------- SUB SECTION
\phantomsection\subsection*{\normalsize NIVEAUX ET POINTS D'EXPERIENCE NECESSAIRES POUR LES ATTEINDRE}
\addcontentsline{toc}{subsection}{NIVEAUX ET POINTS D'EXPERIENCE NECESSAIRES POUR LES ATTEINDRE}

Les personnages-joueurs possédant des aptitudes psioniques progressent de la façon standard dans le type de personnage qu'ils ont choisi au départ. Néanmoins, à partir du niveau 1, ils ont la possibilité d'acquérir une aptitude de type psionique. Les aptitudes psioniques sont listées dans la section \textbf{SORTS}. La probabilité de gagner une aptitude est de 10\% par niveau d'expérience, de sorte qu'un personnage de niveau 1 dispose d'une chance de 10\% d'avoir une aptitude psionique, un personnage de niveau 2 aura 20\% de chances, et ainsi de suite jusqu'au personnage de niveau 10 ayant 100\% de chances.

\bigskip

La sélection de l'aptitude de type psionique est faite aléatoirement, mais si le sort indique une aptitude déjà possédée, il est nécessaire de rejouer  jusqu'à ce qu'une aptitude non possédée par le personnage soit tirée. Quand le personnage dispose de 100\% de chances de gagner une aptitude (10\textsuperscript{ème} niveau), le personnage peut choisir n'importe quelle aptitude quand il gagne un niveau d'expérience.

\begin{center}
\includegraphics[scale=0.17]{./images/demon-typeVI.jpg}
\end{center}

%----------------------------------------------------- SUB SECTION
\phantomsection\subsection*{\normalsize COMBAT PSIONIQUE}
\addcontentsline{toc}{subsection}{COMBAT PSIONIQUE}

Il y a basiquement deux formes d'attaques psioniques : 1) la forme dans laquelle il n'y a pas d'attaque en retour et 2) la forme qui est un échange d'attaques et de défenses où deux créatures avec des aptitudes psioniques sont impliquées. Certains dispositifs magiques ou aptitudes psioniques limitées modifieront le cas 1) ci-dessus. Il est aussi possible que certaines créatures dotées d'aptitudes psioniques aient une forme d'attaque qui affectera uniquement les autres formes de vie dotées de capacités psioniques. Quand le combat psionique se produit, aucune autre action ne peut être effectuée.

\medskip

Les attaques psioniques sur les créatures non-psioniques ne peuvent être faites que si l'attaquant a une force d'attaque psionique de plus de 120. La force d'attaque psionique est déterminée en additionnant le \textbf{potentiel psychique} au nombre d'aptitudes psioniques multiplié par deux et au nombre de modes d'\textbf{attaques psioniques} et de \textbf{défenses psioniques} multiplié par cinq. Par exemple, un personnage avec un potentiel psychique de 37, 6 aptitudes psioniques et 5 modes d'attaques et de défenses aurait une force d'attaque psionique de 73 (37 + 12 + 25 (5x5)). Les dépenses précédentes en points de force psionique sont considérées avec un ratio de 50\%, ce qui fait qu'un usage de 12 points réduit la force d'attaque de 6 points. Les forces d'attaques psioniques des monstres sont exposées dans les paragraphes traitant des monstres dotés de pouvoirs psioniques.

\medskip

Après avoir fait la première attaque, ou dans le cas où les opposants annoncent simultanément qu'ils attaquent de manière psionique (ou dans le cas où le monstre le fait automatiquement et le personnage annonce qu'il le fait), la séquence d'attaque est déterminée comme suit : chaque opposant fait un jet de pourcentage et ajoute le résultat à sa force d'attaque psionique. Le plus haut score attaque en premier.

\bigskip

{\parindent0.5cm
\begin{tabular}{llcll}
\multicolumn{2}{l}{\textbf{Modes d'attaques, toutes classes}} && \multicolumn{2}{l}{\textbf{Modes de défenses, toutes classes}} \\
A. & Onde de choc psionique (20)  && F. & Esprit vide (1) \\
B. & Poussée de l'esprit (10) 	  && G. & Bouclier de pensées (2) \\
C. & Coup de fouet sur l'ego (15) && H. & Barrières mentales (4) \\
D. & Imposition d'identité (10)   && I. & Forteresse intellectuelle (7) \\
E. & Écrasement psychique (25*)   && J. & Tour de volonté de fer (10) \\
&&&& \\
\multicolumn{5}{p{15cm}}{(Le coût d'utilisation en points de force psionique est montré entre parenthèses)} \\
\multicolumn{5}{p{15cm}}{*Si le joueur possède moins de points, altérer la probabilité de succès en \% en conséquence} \\
\end{tabular}}

\medskip

Tout personnage doué de pouvoirs psychiques gagne immédiatement son premier mode d'attaque (onde de choc psionique) dès qu'il gagne sa première aptitude. Les aptitudes devraient être sélectionnées aléatoirement, mais un personnage ne devrait \textbf{jamais} avoir plus d'aptitudes supérieures que d'aptitudes basiques. Dans la sélection aléatoire, il est suggéré de mettre un poids supérieur aux probabilités de gain d'aptitudes liées à des aptitudes déjà possédées, par exemple empathie augmenterait les chances de gagner perception extrasensorielle, télépathie avec les animaux et projection télépathique. Les modes d'attaques additionnels sont gagnés à hauteur de un à chaque quatre aptitudes (cinq pour les guerriers). Les modes de défenses sont gagnés à hauteur de un à chaque trois aptitudes (quatre pour les guerriers).

\bigskip

La force psionique totale est deux fois la force d'attaque psionique (ou la force psionique d'attaque et de défense additionnées).Pour les détails sur la restauration des points de force psionique, se référer à la section FORCE PSIONIQUE. %TODO vérifier que cette section existe

%----------------------------------------------------- SUB SECTION
\phantomsection\subsection*{\normalsize MODES DE D'ATTAQUE ET DE DEFENSE PSIONIQUES}
\addcontentsline{toc}{subsection}{MODES DE D'ATTAQUE ET DE DEFENSE PSIONIQUES}

\begin{tabular}{p{5cm} >{\centering\arraybackslash}p{2.5cm}>{\centering\arraybackslash}p{2.5cm}>{\centering\arraybackslash}p{2.5cm}}
&\multicolumn{3}{c}{\textbf{Portée}} \\
\textbf{Mode d'attaque} & \textbf{Courte} & \textbf{Moyenne} & \textbf{Longue} \\
Écrasement psychique    & 2m & --   & -- \\
Onde de choc psionique  & 1m & 2.5m & 4m \\
Poussée de l'esprit     & 3m & 6m   & 9m \\
Coup de fouet sur l'ego & 2m & 4m   & 6m \\
Imposition d'identité   & 4m & 8m   & 12m \\
\end{tabular}

\medskip

La portée courte augmente de 1/3m (et les autres portées augmentent de la même façon proportionnellement) avec chaque niveau de maîtrise d'une capacité d'attaque.

\medskip

Les attaques à portée moyenne font seulement 80\% des dommages précisés. Les attaques à longue portée font seulement 50\% des dommages précisés.

\bigskip

\begin{tabular}{p{7.5cm}p{6cm}}
\textbf{Mode de défense}    & \textbf{Protection maximale pour} \\
Esprit vide                 & Individu seul \\
Bouclier de pensées         & Individu seul \\
Barrière mentale            & Individu seul \\
Forteresse intellectuelle   & Cercle de 3m autour de l'individu \\
Tour de volonté de fer      & Cercle de 1m autour de l'individu \\
\end{tabular}

\bigskip

Les attaques sur un individu surpris sont gérées dans la MATRICE DES ATTAQUES PSIONIQUES SPECIALES.

\medskip

\begin{tabular}{c>{\centering\arraybackslash}p{1.6cm}>{\centering\arraybackslash}p{1.6cm}>{\centering\arraybackslash}p{1.6cm}>{\centering\arraybackslash}p{1.6cm}>{\centering\arraybackslash}p{1.6cm}>{\centering\arraybackslash}p{1.6cm}>{\centering\arraybackslash}p{1.6cm}}
\textbf{Force} &&&&&& \\
\textbf{d'attaque} & \multicolumn{7}{c}{\textbf{Potentiel psionique du défenseur}} \\
\textbf{psionique} & \textbf{01--10} & \textbf{11--25} & \textbf{26--50} & \textbf{51--75} & \textbf{76--90} & \textbf{91--99} & \textbf{00} \\
01--20      & E & E & 40 & 30 & 20 & 10 & 5 \\
21--40      & E & E & E  & 40 & 30 & 20 & 10 \\
41--60      & B & E & E  & E  & 40 & 30 & 20 \\
61--80      & B & E & E  & E  & E  & 40 & 30 \\
81--90      & H & B & E  & E  & E  & E  & 40 \\
91--00      & H & H & B  & E  & E  & E  & E \\
101--110    & M & H & H  & B  & E  & E  & E \\
111--120    & M & M & H  & H  & B  & B  & E \\
121 et plus & M & M & M  & H  & H  & H  & B \\
\end{tabular}

\bigskip

\begin{tabular}{lp{14.5cm}}
E = & Etourdi pour 5--20 tours, pas d'attaque psionique \\
B = & Blessure psychique, récupération en 1--6 mois, pas d'attaque psionique \\
H = & Handicapé psionique de manière permanente, perd toutes ses aptitudes \\
M = & Mort \\
5--40 = & Nombre de points d'attaque psionique perdus -- récupérés en 1--6 jours \\
Note : & L'attaque Coup de fouet sur l'ego qui donne un résultat "M" veut dire stupidité et le résultat "H" doit être considéré comme "B". L'attaque Imposition psychique avec un résultat de "B", "H", ou "M" signifie que le défenseur est sous le contrôle de l'attaquant jusqu'à ce qu'il soit libéré. \\
\end{tabular}

%----------------------------------------------------- SUB SECTION
\phantomsection\subsection*{\normalsize MATRICE A : ATTAQUE PSIONIQUE SUR NON-PSIONIQUE}
\addcontentsline{toc}{subsection}{MATRICE A : ATTAQUE PSIONIQUE SUR NON-PSIONIQUE}

\begin{tabular}{c>{\centering\arraybackslash}p{2.6cm}>{\centering\arraybackslash}p{2.6cm}>{\centering\arraybackslash}p{2.6cm}l}
\textbf{Intelligence} & \multicolumn{3}{c}{\textbf{Jet de sauvegarde par portée d'attaque}} & \textbf{EFFET SI SAUVE-} \\
\textbf{du défenseur} & \textbf{Courte} & \textbf{Moyenne} & \textbf{Longue} & \textbf{GARDE ECHOUEE} \\
3--4    & 19 & 18 & 17 & Mort \\
5--7    & 17 & 16 & 15 & Coma 1--4 jours \\
8--10   & 15 & 14 & 13 & Sommeil 20--120 min. \\
11-12   & 13 & 12 & 11 & Etourdi 1--4 tours \\
13-14   & 11 & 10 &  9 & Confus 1--6 tours \\
15-16   &  9 &  8 &  7 & Furieux 1--8 tours \\
17      &  7 &  6 &  5 & Esprit affaibli \\
18      &  5 &  4 &  3 & Folie permanente \\
19      &  3 &  2 &  1 & Folie 1--4 semaines \\
20 \& + &  1 &  0 & --1 & Folie 2--12 jours \\
\end{tabular}

\bigskip

\newpage

AJUSTEMENTS AU JET DE SAUVEGARDE :

\medskip

\begin{tabular}{p{4cm}c p{2.8cm} p{6.5cm} r}
\multicolumn{2}{c}{\textbf{Ajouts au dé}} && \multicolumn{2}{c}{\textbf{Soustractions au dé}} \\
Magicien & +1 && Médaillon ESP                          & --5 \\
Clerc    & +2 && Sort relié aux pouvoirs psioniques*    & --4 \\
Elfe     & +2 && Etourdi                                & --3 \\
Nain     & +4 && Confus                                 & --2 \\
Halfling & +4 && Enragé                                 & --1 \\
Heaume télépathique & +4 && Esprit affaibli & ** \\
&&& Fou & *** \\
\end{tabular}

\begin{tabular}{rp{15.2cm}}
\multicolumn{1}{r}{*} & Voir la liste des aptitudes psioniques plus loin pour comparaisons \\
\multicolumn{1}{r}{**} & Traiter un esprit affaibli comme une personne à l'intelligence de 3--4 \\
\multicolumn{1}{r}{***} & Les individus fous ne peuvent être attaqués psioniquement que par l'"Imposition" (voir APTITUDES PSIONIQUES).
\end{tabular}

\medskip

Un heaume de télépathie porté par le défenseur \textbf{étourdira} l'attaquant pour trois tours si le défenseur réussit son jet de sauvegarde.

%----------------------------------------------------- SUB SECTION
\phantomsection\subsection*{\normalsize MATRICE B : COMBAT PSIONIQUE COMPLET, AVEC DOMMAGES}
\addcontentsline{toc}{subsection}{MATRICE B : COMBAT PSIONIQUE COMPLET, AVEC DOMMAGES}

\begin{tabular}{cl>{\centering\arraybackslash}p{2cm}>{\centering\arraybackslash}p{2cm}>{\centering\arraybackslash}p{2cm}>{\centering\arraybackslash}p{2cm}>{\centering\arraybackslash}p{2cm}}
\small\textbf{Force} & & \multicolumn{5}{c}{\small\textbf{Mode défensif}} \\
\small\textbf{psionique} & \small\textbf{Mode} & \small\textbf{Esprit} & \small\textbf{Bouclier} & \small\textbf{Barrière} & \small\textbf{Forteresse} & \small\textbf{Tour de vo-} \\
\small\textbf{totale} & \small\textbf{offensif} & \small\textbf{vide} & \small\textbf{de pensées} & \small\textbf{mentale} & \small\textbf{intellectuelle} & \small\textbf{lonté de fer} \\

01 & Onde de choc      &    2 & 3    & 3 & 1 & 0 \\
   & psionique &&&&&\\
   & Poussée de &   10 & 3    & 0 & 0 & 1 \\
   & l'esprit &&&&&\\
à  & Coup de fouet      &    6 & 2    & 0 & 0 & 0 \\
   & sur l'ego &&&&&\\
   & Imposition &    1 & 4    & 6 & 0 & 1 \\
   & d'identité &&&&&\\
20 & Écrasement & 01\% & -    & - & - & - \\
   & psychique &&&&&\\

21 & Onde de choc      &    3 & 7    & 4 & 2 & 0 \\
   & psionique &&&&&\\
   & Poussée de   &   12 & 5    & 1 & 0 & 3 \\
   & l'esprit &&&&&\\
à  & Coup de fouet      &    8 & 4    & 0 & 0 & 0 \\
   & sur l'ego &&&&&\\
   & Imposition &    2 & 5    & 8 & 1 & 2 \\
   & d'identité &&&&&\\
40 & Écrasement & 02\% & 01\% & - & - & - \\
   & psychique &&&&&\\
\end{tabular}

\begin{tabular}{cl>{\centering\arraybackslash}p{2cm}>{\centering\arraybackslash}p{2cm}>{\centering\arraybackslash}p{2cm}>{\centering\arraybackslash}p{2cm}>{\centering\arraybackslash}p{2cm}}
\small\textbf{Force} & & \multicolumn{5}{c}{\small\textbf{Mode défensif}} \\
\small\textbf{psionique} & \small\textbf{Mode} & \small\textbf{Esprit} & \small\textbf{Bouclier} & \small\textbf{Barrière} & \small\textbf{Forteresse} & \small\textbf{Tour de vo-} \\
\small\textbf{totale} & \small\textbf{offensif} & \small\textbf{vide} & \small\textbf{de pensées} & \small\textbf{mentale} & \small\textbf{intellectuelle} & \small\textbf{lonté de fer} \\


41 & Onde de choc      &    4 & 9    & 5    & 3 & 0 \\
   & Poussée   &   14 & 7    & 02   & 1 & 4 \\
à  & Coup de fouet      &   10 & 6    & 0    & 0 & 0 \\
   & Imposition &    3 & 7    & 10   & 3 & 4 \\
60 & Écrasement & 04\% & 02\% & 01\% & - & - \\

61 & Onde de choc      &    6 & 11   & 7    & 4    & 0 \\
   & Poussée    &   16 & 9    & 4    & 2    & 5 \\
à  & Coup de fouet      &   13 & 9    & 1    & 0    & 1 \\
   & Imposition &    4 & 9    & 13   & 5    & 7 \\
80 & Écrasement & 08\% & 04\% & 02\% & 01\% & - \\

81 & Onde de choc      &    9 & 14   & 9    & 5    & 0  \\
   & Poussée   &   18 & 11   & 6    & 3    & 6  \\
à  & Coup de fouet      &   17 & 13   & 2    & 0    & 2  \\
   & Imposition &    6 & 11   & 16   & 8    & 10 \\
90 & Écrasement & 10\% & 06\% & 04\% & 01\% & -  \\

91  & Onde de choc      &   13 & 17   & 11   & 7    & 1    \\
    & Poussée   &   20 & 13   & 8    & 4    & 7    \\
à   & Coup de fouet      &   22 & 17   & 4    & 1    & 3    \\
    & Imposition &    8 & 14   & 19   & 11   & 13   \\
100 & Écrasement & 12\% & 08\% & 06\% & 02\% & 01\% \\

101 & Onde de choc      &   18 & 20   & 13   & 9    & 2    \\
    & Poussée   &   23 & 15   & 10   & 5    & 8    \\
à   & Coup de fouet      &   28 & 21   & 6    & 2    & 9\footnotemark \\
    & Imposition &   10 & 17   & 23   & 15   & 18   \\
110 & Écrasement & 15\% & 10\% & 08\% & 03\% & 02\% \\

111 & Onde de choc      &   24 & 23   & 15   & 11   & 3    \\
    & Poussée   &   26 & 18   & 13   & 7    & 10   \\
à   & Coup de fouet      &   35 & 27   & 8    & 3    & 6    \\
    & Imposition &   13 & 21   & 27   & 19   & 24   \\
120 & Écrasement & 20\% & 14\% & 10\% & 05\% & 03\% \\

121  & Onde de choc      &   30 & 27   & 18   & 14   & 5    \\
     & Poussée   &   29 & 22   & 17   & 10   & 12   \\
à    & Coup de fouet      &   43 & 33   & 11   & 5    & 8    \\
     & Imposition &   17 & 25   & 31   & 23   & 30   \\
plus & Écrasement & 25\% & 18\% & 13\% & 07\% & 04\% \\

\end{tabular}\footnotetext{Ce chiffre est une faute de frappe. Il ne suit pas la logique de progression des dommages sur cette colonne. Il faut lire 4 ou 5 (NdT).}

\newpage

Un heaume de télépathie porté par le défenseur \textbf{étourdira} l'attaquant pour trois tours si le défenseur réussit son jet de sauvegarde.

\bigskip

Un heaume de télépathie augmente la force psionique de 40.

\bigskip

La table indique le nombre de points de dommages encaissés par les capacités psioniques de l'opposant, excepté la ligne concernant l'ECRASEMENT PSYCHIQUE. Quand cette attaque est tentée, les seules défenses pouvant être utilisées sont BOUCLIER DE PENSEES ou l'ABSENCE de défense, mais si le jet de pourcentage est réussi, l'attaque tue instantanément le défenseur\footnote{Cette indication est incohérente avec le contenu de la table, contenu dans lequel on voit que les cinq types de défenses sont utilisables contre l'écrasement psychique (NdT).}.

\bigskip

Quand un combattant en est réduit à ne plus avoir de capacités défensives, alors toutes les attaques sur lui sont considérées comme devant utiliser la \textbf{Matrice des attaques psioniques spéciales} ci-dessous\footnote{Cette matrice est, de fait, au dessus (NdT).}.

\bigskip

Les capacités psioniques de défense sont les mêmes que la force d'attaque psionique.

\bigskip

La portée courte augmente de 1/3m (et les autres portées augmentent de la même façon proportionnellement) avec chaque niveau de maîtrise d'une capacité d'attaque.

\bigskip

Les attaques à portée moyenne font seulement 80\% des dommages précisés. Les attaques à longue portée font seulement 50\% des dommages précisés.

\bigskip

Les attaques sur un individu surpris sont gérées dans la MATRICE DES ATTAQUES PSIONIQUES SPECIALES.

\bigskip

L'utilisation des pouvoirs psioniques alertera toute créature douée de pouvoirs psioniques dans la portée du pouvoir utilisé, que quelque chose impliquant des pouvoirs psioniques est en train d'arriver. Si le pouvoir est utilisé de manière continue, les probabilités d'identifier la direction et le pouvoir eux-mêmes augmentent. La chance de base est de 10\% pour chaque pouvoir, et la chance augmente de 10\% pour chaque tour d'utilisation de la même aptitude. L'usage d'une aptitude différente rendra l'identification impossible mais pas la direction. Quand la direction est trouvée, la force relative du pouvoir peut être déterminée au tour suivant.

\bigskip

Les aptitudes supérieures alertent les autres créatures psioniques sur une portée double de celle de l'aptitude. Le combat psionique (modes d'attaques) alertent les créatures psioniques sur une portée triple de la capacité psionique (exception faite de Poussée de l'esprit et Imposition d'identité où la détection ne se fait au maximum qu'à la portée de la capacité).

\bigskip

Noter que les sorts qui dupliquent les pouvoirs psioniques ou y sont similaires attireront de même l'attention des créatures psioniques. Cela inclut aussi les objets magiques qui tombent dans cette catégorie.

%----------------------------------------------------- SUB SECTION
\phantomsection\subsection*{\normalsize RESTAURATION DE L'ENERGIE PSIONIQUE}
\addcontentsline{toc}{subsection}{RESTAURATION DE L'ENERGIE PSIONIQUE}

Les points de force psionique dépensés peuvent être restaurés par l'arrêt total de toute activité psionique. La vitesse de restauration dépend du type d'activité non-psionique que le personnage psionique pratiquera :

\begin{center}
\begin{tabular}{cp{0.3cm}c}
\textbf{Activité} && \textbf{Gain de points de force psionique} \\
Marcher, parler \& activités identiques && 6 points/heure \\
Se reposer tranquillement && 12 points/heure \\
Dormir && 24 points/heure \\
\end{tabular}
\end{center}

\newpage
%----------------------------------------------------- SUB SECTION
\phantomsection\subsection*{\normalsize APTITUDES PSIONIQUES}
\addcontentsline{toc}{subsection}{APTITUDES PSIONIQUES}

\textbf{Guerriers (incluant les Paladins et les Rangers) \& Voleurs (incluant les Assassins)}

\bigskip

\begin{tabular}{p{7.5cm}p{0.3cm}p{7.5cm}}
APTITUDES BASIQUES (coût à l'usage) && APTITUDES SUPERIEURES (coût à l'usage) \\
Réduction (aucun) && Contrôle de l'énergie (spécial) \\
Expansion (spécial) && Télékinésie (3/tour) \\
Lévitation (1/tour) && Marche dimensionnelle (spécial) \\
Domination (spécial) && Projection astrale (spécial) \\
Contrôle de l'esprit sur le corps (aucun) && Réarrangement moléculaire (spécial) \\
Invisibilité (2/tour) && Manipulation moléculaire (50) \\
Prémonition (spécial) && Contrôle du corps (5/tour) \\
Hibernation (aucun) && Barrière de l'esprit (aucun) \\
Changer le poids du corps (1/tour) && \\
Clairaudience (2/tour) && \\
Clairvoyance (2/tour) && \\
Corps comme arme (aucun) && \\
\end{tabular}

\bigskip

\textbf{Magiciens (incluant les Illusionnistes)}

\bigskip

\begin{tabular}{p{7.5cm}p{0.3cm}p{7.5cm}}
Détection du Mal/Bien (aucun) && Projection télépathique (3/tour) \\
Détection de la magie (1/tour) && Prémonition (spécial) \\
Perception extrasensorielle (1/tour) && Porte dimensionnelle (10) \\
Hypnose (spécial) && Télékinésie (3/tour) \\
Lévitation (1/tour) && Téléportation (20) \\
Clairaudience (1/tour) && Projection astrale (spécial) \\
Clairvoyance (1/tour) && Forme éthérée (5/tour) \\
Réduction (aucun) && Altération de la forme (spécial) \\
Expansion (spécial) && \\
Agitation moléculaire (2/tour) && \\
\end{tabular}

\bigskip

\textbf{Clercs (incluant les Moines et les Druides\footnote{Incohérence avec la mention du début du livret indiquant que les moines et les druides ne pouvaient pas avoir de pouvoirs mentaux (NdT).})}

\bigskip

\begin{tabular}{p{7.5cm}p{0.3cm}p{7.5cm}}
Détection du Mal/Bien (aucun) && Réarrangement moléculaire (5/tour) \\
Empathie (aucun) && Altération de l'aura (spécial) \\
Lévitation (1/tour) && Prémonition (spécial) \\
Hypnose (1/tour) && Projection télépathique (3/tour) \\
Domination (spécial) && Marche dimensionnelle (spécial) \\
Perception extrasensorielle (1/tour) && Projection astrale (spécial) \\
Ajustement cellulaire (spécial) && Domination des masses (spécial) \\
Contrôle de l'esprit sur le corps (aucun) &&  Voyage probabiliste (spécial) \\
Changer le poids du corps (1/tour) && \\
Télépathie avec les animaux (2/tour) && \\
\end{tabular}

\newpage
%----------------------------------------------------- SUB SECTION
\phantomsection\subsection*{\normalsize EXPLICATION DES APTITUDES PSIONIQUES}
\addcontentsline{toc}{subsection}{EXPLICATION DES APTITUDES PSIONIQUES}

%- - - - - - - - - - - - - - - - - - - - - - - - - - - SUB SUB SECTION
\phantomsection\subsubsection*{Guerriers}
\addcontentsline{toc}{subsubsection}{Guerriers}

\pdfbookmark[4]{Réduction}{ew-reduction}\textbf{Réduction} : la capacité de rendre le corps plus petit en taille. La réduction est approximativement de 1/3 de mètre par niveau à partir duquel l'individu a possédé l'aptitude, de sorte qu'après six niveaux de possession, l'individu peut devenir aussi petit qu'un minuscule insecte.

\bigskip

\pdfbookmark[4]{Expansion}{ew-expansion}\textbf{Expansion} : la capacité pour le corps de devenir plus grand en taille. L'expansion est approximativement de 2/3 de mètre par niveau à partir duquel l'individu a possédé l'aptitude, jusqu'à un maximum de 12 niveaux (croissance additionnelle de 8m). La croissance en masse et en force est proportionnée, de sorte qu'au maximum, la croissance de la force atteint celle d'un géant des tempêtes\footnote{Voir page \pageref{monstre-geant-des-tempetes} (NdT).}. Il est possible de rester à sa taille maximale pendant deux tours, mais chaque niveau en dessous du maximum accroît l'endurance pour un tour, de sorte que si l'expansion potentielle était de 4m, une expansion de seulement 2m permettrait à l'individu de rester à cette taille pour cinq (2 + 3) tours de jeu.

\bigskip

\pdfbookmark[4]{Lévitation}{ew-levitation}\textbf{Lévitation} : De manière similaire à la lévitation magique\footnote{Voir page \pageref{sort-levitation} (NdT).}, cette aptitude permet à l'individu de léviter un tour par niveau de possession de l'aptitude. Ainsi, si l'aptitude a été possédée depuis un niveau, la personne peut léviter un tour : si elle a été possédée depuis deux niveaux, deux tours de plus sont ajoutés, ce qui fait un total de trois ; au troisième niveau de possession, trois tours sont ajoutés, et ainsi de suite.

\bigskip

\pdfbookmark[4]{Domination}{ew-domination}\textbf{Domination} : La capacité de forcer quelqu'un à agir selon votre volonté. L'utilisation de cette aptitude requiert une grande concentration, et elle utilise des points de force psionique à hauteur de un point par niveau de créature dominée par minute de domination. Si la domination requiert le dominé de faire des choses qui sont grandement contre sa volonté, la dépense de points de force psionique est doublée.

\bigskip

\pdfbookmark[4]{Contrôle de l'esprit sur le cor}{ew-controle-esc}\textbf{Contrôle de l'esprit sur le corps} : la capacité de supprimer certains besoins corporels (ou de les satisfaire avec des moyens psioniques) ; nourriture, eau, et sommeil peuvent être complètement ignorés pour deux jours par niveau de possession du pouvoir. Ainsi, une personne ayant possédé l'aptitude depuis deux niveaux peut se passer de dormir, manger ou boire pour une période allant jusqu'à quatre jours. Plus tard, néanmoins, la personne \textbf{doit} passer un nombre de jours équivalent à se reposer pour restaurer son aptitude : un échec à faire cela ne mettra pas à mal le corps, mais l'aptitude ne sera plus utilisable tant qu'un tel repos ne sera pas pris.

\bigskip

\pdfbookmark[4]{Invisibilité}{ew-invisibilite}\textbf{Invisibilité} : cette aptitude permet à l'individu de ne pas être détecté, bien que la personne dans cet état ne puisse pas faire d'actions violentes tant qu'elle est invisible. Pour chaque niveau de possession de l'aptitude, elle est capable d'échapper à un nombre de niveaux équivalent de créatures, soit 1 niveau au premier niveau de possession, 3 niveaux au second niveau de possession,  6 niveaux au troisième niveau de possession, 10 niveaux au quatrième niveau de possession, 15 niveaux au cinquième niveau de possession, et ainsi de suite.

\bigskip

\pdfbookmark[4]{Prémonition}{ew-premonition}\textbf{Prémonition} : la capacité d'estimer la meilleure probabilité de déroulé des événements, ou d'estimer le résultat le plus probable d'actions entreprises. Ce pouvoir ne s'applique qu'au futur immédiat. L'estimation devient plus juste quand le niveau du personnage auquel il a acquis l'aptitude augmente, pourvu que le nombre de facteurs inconnus reste constant. La précision de la prémonition dépend aussi de la
combinaison des scores d'intelligence et de sagesse :

\bigskip

\begin{tabular}{c>{\centering\arraybackslash}p{3.2cm}>{\centering\arraybackslash}p{3.2cm}>{\centering\arraybackslash}p{3.2cm}}
\textbf{Total des scores} & \multicolumn{3}{c}{\textbf{Probabilité de prémonition par difficulté}} \\
\textbf{Intelligence et Sagesse} & \textbf{Faible} & \textbf{Moyenne} & \textbf{Haute} \\
Inférieur à 30 & 40\% & 30\% & 20\% \\
30--33         & 50\% & 35\% & 25\% \\
34--35         & 65\% & 45\% & 35\% \\
36 \& plus     & 70\% & 50\% & 40\% \\
\end{tabular}

\medskip

Pour chaque niveau à partir duquel l'aptitude a été gagnée, la probabilité d'être capable de prédire augmente d'un pourcentage égal au nombre de niveaux, cela de manière cumulative (2 niveaux impliquent 2\%, 3 niveaux 5 \%, etc.) mais jamais au delà d'une probabilité maximale de prédiction de 99\%. La dépense de force psionique est directement reliée au nombre de facteurs inconnus qui doivent être prédits, ce qui veut dire que s'il existe six facteurs inconnus pouvant être basiquement résolus, le coût est de 6 points, et le coût n'est pas connu de celui qui prédit jusqu'à ce que la prédiction soit réalisée. (Afin de prédire les résultats d'une mêlée, par exemple, chaque attaque doit être faite et comptée comme inconnue, et, dans une mêlée impliquant plusieurs individus et plusieurs monstres, le coût par round de mêlée pourrait facilement atteindre ou dépasser 10 points.) Si l'individu ayant des pouvoirs psioniques ne dispose pas de suffisamment de points pour prévoir complètement, alors la prémonition cesse au moment où il n'a plus de force pour continuer. Le temps est aussi un facteur dans la prémonition -- une durée courte implique typiquement un facteur de difficulté faible. Si 1--4 tours peuvent être considérés comme une durée courte, 5--30 tours est de difficulté moyenne et tout ce qui dépasse 30 tours (5 heures) devient une prémonition de difficulté haute ; néanmoins, les facteurs inconnus vont altérer cette règle, de sorte qu'une prémonition court terme avec beaucoup de facteurs inconnus (basiquement insolubles) devient une prémonition de haute difficulté. \textbf{N.B. La prémonition dépend entièrement de l'arbitre, et il doit attacher la plus grande attention à l'usage de cette aptitude.}

\bigskip

\pdfbookmark[4]{Hibernation}{ew-hibernation}\textbf{Hibernation} : cette aptitude permet de suspendre virtuellement toutes les fonctions vitales du corps. L'individu qui dispose de cette aptitude est capable de se "régler" pour se réveiller à un moment dans le futur et de redémarrer ses fonctions. Par niveau de possession de l'aptitude, l'individu est capable d'hiberner pendant une semaine par niveau de manière cumulative (une semaine pour le premier niveau de possession, trois semaines pour le deuxième niveau de possession, etc.).L'individu hibernant ne peut pas être réveillé avant le moment qu'il a lui-même "réglé" pour son réveil. Pour chaque semaine passée en hibernation, l'individu doit passer une journée d'activité normale avant de pouvoir hiberner de nouveau.

\bigskip

\pdfbookmark[4]{Changer le poids du corps}{ew-changer-poids-corps}\textbf{Changer le poids du corps} : cette aptitude permet à l'individu d'ajuster le poids du corps à la surface sur laquelle il marche, de sorte qu'il ne s'y enfonce pas, par exemple dans l'eau, les sables mouvants, la boue, etc. Pour chaque niveau à partir duquel l'individu a possédé cette aptitude, il est capable de changer le poids de son corps une heure par jour.

\bigskip

\pdfbookmark[4]{Clairaudience}{ew-clairaudience}\textbf{Clairaudience} : l'aptitude d'entendre à distance, l'individu possédant ce pouvoir est capable d'entendre ce qui se passe jusqu'à 9 mètres de distance, mais le pouvoir est directionnel. 1/3m de pierre équivaut à 3m d'espace vide. Après chaque niveau auquel l'individu a acquis cette aptitude, ce dernier gagne une distance additionnelle de 3m par niveau cumulatif (au second niveau de possession, cela veut dire 6m, au troisième niveau 9m de plus, ou un total de 24m\footnote{9m + 6m + 9m = 24m (NdT).}, etc.). Ce pouvoir peut être utilisé en conjonction avec une boule de cristal. Il est sujet à des dispositifs d'entraves magiques et non magiques, comme mentionné dans les explications du sort du même nom\footnote{Voir page \pageref{sort-clairaudience} (NdT).}.

\bigskip

\pdfbookmark[4]{Clairvoyance}{ew-clairvoyance}\textbf{Clairvoyance} : comme l'aptitude de clairaudience ci-dessus, excepté que la portée est dix fois supérieure, et au septième niveau de possession, la portée devient illimitée en distance.

\bigskip

\pdfbookmark[4]{Corps comme arme}{ew-corps-comme-arme}\phantomsection\label{ew-corps-comme-arme}\textbf{Corps comme arme} : cette aptitude requiert de la personne qui l'a obtenue de renoncer à l'utilisation de toute arme et armure pour que son corps assume leurs fonctions. L'individu altère psioniquement son corps pour l'endurcir pour frapper ou se défendre. Au premier niveau de possession de l'aptitude, cette aptitude lui donne une classe d'armure de 8, et avec chaque niveau de possession, la classe d'armure s'améliore, ce qui signifie une classe d'armure de 7 au second niveau, de 6 au troisième niveau, etc. L'attaque progresse de manière similaire\footnote{Dans \texttt{OD\&D}, dans le système alternatif (à Chainmail) d'attaques décrit en pages 19 et 20 du volume 1, \texttt{Men \& Magic}, la probabilité de toucher lors d'une attaque ne dépend que du niveau du personnage, pas de son arme. Les dommages sont aussi constants selon les armes (1d6 si pas de modificateurs). Dans le premier supplément \texttt{Greyhawk}, en page 13, un système de modificateurs est proposé pour inclure la gestion des armes dans la probabilité de toucher. Ce système va avec un système de dommages par arme et un système de dommage par monstre. Les détails du pouvoir Corps comme arme utilisent cette extension. Voir page \pageref{combat-alternatif} (NdT). } :

{\parindent2cm\begin{tabular}{cp{2.5cm}c}
\textbf{Niveau de maîtrise} && \\
\textbf{de Corps comme arme} && \textbf{Attaque équivalente à*} \\
premier     && dague \\
deuxième    && hache à main \\
troisième   && masse \\
quatrième   && hache de bataille \\
cinquième   && épée \\
sixième     && épée + 1 \\
septième    && épée + 2 \\
huitième    && épée + 3 \\
neuvième    && épée + 4 \\
dixième     && épée + 5 \\
\end{tabular}}

\bigskip

* La probabilité de toucher due au type d'arme est toujours la plus favorable si l'on considère la classe d'armure qui s'oppose à l'attaque, tandis que les dommages sont donnés par le type d'armes équivalent indiqué, de sorte que celui qui possède l'aptitude depuis trois niveaux frappe comme une dague, une hache à main, ou une masse et inflige les dommages qui sont ceux d'une masse.

\bigskip

Tous les plus sur l'arme équivalente s'appliquent à la probabilité de toucher ainsi qu'aux dommages. Noter que, en ce qui concerne le facteur arme, le Corps comme arme est considéré comme ayant une classe de moins que la dague en ce qui concerne le facteur vitesse, mais la même classe en ce qui concerne la longueur\footnote{Décaler d'une colonne dans la table des modificateurs du système de combat alternatif de \texttt{Greyhawk} si la vitesse entre en jeu, voir \pageref{combat-alternatif} (NdT).}.

\bigskip

\pdfbookmark[4]{Contrôle de l'énergie}{ew-controle-energie}\textbf{Contrôle de l'énergie} : cette aptitude permet à l'utilisateur de canaliser l'énergie dirigée vers lui autour de son corps et de la dissiper. Ainsi, si un sort est dirigé sur lui ou sur l'endroit où il se trouve, il peut utiliser son aptitude pour rendre l'énergie du sort inoffensive. Le coût d'utilisation de cette aptitude de 5 points de force psionique par niveau d'énergie dissipée. (Comme règle générale, considérez chaque dé de dommage qui peut être fait par l'énergie comme un niveau, et si aucun dé de dommage n'est applicable, le niveau du sort peut être utilisé comme mesure de niveau.)

\bigskip

\pdfbookmark[4]{Télékinésie}{ew-telekinesie}\textbf{Télékinésie} : la capacité de bouger les objets par le pouvoir de l'esprit. Le possesseur est capable de bouger un poids de 50 pièces d'or par niveau de maîtrise, cumulatif, ce qui implique qu'au second niveau, il peut bouger un poids (maximum) de 150 pièces d'or, et au troisième un poids de 300 pièces d'or, et ainsi de suite. Le temps pendant lequel il peut faire cela est une fonction de l'énergie psionique.

\bigskip

\pdfbookmark[4]{Marche dimensionnelle}{ew-marche-dimensionnelle}\textbf{Marche dimensionnelle} : la maîtrise de cette aptitude permet à l'individu de se déplacer entre les dimensions pour arriver à un endroit distant en un temps relativement court. Le problème de se perdre en route demeure néanmoins, ce qui implique qu'il faille utiliser la table suivante pour déterminer la durée réelle du déplacement. La durée de base est d'une heure pour 160km de distance :

\bigskip

\begin{tabular}{l>{\centering\arraybackslash}p{2.1cm}>{\centering\arraybackslash}p{2.1cm}>{\centering\arraybackslash}p{2.1cm}>{\centering\arraybackslash}p{2.1cm}>{\centering\arraybackslash}p{2.1cm}}
& \multicolumn{5}{c}{\textbf{Altération du temps par jet de dé}} \\
\textbf{Niveau de maîtrise} & \textbf{1--2} & \textbf{3--5} & \textbf{6--8} & \textbf{9-11} & \textbf{12} \\
premier             & +100\% & +50\% & +25\% & +10\% & 0 \\
deuxième--quatrième & +100\% & +25\% & +10\% & 0     & 0 \\
cinquième--septième &  +50\% & +10\% & 0     & 0     & --10\% \\
huitième et au delà &  +25\% &     0 & 0     & --10\% & --50\% \
\end{tabular}

\bigskip

\pdfbookmark[4]{Projection astrale}{ew-projection-astrale}\textbf{Projection astrale} : cette aptitude est similaire à celle du sort du même nom\footnote{Voir page \pageref{sort-astral} (NdT).}. Quand elle est projetée astralement, la personne ne peut pas être détectée excepté par quelques rares créatures, et son corps astral n'est pas sujet aux dangers habituels. Au premier niveau de maîtrise, le possesseur ne peut avancer qu'au rythme de la marche ; au second, il peut courir aussi vite qu'un petit cheval ; au troisième, il est capable de voler aussi vite qu'un rokh, et la vitesse, après, double avec chaque niveau de maîtrise ; de plus, au dixième niveau de maîtrise, le possesseur de l'aptitude est capable de se projeter dans l'espace à la vitesse de la lumière. Les dangers sont basiquement de deux types : premièrement, il est possible de rencontrer une créature qui peut opérer dans le plan astral (les démons le font, les méduses et les basilics regardent dedans, etc.). Secondement, le corps astral est attaché au corps physique par un cordon d'argent. Si ce cordon est cassé, alors le corps physique et le corps astral meurent. Un vent psychique affecte aussi les personnes projetées astralement comme suit :

\bigskip

{\parindent0.7cm\begin{tabular}{c >{\centering\arraybackslash}p{5cm} >{\centering\arraybackslash}p{5cm}}
\textbf{Niveau de maîtrise de} & \multicolumn{2}{c}{\textbf{Chance pour un vent psychique...}} \\
\textbf{la projection astrale} & \textbf{Emporté} & \textbf{Perte de 1--100 jours} \\
premier            & 08\% & 20\% \\
deuxième           & 07\% & 18\% \\
troisième          & 05\% & 15\% \\
quatrième          & 04\% & 12\% \\
cinquième          & 04\% & 10\% \\
sixième            & 02\% & 07\% \\
septième--neuvième & 01\% & 05\% \\
dixième            & --   & 02\% \\
\end{tabular}}

\medskip

La chance de base qu'un vent psychique souffle dans un rayon de 160km autour du corps physique est de 10\%. La chance qu'un vent psychique souffle au delà de cette distance est de 50\%. Il y a 90\% de chances qu'il y ait un tel vent dans l'espace.

\bigskip

Etre emporté casse le lien d'argent. Perdre entre 1--10 jours se produit lorsque la tentative échoue et que le corps astral est projeté à l'intérieur plutôt qu'à l'extérieur. De 1--100 jours seront perdus en raison de la désorientation due au déchirement de l'esprit. Il n'y a pas de coût psionique pour cette aptitude.

\bigskip

\pdfbookmark[4]{Réarrangement moléculaire}{ew-rearrange-mol}\textbf{Réarrangement moléculaire} : avec cette aptitude, le possesseur est capable d'altérer les molécules des substances métalliques en une autre structure, ainsi les transformant en des métaux différents. Cela, en effet, transmute les métaux, mais ne peut être exécuté qu'une fois par mois au coût de 2 points psioniques par poids de pièce d'or changée. Le poids maximal par niveau de maîtrise est de 10 pièces d'or.

\bigskip

\pdfbookmark[4]{Manipulation moléculaire}{ew-manip-mol}\textbf{Manipulation moléculaire} : la capacité de décaler les arrangement moléculaires de façon à créer une substance de faible résistance. Avec chaque niveau de maîtrise, le possesseur devient plus adepte de la manipulation :

\bigskip

{\parindent0.5cm\begin{tabular}{cc}
\textbf{Niveau de maîtrise} & \textbf{Capable de manipuler l'équivalent de} \\
premier    & cordelettes fines \\
deuxième   & cordes fines \\
troisième  & cordes épaisses ou lanières de cuir\\
quatrième  & câbles \\
cinquième  & chaînes légères \\
sixième    & chaînes lourdes \\
septième   & fers et menottes\\
huitième   & barres de fer, 2.5cm de diamètre\\
neuvième   & barres d'acier, 2.5cm de diamètre \\
dixième    & murs épais de pierre, 2/3m d'épaisseur, trou de la taille d'un homme \\
\end{tabular}}

\bigskip

\pdfbookmark[4]{Contrôle du corps}{ew-controle-corps}\textbf{Contrôle du corps} : la capacité d'adapter le corps à des températures extrêmes ou des éléments destructifs/hostiles (fumées empoisonnées, eau, acide) : cela permet au possesseur de traverser le feu, de respirer sous l'eau, etc., cela pour une durée limitée dépendant du niveau de maîtrise qu'il possède. Comme règle générale, assumer que l'individu est capable de résister à l'équivalent d'un dé de dommages causé par la substance ou l'environnement pendant un tour (dix minutes). Cela implique qu'il pourrait traverser un feu normal ou rester sous l'eau pendant un tour, mais dans un environnement plus hostile, la limite du temps d'exposition serait réduite en conséquence. Pour chaque niveau de maîtrise, le possesseur gagne de manière cumulative une période identique, soit deux périodes au niveau deux, trois périodes au niveau trois, et ainsi de suite. Au dixième niveau de maîtrise, le possesseur aurait 1 + 2 + 3 + 4 + 5 + 6 + 7 + 8 + 9 + 10 = 55 périodes basiques de temps.

\bigskip

\pdfbookmark[4]{Barrière de l'esprit}{ew-barriere-esprit}\textbf{Barrière de l'esprit} : cette aptitude protège le corps physique et l'esprit d'une possession. Elle peut être utilisée quand le corps est abandonné (comme en cas de projection astrale\footnote{Voir page \pageref{sort-astral} (NdT).}) ou à d'autres moments pour le protéger de possessions par urnes magiques\footnote{Voir page \pageref{sort-urne-magique} (NdT).}, démons ou diables. Pour autant, cette aptitude ne fonctionne pas contre des attaques psioniques. La chance pour que le possesseur puisse établir la barrière de son esprit avec succès est de 10\% par niveau de maîtrise. Après le dixième niveau de maîtrise, le pourcentage de chances qu'il soit capable de localiser l'urne ou amulette de l'être qui tente de le posséder croît de la même façon.

%- - - - - - - - - - - - - - - - - - - - - - - - - - - SUB SUB SECTION
\phantomsection\subsubsection*{Magiciens}
\addcontentsline{toc}{subsubsection}{Magiciens}

\pdfbookmark[4]{Détection du Mal/du Bien}{ew-detection-mal-bien}\textbf{Détection du Mal/du Bien} : cette aptitude est simplement le pouvoir de détecter l'aura qui émane de l'esprit des créatures -- ou qui reste sur les objets ou dans les lieux si l'aura est exceptionnellement forte, sinon elle ne fonctionne pas pour les non-sensibles. Il n'y a pas d'utilisation de points de force psionique pour détecter le mal ou le bien.

\bigskip

\pdfbookmark[4]{Détection de la magie}{ew-detection-magie}\textbf{Détection de la magie} : bien que cette aptitude soit similaire en nature aux autres types de détection, la magie opère sur un plan différent, ce qui implique que le possesseur de ce pouvoir soit obligé de dépenser 1 point de force psionique pour chaque tour dans lequel il tente de détecter la magie. Après trois niveaux de maîtrise de l'aptitude, le possesseur possède une chance cumulative de 10\% de déterminer quelle sorte de magie est impliquée (et pas uniquement que certaines forces magiques sont là), soit au quatrième niveau de progression, il possède une chance de 20\% de déterminer la nature basique du sort qui est en cours ou a été lancé.

\bigskip

\pdfbookmark[4]{Perception extrasensorielle}{ew-esp}\textbf{Perception extrasensorielle} : un pouvoir similaire au sort du même nom\footnote{Voir page \pageref{sort-esp} (NdT).}, excepté que la portée est le double de celle du sort, soit 4m. Noter que cette aptitude permet au possesseur d'être \og à l'écoute \fg{} des pensées, et qu'il y a une différence avec le fait de recevoir et de transmettre des pensées de manière télépathique.

\bigskip

\pdfbookmark[4]{Hypnose}{ew-hypnose}\textbf{Hypnose} : l'aptitude ressemble au sort de suggestion des magiciens\footnote{Voir page \pageref{sort-suggestion} (NdT).}, mais il n'affectera pas les personnes très stupides ou hautement intelligentes. Pour chaque niveau de maîtrise, le possesseur est capable de toucher le même nombre de niveaux de créatures. Ainsi, au premier niveau de possession, l'individu n'est capable de toucher qu'une créature de niveau 1 ; au deuxième niveau de maîtrise, le possesseur est capable d'affecter 3 niveaux (1 + 2) ; au troisième niveau de maîtrise, le nombre de niveaux saute à 6 (1 + 2 + 3), et ainsi de suite. Le coût d'utilisation de cette aptitude est de 1 point de force psionique pour chaque niveau de créature affecté. Si l'intelligence de la créature sur laquelle l'aptitude est utilisée est comprise entre 13 et 16, un jet de sauvegarde contre la magie est autorisé, et s'il est réussi, le pouvoir ne la touche pas. La suggestion post-hypnose aura une chance cumulative de 5\% par jour de disparaître.

\bigskip

\pdfbookmark[4]{Lévitation}{ew-levitation-ma}\textbf{Lévitation} : identique à l'aptitude des guerriers.

\bigskip

\pdfbookmark[4]{Clairaudience}{ew-clairaudience-ma}\textbf{Clairaudience} : identique à l'aptitude des guerriers.

\bigskip

\pdfbookmark[4]{Clairvoyance}{ew-clairvoyance-ma}\textbf{Clairvoyance} : identique à l'aptitude des guerriers.

\bigskip

\pdfbookmark[4]{Réduction}{ew-reduction-ma}\textbf{Réduction} : identique à l'aptitude des guerriers.

\bigskip

\pdfbookmark[4]{Expansion}{ew-expansion-ma}\textbf{Expansion} : identique à l'aptitude des guerriers.

\bigskip

\pdfbookmark[4]{Agitation moléculaire}{ew-agitation-mol}\textbf{Agitation moléculaire} : cette aptitude permet au possesseur de faire bouger les molécules d'une chose plus rapidement que la normale. Bien que seulement un petit nombre d'entre elles puisse être affecté, si l'agitation dure pendant dix tours, les effets suivants seront constatés :

\medskip

\begin{tabular}{p{8cm}p{8cm}}
\textbf{Type de matériau}   & \textbf{Effet} \\
Papier, paille              & feu avec flammes vives \\
bois sec                    & le bois se consume \\
chair                       & boursouflures* \\
métal                       & chaud au toucher** \\
\end{tabular}

\bigskip

* A chaque tour, la créature prendra 1 point de dommages, cumulatif (1, 2, 3, 4, etc.) si l'aptitude continue à être utilisée contre lui.

\bigskip

** Devient brûlant comme via le sort Chauffer le métal des druides\footnote{Voir page \pageref{sort-chauffe-metal} (NdT).}, et refroidira à la même vitesse si l'attention de la personne munie de pouvoirs psioniques quitte l'objet.

\bigskip

Même si la quantité de matériau sur laquelle le possesseur de l'aptitude exerce son influence ne change pas avec des niveaux additionnels de maîtrise, le temps requis pour atteindre les effets décrits ci-dessus diminue de un tour tous les niveaux de maîtrise au dessus du premier. Noter que l'objet affecté doit être visible (cela inclut la clairvoyance) de l'individu psionique.

\bigskip

\pdfbookmark[4]{Projection télépathique}{ew-projection-telepathique}\textbf{Projection télépathique} : cette aptitude est assez similaire au pouvoir conféré par le heaume de télépathie\footnote{Voir page \pageref{objet-heaume-telepathie} (NdT).}. L'individu avec cette aptitude est capable d'envoyer des messages télépathiques à n'importe quelle personne  disposant de l'aptitude Perception extrasensorielle (qu'elle soit psionique ou magique). De plus, le possesseur de l'aptitude est capable d'influence un niveau de créature par niveau de maîtrise de l'aptitude. (Ainsi un troisième niveau de télépathie permet à l'utilisateur d'influence six niveaux de créatures, soit six créatures de niveau 1, ou une créature de niveau 6, ou n'importe quelle combinaison des six niveaux.)

\bigskip

Même les créatures basiquement stupides ou hautement intelligentes peuvent être influencées de manière télépathique. La portée du pouvoir est de 2m plus le niveau de maîtrise de l'individu, de manière cumulative (premier niveau = +1/3m, deuxième niveau =+1m, troisième niveau = +2m, quatrième niveau = 3m, etc.). Au dixième niveau, la portée double. Note : un heaume de télépathie double le pouvoir et la portée de l'aptitude et octroie, en plus, au possesseur les effets d'un bonus de +4 sur son intelligence .

\bigskip

\pdfbookmark[4]{Prémonition}{ew-premonition-ma}\textbf{Prémonition} : identique à l'aptitude des guerriers.

\bigskip

\pdfbookmark[4]{Porte dimensionnelle}{ew-porte-dimensionnelle}\textbf{Porte dimensionnelle} : cette aptitude est exactement la même que le sort du même nom\footnote{Voir page \pageref{sort-porte-dimensionnelle} (NdT).}, excepté que l'individu ayant des pouvoirs psioniques dépense des points de force psionique pour accomplir la téléportation limitée.

\bigskip

\pdfbookmark[4]{Télékinésie}{ew-telekinesie}\textbf{Télékinésie} : identique à l'aptitude des guerriers.

\bigskip

\pdfbookmark[4]{Téléportation}{ew-teleportation}\textbf{Téléportation} : l'aptitude est exactement la même que le sort du même nom\footnote{Voir page \pageref{sort-teleporter} (NdT).}, excepté qu'elle coûte de l'énergie psionique à exécuter. Si de l'énergie psionique additionnelle est dépensée, la chance d'arriver trop bas ou trop haut est altérée proportionnellement ; ainsi, si 10 points additionnels sont dépensés dans la téléportation, les risques d'arriver top bas ou trop haut sont réduits de 5\% chacun.

\bigskip

\pdfbookmark[4]{Projection astrale}{ew-projection-astrale-ma}\textbf{Projection astrale} : identique au pouvoir disponible pour les guerriers, excepté que des sorts peuvent être utilisés comme détaillé dans la description du \og sort astral \fg{}\footnote{Voir page \pageref{sort-astral} (NdT).}.

\bigskip

\pdfbookmark[4]{Forme éthérée}{ew-forme-etheree}\textbf{Forme éthérée} : ce pouvoir confère la même aptitude que la potion magique de forme éthérée\footnote{Voir page \pageref{objet-huile-etheree} (NdT).}. De fait, la personne psionique altère les vibrations de son corps pour les aligner avec celles d'un autre plan. Noter que tant que cette aptitude n'est pas maîtrisée depuis plusieurs niveaux, il n'est pas possible de porter beaucoup de choses, car l'état éthéré s'étend uniquement à un poids d'équipement/encombrement de 50 pièces d'or par niveau de maîtrise. Les individus éthérés sont affectés par le vent psychique (détaillé dans le paragraphe Projection astrale) comme suit : la chance que le vent souffle est de 1\%, et cela doit être testé à chaque tour durant lequel l'individu est dans sa forme éthérée. S'il souffle, l'individu éthéré ne sera pas tué, mais la chance qu'il a d'être perdu est \textbf{doublée}, mais, à ce moment, il n'y a plus de dépense de points de force psionique pour rester éthéré, car l'individu est perdu dans le plan et le restera pour le temps décidé par le jet de dés.

\bigskip

\pdfbookmark[4]{Altération de la forme}{ew-alteration-forme}\textbf{Altération de la forme} : ce pouvoir est assez similaire au sort de métamorphose\footnote{Voir page \pageref{sort-metamorphose} (NdT).}. Le possesseur est capable d'altérer sa forme en presque n'importe quoi, mais il n'y a pas de gain correspondant aux caractéristiques de la forme assumée -- ni de perte de compétences de la personne qui a altéré sa forme. Le coût basique est de 5 points d'énergie psionique pour changer sa forme, avec des changements extrêmes en taille, masse ou composition moléculaire, coûtant de manière additionnelle :

\bigskip

\begin{tabular}{p{10cm}l}
\textbf{Exemple d'altération extrême} & \textbf{Coût psionique} \\
Changement de poids de +/-- poids de 1000 pièces d'or & 2 points/1000 \\
Changement en végétal* & 10 points \\
Changement en minéral & 50 points \\
\end{tabular}

\bigskip

* Le changement dans l'autre sens vers le monde animal est chargé de la même façon en points de force psionique.

%- - - - - - - - - - - - - - - - - - - - - - - - - - - SUB SUB SECTION
\phantomsection\subsubsection*{Clercs}
\addcontentsline{toc}{subsubsection}{Clercs}

\pdfbookmark[4]{Détection du Mal/du Bien}{ew-detection-mal-cl}\textbf{Détection du Mal/du Bien} : identique à l'aptitude des magiciens.

\bigskip

\pdfbookmark[4]{Empathie}{ew-empathie}\textbf{Empathie} : cette aptitude permet au possesseur de ressentir les émotions basiques ou les besoins de n'importe quelle créature consciente. C'est-à-dire qu'il peut sentir l'amour, la haine, l'hostilité, la bienveillance, la rage, la peur, la curiosité, le doute, la faim, la soif, et ainsi de suite. La portée de cette aptitude est seulement de 2/3m au premier niveau de maîtrise, mais avec chaque niveau de progression, le possesseur est capable d'étendre son aptitude de 2/3m, si bien qu'au troisième niveau de maîtrise, il peut être empathique dans un rayon de 2m.

\bigskip

\pdfbookmark[4]{Lévitation}{ew-levitation-cl}\textbf{Lévitation} : identique à l'aptitude des guerriers.

\bigskip

\pdfbookmark[4]{Hypnose}{ew-hypnose-cl}\textbf{Hypnose} : identique à l'aptitude des magiciens.

\bigskip

\pdfbookmark[4]{Domination}{ew-domination-cl}\textbf{Domination} : identique à l'aptitude des guerriers.

\bigskip

\pdfbookmark[4]{Perception extrasensorielle}{ew-esp-cl}\textbf{Perception extrasensorielle} : identique à l'aptitude des magiciens.

\bigskip

\pdfbookmark[4]{Ajustement cellulaire}{ew-ajust-cel}\textbf{Ajustement cellulaire} : cette aptitude permet au possesseur de soigner les blessures ou les maladies. Le coût en points de force psionique pour soigner les blessures est de 2 points pour un point de dommages. Le coût pour soigner les maladies est une base de 20 points pour les maladies mineures, et doit être ajusté à la hausse par l'arbitre pour les maladies sévères ou les cas avancés. De plus, le nombre de points de dommages qui peuvent être soignés pendant une période de 24 heures par le possesseur de cette aptitude est dicté par le niveau de maîtrise qu'il possède ; pour chaque niveau, il gagne la capacité de soigner 10 points, ce qui veut dire, par exemple, qu'au second niveau de maîtrise, jusqu'à 20 points de dommages peuvent être soignés (avec un coût en points de force psionique égal à 40, bien entendu).

\bigskip

\pdfbookmark[4]{Contrôle de l'esprit sur le corps}{ew-control-esp-corp-cl}\textbf{Contrôle de l'esprit sur le corps} :  identique à l'aptitude des guerriers.

\bigskip

\pdfbookmark[4]{Changer le poids du corps}{ew-changer-poids-corps-cl}\textbf{Changer le poids du corps} : identique à l'aptitude des guerriers.

\bigskip

\pdfbookmark[4]{Télépathie avec les animaux}{ew-telepathie-animale}\textbf{Télépathie avec les animaux} : cette aptitude donne au possesseur le pouvoir de communiquer avec des créatures conscientes par contact mental direct, mais elle ne permet pas de commander ou d'influencer la créature avec qui est établie la communication. L'aptitude dépend du niveau de maîtrise de la personne dotée du pouvoir :

\bigskip

\begin{tabular}{>{\centering\arraybackslash}p{8cm}>{\centering\arraybackslash}p{7cm}}
\textbf{Niveau de maîtrise} & \textbf{Peut communiquer avec} \\
premier & mammifères \\
deuxième & oiseaux \\
troisième & reptiles \& amphibiens \\
quatrième & poissons et créatures similaires \\
cinquième & insectes \\
sixième & animaux \og monstrueux \fg \\
septième & plantes \\
\end{tabular}

\bigskip

\pdfbookmark[4]{Réarrangement moléculaire}{ew-rearr-mol-cl}\textbf{Réarrangement moléculaire} : identique à l'aptitude des guerriers.

\bigskip

\pdfbookmark[4]{Altération de l'aura}{ew-alter-aura}\textbf{Altération de l'aura} : cette aptitude est étroitement reliée au sort Délivrance des malédictions\footnote{Voir page \pageref{sort-delivrance-malediction} (NdT).}en ce qu'une malédiction placée sur quelque chose ou quelqu'un se distingue facilement par son aura. L'individu possédant cette aptitude est capable de reconnaître l'aura défavorable et de l'altérer, mais le coût de la reconnaissance est de 1 point de force psionique par niveau de malédiction, et l'altération ne peut être faite qu'au coût additionnel de 5 points de force par niveau de malédiction.

\bigskip

\pdfbookmark[4]{Prémonition}{ew-premonition-cl}\textbf{Prémonition} : identique à l'aptitude des magiciens.

\bigskip

\pdfbookmark[4]{Projection télépathique}{ew-proj-tele-cl}\textbf{Projection télépathique} : cette aptitude est étroitement reliée à la projection télépathique des magiciens, excepté que le possesseur est capable d'envoyer des suggestions d'émotions basiques à deux fois le nombre de niveaux de créatures équivalent au nombre de suggestions/messages télépathiques capables d'être envoyés\footnote{Sous-entendu, par le magicien (NdT)}. Le heaume de télépathie\footnote{Voir page \pageref{objet-heaume-telepathie} (NdT).} augmente la capacité télépathique de la même manière qu'il permet de faire de la télépathie.

\bigskip

\pdfbookmark[4]{Marche dimensionnelle}{ew-marche-dim-cl}\textbf{Marche dimensionnelle} : identique à l'aptitude des guerriers.

\bigskip

\pdfbookmark[4]{Projection astrale}{ew-proj-astr-cl}\textbf{Projection astrale} : identique à l'aptitude des magiciens.

\bigskip

\pdfbookmark[4]{Domination des masses}{ew-domination-masses}\textbf{Domination des masses} : cette aptitude permet au possesseur d'utiliser sa capacité de domination sur de multiples individus. Le coût en points de force psionique est le même que celui utilisé pour Domination, mais cette aptitude permet au possesseur d'exercer sa dominance de manière continue après la dépense initiale, et une dépense continuelle n'est pas nécessaire. Néanmoins, la domination de masse ne causera jamais d'acte entièrement contre la volonté collective dans tous les cas de figure. Les possibles niveaux influencés et la durée de cette domination dépendent tous deux du niveau de maîtrise de l'individu qui possède cette aptitude. Pour chaque niveau de maîtrise possédé, l'individu est capable de dominer 5 niveaux de créatures pour deux tours, et au septième niveau de maîtrise, la période de domination devient une semaine entière, et ensuite, la période est étendue d'une semaine par niveau additionnel de maîtrise. Noter que les créatures extrêmement intelligentes ne peuvent pas être dominées, tout comme celles avec des personnalités très fortes ne peuvent pas être dominées avec succès pour une durée quelconque.

\bigskip

\pdfbookmark[4]{Voyage probabiliste}{ew-voyage-proba}\textbf{Voyage probabiliste} : au moyen de cette aptitude, le possesseur est capable de pénétrer dans des mondes parallèles et entrer dans les plans différents. Cela est extrêmement dangereux néanmoins, car cela correspond à de la projection astrale avec l'enveloppe corporelle emportée avec soi. Le vent psychique affecte le voyageur probabiliste, comme s'il se projetait dans l'espace. Pour chaque probabilité ou plan croisé, 10 points d'énergie sont dépensés de manière psionique. Le voyageur est capable d'entrer en communion avec des pouvoirs amicaux, par exemple -- ou risquer d'entrer dans des plans hostiles à son alignement, ou tenter d'explorer les probabilités suivant une ligne de conduite faite sienne.

\newpage

%==========================================================================SECTION
\phantomsection\section*{Note du traducteur}
\addcontentsline{toc}{section}{Note du traducteur}

Ici se termine la partie relative aux pouvoir psioniques extraite du supplément II, \texttt{Eldritch Wizardry}.

\bigskip

A partir de la page suivante, nous reprenons d'autres sections de \texttt{Eldritch Wizardry} tout d'abord, puis des sections de \texttt{OD\&D} et de \texttt{Greyhawk}, sections utiles à la compréhension des pouvoirs psioniques, dans le but que cette traduction n'ait pas besoin de faire référence aux autres livrets de la série, disponibles en américain la plupart du temps.

\bigskip

Les traductions sont intégrales pour les articles sélectionnés mais, parfois, la mise en page ne respecte pas celle des livrets originaux. Lorsque c'est le cas, une indication est placée en note de bas de page.

\newpage
%----------------------------------------------------- SUB SECTION
\phantomsection\subsection*{\normalsize EXPLICATION DES SORTS}
\addcontentsline{toc}{subsection}{EXPLICATION DES SORTS}

\begin{center}
\textbf{[SELECTION]}
\end{center}

%- - - - - - - - - - - - - - - - - - - - - - - - - - - SUB SUB SECTION
\phantomsection\subsubsection*{\textit{Clercs (Druides) :}}
\addcontentsline{toc}{subsubsection}{Magiciens}

\textbf{Deuxième niveau}

\bigskip

\pdfbookmark[4]{Chauffe métal}{pb-chauffe-metal}\phantomsection\label{sort-chauffe-metal}\textbf{Chauffer le métal}\footnote{La présentation de ce sort diffère un peu de la présentation originale pour mettre en exergue le détail des deux tours. Dans la version originale, tout n'est qu'un long paragraphe continu avec des parenthèses ambiguës (NdT).} : ce sort permet aux druides de chauffer progressivement les objets de nature ferreuse en les faisant passer par chaud, très chaud puis brûlant. La quantité de métal pouvant être affectée par ce sort dépend du niveau du druide. Pour chaque niveau atteint, le druide peut affecter approximativement un poids de 200 pièces d'or de métal ferreux. La chair en contact avec de métal chauffé souffrira de blessures et de dommages en conséquence.

\bigskip

La résistance au feu neutralise les effets de ce sort.

\bigskip

Le métal demeure à une température brûlante pendant deux tours :

\bigskip

\begin{itemize}
\item Le premier causera des cloques à une main, la rendant inutilisable pour 1 jour, ou causera 1--2 points de dommages aux autres parties du corps, exceptée la tête qui prendra un point de dommage ce qui étourdira la personne ;
\item Le second causera de graves brûlures à une main qui serait toujours en contact avec le métal, la rendant inutilisable pour 1--3 semaines, causant deux points de dommages additionnels à la tête en contact avec le métal chaud -- et causant la perte de conscience de la créature affectée pendant 2--8 tours.
\end{itemize}

\bigskip

Portée : 1m.

